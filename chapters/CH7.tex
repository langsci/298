\chapter{Simple clauses}
\label{sec:SC}

\todo{replace \hfill in this section to right align text in first line of examples}

In this chapter, I describe the different types of simple clauses in Gyeli. The distinction of simple clause types is based on their internal structure and mainly concerns different types of predicates.  I first outline copula constructions including non-verbal and verbal copula elements in \sectref{sec:nonverbalC}. I then discuss verbal clauses, grammatical relations, and basic clause types in \sectref{sec:verbalC} along with sentential modification. \sectref{sec:IS} is dedicated to information structure phenomena.  In \sectref{sec:specialC}, I discuss special clause types, including questions, possessor raising, and comparison constructions.





\section[Copula constructions]{Non-verbal and verbal copula constructions}
\label{sec:nonverbalC}

Gyeli has copula clauses with both non-verbal and verbal copula constructions. They are typically comprised of a subject, a copula, and a predicate which is sometimes called a `copula complement'.  There are copula forms in some languages, such as {\itshape ni} in Swahili in (\ref{John1}), which are clearly non-verbal as they do not inflect for person, tense, aspect, or mood. In this construction, {\itshape John} is the subject, {\itshape ni} the copula, and {\itshape mkubwa} `big' the predicate.

\begin{exe}
\ex\label{John1} John {\bfseries ni} mkubwa `John is big'
\end{exe}

In English, the copula in (\ref{John}) is a verbal element, although the overall clause structure is the same.

\begin{exe}
\ex\label{John} John {\bfseries is} big
\end{exe}

\noindent \citet[225]{dryer2007a} suggests that, even though the copula {\itshape is} is an inflected form of the verb {\itshape be}, the verb should not be regarded as the predicate since {\itshape tall} takes over the function of a predicate. He notes that: 
\begin{quote} `The verb {\itshape be} is more of a function word than a predicate; its function can be thought of as combining with nonverbal predicates to form what is syntactically a verbal predicate.' (p.225)
\end{quote}

Based on the argument that the clauses in (\ref{John1}) and (\ref{John}) are structurally the same, while the parts of speech status and morphosyntactic behavior of their copula elements differ, I treat both non-verbal and verbal copulas in Gyeli within the same chapter, although in different sections. Another argument for organizing non-verbal and verbal copulas within the same clause type is that the choice of either one in Gyeli often depends on the tense, aspect, mood, and polarity category of the clause. In (\ref{banana1}), a non-verbal copula is used in the present, whereas a verbal copula, an inflected form of {\itshape bɛ̀} `be', has to be used in (\ref{banana2}) for its negated version and in (\ref{banana3}) for the past.

\begin{exe}
\ex\label{banana}
\begin{xlist}
\ex\label{banana1}
  \glll lènjù {\bfseries léè} nábèbè   \\
        le-njù léè nábèbè  \\
          le5-banana 5.COP red  \\
    \trans `The banana is red.'
\ex\label{banana2}
  \glll lènjù {\bfseries lé} {\bfseries bɛ́lɛ́} nábèbè   \\
        le-njù le-H bɛ̀-lɛ nábèbè  \\
          le5-banana 5-PRES be-NEG red  \\
    \trans `The banana is not red.'
\ex\label{banana3}
  \glll lènjù {\bfseries lè} {\bfseries bɛ́} nábèbè   \\
        le-njù le bɛ̀-H nábèbè  \\
          le5-banana 5.PST1 be-R red  \\
    \trans `The banana was red.'
\end{xlist}
\end{exe}

This is in line with \posscitet{dryer2007a} observation that copula constructions differ structurally and cross-linguistically in different respects. First, as (\ref{banana}) shows, the grammatical status of the copula can differ, even within the same language. According to \citet[225-227]{dryer2007a}, non-verbal copulas have cross-linguistically different morphosyntactic shapes, ranging from words to clitics and affixes. 

Second, Dryer points out that there are three types of predicates, namely adjectival, nominal, and locative predicates. 
Semantically, copula constructions encode two different types of relations which are, according to \citet[1-2]{curnow2001}, identity relations and classifications, as exemplified in (\ref{man}).

\begin{exe}
\ex\label{man} 
\begin{xlist}
\ex Identity: `That man is my father.'
\ex Classification: `That man is a teacher.'
\end{xlist}
\end{exe}

In Gyeli, both identity and classification relations are expressed by copula constructions. Gyeli copula constructions differ in the type of predicate and the type of copula. The predicate ranges from nominal to locative and adjective/quantifier (the equivalent to adjectival predicates in other languages) predicates. Also, demonstratives and possessive pronouns can serve as predicates as well as deictic elements, as I will show for the various copula types below.

Gyeli has six different copula types, three of which are non-verbal and three verbal, as shown in Table \ref{Tab:COP}. The non-verbal copula types can only be used in affirmative clauses which occur in the \textsc{present}. The most frequent copula in the corpus is the STAMP copula that is expressed by a special STAMP form. It merges the subject and the copula in one morpheme and constitutes the most frequent of all copula constructions found in the corpus (43.7\%). Another non-verbal copula is the invariable identificational marker {\itshape wɛ́} which represents 11.6\% of the copular clauses. There are also instances where the copula is zero-expressed. This construction, however, is only found in elicitations and does not occur in the corpus. All non-verbal copulas are restricted to the \textsc{present} tense-mood category. If other tense-mood categories are to be encoded, as well as negation, the verbal copula {\itshape bɛ̀} `be' is used.

\begin{table}[!h]
\centering
%\scalebox{0.9}{
\begin{tabular}{l|llll}
 \midrule
Status                                &  Copula element & Label                      &  \multicolumn{2}{l}{Corpus frequency} \\
 \midrule
\multirow{3}{*}{non-verbal} & STAMP form        & STAMP copula (COP) & 49    & (43.7\%) \\
                                     & {\bfseries wɛ́}                    & identificational (ID) & 13   & (11.6\%)  \\
                                     & $\emptyset$-copula &                                &  0    &   \\  \midrule
\multirow{3}{*}{verbal} & {\bfseries bɛ̀} `be'               &                                 & 27   & (24.1\%) \\
                                   & {\bfseries múà} `be almost'             &                                 &   6    & (5.4\%) \\
				& {\bfseries bùdɛ́} `have'        &                                 &  17   & (15.2\%)   \\ 
 \midrule
Total                           &                                   &                                 &   112        & \\
 \midrule
\end{tabular}
\caption{Copula types}
\label{Tab:COP}
\end{table}

Two of the verbal copulas are forms of `be': {\itshape bɛ̀} and {\itshape múà}. One is the more general and more frequent {\itshape bɛ̀} (24.1\% of all copula constructions in the corpus) and one is {\itshape múà} (5.4\%) which is also used as the \textsc{prospective} auxiliary (\sectref{sec:PROSP}). {\itshape bùdɛ́} `have' is the third verbal copula. It covers 15.2\% of all copular constructions and is mostly used in predicate possession of the \textsc{present}.

I will describe each copula type in the following, providing examples and information on its distribution. This will also show that not every copula behaves like a real copula element in every context, i.e.\ linking a subject to a copula complement. In some cases, some copula elements also take over functions such as presentational or existential markers which do not require a predicate and thus are then not strictly speaking copulas in all contexts.


\subsection{STAMP copula}
\label{sec:COP}

The STAMP copula (COP) takes a  special form of the STAMP marker which is identical to the STAMP marker of the \textsc{future} tense-mood category, as discussed in \sectref{sec:fut}. It has a long vowel with a default HL tonal pattern for all agreement classes and speech act participants, except for the first and second person singular and agreement class 1 where the long vowel takes a L tone. 

\paragraph{Predication types} Unlike all other copula types, the STAMP copula agrees with the subject in gender. The STAMP copula can link a nominal subject to different predicatation types. In (\ref{ID1a}), the predicate is nominal, expressing a classification relation: Ada is a member of the set of teachers.  


\begin{exe}
\ex\label{ID1a}
  \glll Àdà {\bfseries àà} ngɛ̀lɛ́nɛ̀ \hfill [nominal] \\
        Àdà àà ngɛ̀lɛ́nɛ̀   \\
          $\emptyset$1.PN 1.COP $\emptyset$1.teacher  \\
    \trans `Ada is a teacher.'
\end{exe}

\noindent (\ref{ID1b}) and (\ref{IDcol}) provide examples where the predicate is an adjective.

\begin{exe}
\ex\label{ID1b}
  \glll Àdà {\bfseries àà} mpà \hfill [adjective] \\
        Àdà àà mpà  \\
          $\emptyset$1.PN 1.COP good  \\
    \trans `Ada is good.'
\end{exe}

\begin{exe} 
\ex\label{IDcol}
  \glll bon mpɔ̀ngɔ̀ sílɛ̃́ɛ̃̀ nà {\bfseries béè} bànáyɛ̂yɛ̂ \hfill [adjective] \\
        bon, mpɔ̀ngɔ̀ sílɛ̃́ɛ̃̀ nà béè ba-náyɛ̂yɛ̂  \\
      good[French] $\emptyset$7.generation finish.COMPL COM 2\textsc{pl}.COP 2-bleached.out  \\
    \trans `Good, the generation has been wiped out, and you are bleached out [= white].'
\end{exe}


\noindent  In (\ref{ID1c}) and (\ref{ID1d}), the predicate is a locative noun phrase.

\begin{exe}
\ex\label{ID1c}
  \glll Àdà {\bfseries àà} ndáwɔ̀ dé tù \hfill [locative]\\
        Àdà áà ndáwɔ̀ dé tù  \\
          $\emptyset$1.PN 1.COP $\emptyset$9.house LOC inside  \\
    \trans `Ada is inside the house.'
\end{exe}


\begin{exe} 
\ex\label{ID1d}
  \glll  bɔ́nɛ́gá {\bfseries báà} ná jìí dé tù \hfill [locative]\\
        b-ɔ́nɛ́gá báà ná jìí dé tù \\
          2-other 2.COP still $\emptyset$7.forest LOC inside  \\
    \trans `The others are still in the forest.'
\end{exe}


\noindent In addition to these predicate types which \citet{dryer2007a} views as the most common ones across languages, the STAMP copula in Gyeli can also be used with locative interrogative words as in (\ref{ID2}) and with deictic elements, as in (\ref{ID3}).

\begin{exe} 
\ex\label{ID2}
  \glll ɛ́ nà mwánɔ̀ {\bfseries nùù} vɛ́ \hfill [interrogative]\\
       ɛ́ nà m-wánɔ̀ nùù vɛ́ \\
       LOC how N1-child 1.COP where  \\
    \trans `What! Where is the child?'
\end{exe}


\begin{exe} 
\ex\label{ID3}
  \glll bã́ yɔ́ɔ̀ {\bfseries yíì} tè \hfill [deictic]\\
        bã́ y-ɔ́ɔ̀ yíì tè \\
         $\emptyset$7.word 7-POSS.2\textsc{sg}  7.COP there \\
    \trans `Your word is there [= I understand you].'
\end{exe}

Also numerals and quantifiers can serve as the copula complement, as in (\ref{IDnum}).

\begin{exe} 
\ex\label{IDnum}
  \glll   búdì báà bàbáà/bvùbvù \hfill [numeral/quantifier] \\
         b-údì báà ba-báà/bvùbvù \\
            ba2-person 2.COP 2-two/many \\
    \trans `The people are two/many.'
\end{exe}



Finally, the STAMP copula can also introduce reported speech in a quotative index. Thus, in (\ref{IDRS}), the STAMP copula {\itshape báà} serves as quotative index to the direct reported speech in the copula complement, marked by square brackets.


\begin{exe} 
\ex\label{IDRS}
  \glll  {\bfseries báà} nâ [wɛ̀ sílɛ̂ kɛ̀ sâ sálɛ́] \hfill [complement]\\
        báà nâ wɛ sílɛ̂ kɛ̀ sâ sálɛ́ \\
        2.COP COMP 2\textsc{sg} finish.IMP go do $\emptyset$7.work  \\
    \trans `They are like `you, finish go do the work'.'
\end{exe}

\paragraph{STAMP copula as the predicate} In the vast majority of cases, the STAMP copula functions as element linking the subject to the predicate. In a few special cases, however, there is no copula complement and the STAMP marker serves as predicate, as in (\ref{ID4}) and (\ref{ID5}) which represent existential clauses. According to \citet[241]{dryer2007a}, 

\begin{quote} ``From a discourse point of view, the primary function of such [existential] clauses is apparently to introduce into the discourse a participant that is new to the hearer.'' \end{quote}
In English, this is often achieved with constructions involving {\itshape there is} or {\itshape there are}.

\begin{exe} 
\ex\label{ID4} 
  \glll bèsá bíndɛ̀ byɛ́sɛ̀ {\bfseries béè} ndáà \\
       be-sá bí-ndɛ̀ by-ɛ́sɛ̀ béè ndáà \\
        be8-thing 8-ANA 8-all 8.COP also \\
    \trans `There are also all these things. [= way of introducing a problem]'
\end{exe}

\begin{exe} 
\ex\label{ID5} 
  \glll lé [yá wɛ́ nyɛ̂]\textsubscript{REL} [bá gyíbɔ́ ngàlɛ́]\textsubscript{REL} {\bfseries yíì} \\
        lé yá wɛ-H nyɛ̂ ba-H gyíbɔ-H ngàlɛ́ yíì \\
       $\emptyset$7.tree 7:ATT 2\textsc{sg}-PRES see 2-PRES call-R $\emptyset$7.PN 7.COP   \\
    \trans `There is the tree that you see that they call {\itshape ngàlɛ́}.'
\end{exe}




\paragraph{Expression of the subject} As mentioned above, a copula links a subject to a predicate. In the previous examples, the shape of the subject was some sort of noun phrase. In (\ref{ID2}) and (\ref{ID5}), the subject is expressed nominally while the subject noun phrase in (\ref{ID4}) is more complex, including two modifiers. The STAMP copula can also encode subject and copula at the same time and thus can occur on its own, without a nominal noun phrase, as in (\ref{ID6}).


\begin{exe} 
\ex\label{ID6}
  \glll mɛ̀ɛ́ lémbòlɛ̀ ɛ́ mpù {\bfseries báà} ndáwɔ̀ dé tù dénè \\
      mɛ̀ɛ́ lémbo-lɛ̀ ɛ́ mpù báà ndáwɔ̀ dé tù dénè \\
        1\textsc{sg}.PRES.NEG know-NEG LOC like.this 2.COP $\emptyset$9.house LOC inside today[Bulu]  \\
    \trans `I don't know how they are in the house today.'
\end{exe}


\noindent This construction type is also used in generic `it is' clauses where the subject is inanimate, but underspecified, as for instance in (\ref{ID7}).

\begin{exe} 
\ex\label{ID7}
  \glll {\bfseries yíì} mpà yɔ̃́ɔ̃̀ wɛ́ kã́ yɔ̂ dúmbɔ́ \\
       yíì mpà yɔ̃́ɔ̃̀ wɛ-H kã̂-H y-ɔ̂ dúmbɔ́ \\
         7.COP good $\emptyset$7.time 2\textsc{sg}-PRES wrap-R 7-NSBJ $\emptyset$7.package\\ 
    \trans `It is good when you wrap it in a (leaf) package.'
\end{exe}

\noindent The {\itshape yíì} STAMP copula is also used in cleft sentences, as shown in \sectref{sec:cleft}.




%\paragraph{Semantic range of STAMP copula}
%Having described the structural properties of subjects and predicates involved in copula constructions with the STAMP copula, I now turn to presenting examples of the STAMP copula's uses from a semantic perspective. As pointed out above in (\ref{ID4}) and (\ref{ID5}), the STAMP copula is used in existential clauses. This is also the case, when the predicate is a deictic element as in (\ref{ID8}) and (\ref{ID9}). Both examples can be interpreted as existential or locative, depending on the context.


%\begin{exe} 
%\ex\label{ID8}
%  \glll bèkɔ́kɔ́ bé nlô bé tè {\bfseries béè} tè \\
 %      be-kɔ́kɔ́ bé nlô bé tè béè tè \\
 %       be8-hollowness 8:ATT $\emptyset$3.head 8:ATT there 8.COP there \\
%    \trans `The skulls there are there.'
%\end{exe}

%\begin{exe} 
%\ex\label{ID9}
 % \glll  {\bfseries yáà} ndáà vâ \\
 %       yáà ndáà vâ \\
 %         1\textsc{pl}.COP also here \\
 %   \trans `We are also here.'
%\end{exe}

%Also, the STAMP copula expresses equational relations, as in (\ref{ID10}). \citet[233]{dryer2007a} notes that true equational clauses are those where the subject and predicate can be reversed, which is true for (\ref{ID10}). 


%\begin{exe} 
%\ex\label{ID10}
%  \glll  jínɔ̀ lɛ́ kwàdɔ̀ yã̂ {\bfseries yíì} Ngòló \\
  %       j-ínɔ̀ lɛ́ kwàdɔ̀ y-ã̂ yíì ngòló \\
   %           le5-name 5:ATT $\emptyset$7.village 7-POSS.1\textsc{sg} 7.COP $\emptyset$3.PN  \\
   % \trans `The name of my village is Ngolo.'
%\end{exe}


%\noindent In contrast, clauses where subject and (nominal) predicate cannot be reversed, are termed `true nominal predicates' by Dryer. (\ref{ID11}) provides an example of such a clause.


%\begin{exe} 
%\ex\label{ID11}
 % \glll {\bfseries béè} bùdì bá vúdũ̂ ndí bwáá gyɛ́sɔ́ mápè'è \\
  %      béè b-ùdì bá vúdũ̂ ndí bwáa-H gyɛ́sɔ-H H-ma-pè'è \\
 %        2\textsc{pl}.COP ba2-person 2:ATT one but 2\textsc{pl}-PRES search-R OBJ.LINK-ma6-wisdom \\
 %   \trans `You (pl) are the same people, but you are looking for wisdom.'
%\end{exe}


%Finally, the STAMP copula can also express predicate possession,  as in (\ref{LocPred}). In this example, the possessor {\itshape mbúmbù} precedes the possessee {\itshape lèbvúú} which serves as the subject. The STAMP copula agrees as expected with the subject in gender and is followed by the predicate which is a locative in this case.


%\begin{exe} 
%\ex\label{LocPred} 
 % \glll mbúmbù lèbvúú {\bfseries léè} nlémò dé \\
  %       mbúmbù le-bvúú léè nlémò dé \\
 %       N1.namesake le5-anger 5.COP $\emptyset$3.heart LOC  \\
  %  \trans `The namesake is anger in the heart (he is angry).'
%\end{exe}












\subsection{Identificational marker {\itshape wɛ́}}
\label{sec:ID}

The identificational marker {\itshape wɛ́} is invariable and does not agree with the subject. The marker occurs in two types of constructions. The primary use is as a copula, linking a subject and a predicate, as in (\ref{we1}). 

\begin{exe}
\ex\label{we1}
  \glll ntɛ́mbɔ́ wã̂ {\bfseries wɛ́} nû \\
        ntɛ́mbɔ́ w-ã̂ wɛ́ nû  \\
          $\emptyset$1.younger.brother 1-POSS.1\textsc{sg} ID 1.DEM.PROX  \\
    \trans `My younger brother is this.'
\end{exe}

In contrast to the STAMP copula, however, {\itshape wɛ́} links a subject only to demonstratives and anaphoric markers. This is why I label {\itshape wɛ́} as identificational marker.
As \citet[1812]{mikkelsen2011} states for English, ``[i]dentificational clauses are characterized by having a demonstrative pronoun or demonstrative phrase in the subject position.'' In Gyeli, the demonstrative does not occur in the subject, but in the predicate position. Nevertheless, I label {\itshape wɛ́} as an identificational marker since it takes over the same function, namely identifying people, places, and the location of things. In (\ref{we1}), the speaker identifies his younger brother by using a deictic demonstrative, at the same time pointing to the person in question. In (\ref{we2}), the chief of Ngolo talks about a scar on his forehead, identifying its location and again pointing to it.


\begin{exe} 
\ex\label{we2}
  \glll     mɛ́ bvú nâ bàmó tè yɔ́ɔ̀ {\bfseries wɛ́} yî \\
          mɛ-H bvû-H nâ bàmó tè y-ɔ́ɔ̀ wɛ́ yî \\
              1\textsc{sg}-PRES think-R COMP $\emptyset$7.scar there 7-NSBJ ID 7.DEM.PROX \\
    \trans `I think, the scar there is this.'
\end{exe}

Apart from demonstratives, anaphoric elements may also occur with the identificational marker {\itshape wɛ́}. This can be the bare anaphoric marker {\itshape ndɛ́} as in (\ref{we3}) which does not take an agreement prefix. 

\begin{exe} 
\ex\label{we3}
  \glll kàndá {\bfseries wɛ́} ndɛ̀ \\
        kàndá wɛ́ ndɛ̀ \\
        $\emptyset$7.proverb ID ANA  \\
    \trans `The story is this.'
\end{exe}

\noindent Also, the anaphoric marker with an agreement prefix occurs in identificational constructions, as shown in (\ref{we4}).

\begin{exe} 
\ex\label{we4}
  \glll bã̂ yã̂ màfwálá {\bfseries wɛ́} yíndɛ̀ \\
        bã̂ y-ã̂ ma-fwálá wɛ́ yí-ndɛ̀ \\
         $\emptyset$7.word 7-POSS.1\textsc{sg} ma6-end ID 7-ANA \\
    \trans `This is my last word.'
\end{exe}


The second type of construction where {\itshape wɛ́} is used is without a predicate. In (\ref{we5}), the parentheses indicate that the use of the demonstrative is optional. Often, the demonstrative is not expressed, so that only the subject and {\itshape wɛ́} surface. In that sense, {\itshape wɛ́} is not a real copula here since it does not link a subject to another constituent. It has its origin, however, in a copula construction. Environments where {\itshape wɛ́} is used phrase-finally, i.e.\ without demonstrative or anaphoric marker, are usually those where the subject is a personal pronoun as in (\ref{we5}).


\begin{exe} 
\ex\label{we5} 
  \glll  nyɛ̀ {\bfseries wɛ́} (nû) \\
         nyɛ wɛ́ (nû)\\
       1.SBJ ID  (1.DEM.PROX)\\
    \trans `This is him.'
\end{exe}

Such identificational constructions show a particular structure when they involve a proper name, as in (\ref{we6}). Here, the personal pronoun is followed by the proper name and the identificational marker {\itshape wɛ́} occurs phrase-finally. They differ from the above examples in that {\itshape wɛ́} is not a linking element, but rather functions as a deictic itself. In this view, it is not surprising that proper name constructions with {\itshape wɛ́} do not involve demonstratives or anaphorics. 

\begin{exe} 
\ex\label{we6} 
  \glll  mhm mɛ̀ Nzìwù {\bfseries wɛ̂} \\
         mhm mɛ Nzìwù wɛ́ \\
       EXCL 1\textsc{sg}.SBJ $\emptyset$1.PN  ID  \\
    \trans `Mhm, I'm Nziwu.'
\end{exe}


Finally, {\itshape wɛ́} is also used in cleft constructions, as shown in (\ref{we7}). The structure of the identificational clause is parallel to the one in (\ref{we5}) without a demonstrative predicate, namely {\itshape nyɛ̀ wɛ́}, except that the subject is more complex, specifying who {\itshape nyɛ̀} is. The identificational clause is  followed by a relative clause which, in this case, does not have an attributive marker to indicate the relative clause.\footnote{For more information on relative clauses, see \sectref{sec:Relativeclauses}.}


\begin{exe} 
\ex\label{we7} 
  \glll ntɛ́mbɔ̀ wà mùdã̂ wã̂ nyɛ̀ {\bfseries wɛ́} [bùdɛ́ mwánɔ̀ wà mùdã̂ mvúdũ̂]\textsubscript{REL} \\
       ntɛ́mbɔ̀ wà m-ùdã̂ w-ã̂ nyɛ wɛ́ bùdɛ-H m-wánɔ̀ wà m-ùdã̂ m-vúdũ̂ \\
        $\emptyset$1.younger.sibling 1:ATT N1-woman 1-POSS.1\textsc{sg} 1 ID have-R N1-child 1:ATT N1-woman 1-one    \\
    \trans `It's my wife's younger sister who has one girl.'
\end{exe}

\noindent As with all other non-verbal copula types, also {\itshape wɛ́} is restricted to the \textsc{present} tense-mood category.



\subsection{Optional $\emptyset$-copula}
\label{sec:0COP}

% todo all copulas or only specific ones can be omitted?
In a few environments, a copula can be optionally omitted. This, however, seems to be restricted to semantic relations of identity between the subject and the predicate. Copula omission in Gyeli is grammatically optional and not grammatically conditioned, even though certain environments seem to favor omission. In all examples presented below, a copula could also be used. Environments which favor copula omission often seem to involve genitive predicates, as in (\ref{0COP1}) and (\ref{0COP2}). Both examples differ, however. In (\ref{0COP1}), the subject is a demonstrative while the predicate is a nominal noun phrase, modified by a possessive pronoun.
The clause could also be expressed with a STAMP copula: {\itshape núù mwánɔ̀ wã̂.} Since examples of copula omission are rare, the sample is not sufficient to make any generalizations about the difference between the use of a STAMP copula in contrast to copula omission. It may be a matter of fast and colloquial speech to omit the copula. It may also be related to information structure. The bare demonstrative as subject, as in (\ref{0COP1}), could thus introduce a new topic, while the STAMP copula may suggest that the topic is already known.\footnote{It is also possible to use the identificational marker {\itshape wɛ́} for (\ref{0COP1}), but in that case, subject and predicate would need to be reversed, making the predicate {\itshape mwánɔ̀ wã̂} the subject and {\itshape nû} the predicate. This construction then differs also in terms of information structure, moving the demonstrative into focus position.}

\begin{exe}
\ex\label{0COP1}
  \glll     nû [mwánɔ̀ wã̂]\textsubscript{PRED}  \\
             nû m-wánɔ̀ w-ã \\
               1.DEM.PROX N1-child 1-1\textsc{sg}.POSS  \\
    \trans `This is my child.'
\end{exe}

In contrast to (\ref{0COP1}), the predicate in (\ref{0COP2}) is a possessive pronoun while the subject is a complex nominal noun phrase, including a demonstrative. Again, it is possible to use a copula, for instance the STAMP copula {\itshape wúù} of agreement class 3, which is deleted in fast speech. 

\begin{exe}
\ex\label{0COP2}
  \glll nkwànɔ̀ wɔ̂ [wã̂]\textsubscript{PRED}   \\
        nkwànɔ̀ wɔ̂ w-ã̂   \\
          $\emptyset$3.honey 3.DEM.PROX 3-1\textsc{sg}.POSS  \\
    \trans `This honey is mine.'
\end{exe}

In addition to genitive predicates, a copula can also be omitted in nominal predication when the subject is a personal pronoun, as in (\ref{0COP3}).


\begin{exe}
\ex\label{0COP3}
  \glll mɛ̀ [nsálɛ̀ gyàngó]\textsubscript{PRED}   \\
        mɛ n-sálɛ̀ gyàngó \\
          1 N1-doer $\emptyset$7.hunt  \\
    \trans `I'm a hunter.'
\end{exe}

\noindent Zero copula constructions always refer to the \textsc{present} tense. If non-verbal predicates are to be expressed in other tense-mood categories, a verbal copula is required. 







\subsection{Verbal copula {\itshape bɛ̀} `be'}
\label{sec:COPbe}

To express copular clauses in other tense-mood categories or to negate them, the verbal copula {\itshape bɛ̀} `be' is used. Additionally, {\itshape bɛ̀} is used in expressing predicate possession by adding the comitative marker {\itshape nà}. Each of these uses is illustrated below.

\paragraph{Tense expression with {\itshape bɛ̀} `be'} The verbal copula {\itshape bɛ̀} can be used in all tense-mood categories. Even though for the \textsc{present} tense-mood category, usually non-verbal copula types are used, also {\itshape bɛ̀} can serve as copula in the \textsc{present}. This seems to mainly occur when the subject is an emphatic pronoun, as in (\ref{beEMPH1}) and (\ref{beEMPH2}). 


\begin{exe} 
\ex\label{beEMPH1} 
  \glll  lûngà yá sã́ wã̂ yɔ́ {\bfseries bɛ́} yíí \\
         lûngà yá sã́ w-ã̂ yɔ́ bɛ̀-H yíí \\
          $\emptyset$7.grave  7:ATT $\emptyset$1.father 1-POSS.1\textsc{sg} 7.EMPH be-R 7.DEM.DIST \\
    \trans `My father's grave is over there.'
\end{exe}


\begin{exe} 
\ex\label{beEMPH2}
  \glll   ngùndyá tè nyɔ́ {\bfseries bɛ́} nyî \\
          ngùndyá tè nyɔ́ bɛ̀-H nyî \\
           $\emptyset$9.raffia there 9.EMPH be-R 9.DEM.PROX  \\
    \trans `The raffia there, it is that.'
\end{exe}

Also, special construction types can trigger the use of {\itshape bɛ̀} as copula in the \textsc{present}. For instance, the copula {\itshape bɛ̀} can occur as second constituent in a coordination of verbs, as in (\ref{becoord}). In order to keep the verbal structure of the first constituent, and share the first constituent's subject {\itshape yí} `it', the copula of the second constituent is verbal as well.

\begin{exe} 
\ex\label{becoord}
  \glll  bon pílì yí báàlá nà {\bfseries bɛ̀} ndɛ̀náà ndɛ̀náà ndáà ná\\
        bon pílì yi-H báàla-H nà bɛ̀ ndɛ̀náà ndɛ̀náà ndáà ná \\
          good[French] when 7-PRES repeat-R COM be like.that like.that also still  \\
    \trans `So, when it continues and is still like this and like that.'
\end{exe}

Another special construction type in the \textsc{present} where a verbal copula is chosen over the non-verbal copulas involves sentential modifiers, as illustrated in (\ref{koobe}). Certain sentential modifiers such as {\itshape kɔ́ɔ̀} `still' require an infinitival construction, as further discussed in \sectref{sec:SentMod}.

\begin{exe} 
\ex\label{koobe}
  \glll nà bí bɛ́sɛ̀ kɔ́ɔ̀ kùrã̂ {\bfseries bɛ̀} dé tù \\
       nà bí b-ɛ́sɛ̀ kɔ́ɔ̀ kùrã̂ bɛ̀ dé tù \\
        COM 1\textsc{pl}.EMPH 2-all still $\emptyset$7.electricity  be LOC inside \\
    \trans `with all of us just electricity be inside.'
\end{exe}

Besides these special cases in the \textsc{present}, the verbal copula {\itshape bɛ̀} is used in other tense-mood categories. This is shown for the \textsc{recent past} in (\ref{bepst1a}) and (\ref{bepst1b}). (\ref{bepst1a}) represents a nominal predicate, while (\ref{bepst1b}) gives an example where the predicate is an interrogative pronoun.


\begin{exe} 
\ex\label{bepst1a}
  \glll   yɔ́ɔ̀ ngã̀ nû à {\bfseries bɛ́} ngã̀  \\
          yɔ́ɔ̀ ngã̀ nû a bɛ̀-H ngã̂ \\
         so $\emptyset$1.healer 1.DEM.PROX 1.PST1 be-R $\emptyset$1.healer   \\
    \trans `So, this healer was a healer.'
\end{exe}


\begin{exe} 
\ex\label{bepst1b}
  \glll     mà {\bfseries bɛ́} vɛ́ \\ 
            ma bɛ̀-H vɛ́ \\
              6.PST1 be-R where \\
    \trans `Where were they [= the houses]?'
\end{exe}


\noindent Similarly, {\itshape bɛ̀} can be used in the \textsc{remote past}, as shown in (\ref{bepst2}).

\begin{exe} 
\ex\label{bepst2}
  \glll  yɔ́ɔ̀ Nzàmbí nɔ́gá núù {\bfseries bɛ́} Nzàmbí wà gyí\\
        yɔ́ɔ̀ Nzàmbí nɔ́-gá núù bɛ̀-H Nzàmbí wà gyí \\
          so $\emptyset$1.PN 1-other 1.PST2 be-R  $\emptyset$1.PN 1:ATT what   \\
    \trans `So this other Nzambi was which Nzambi?'
\end{exe}

Finally, the verbal copula {\itshape bɛ̀} can even take the \textsc{absolute completive} aspect marker {\itshape mà}, as shown in (\ref{bema}). This, however, seems to be the only possible combination of verbal copula and aspect. Also, it is noteworthy that this construction has been observed several times with the Mabi version of the completive aspect marker {\itshape mà} as an instance of code-switching, but has never been noticed with the Gyeli form of the aspect marker {\itshape mɔ̀}.


\begin{exe} 
\ex\label{bema}
  \glll  wú {\bfseries bɛ́} {\bfseries mà} bî ndáwɔ̀ dé tù \\
         wú bɛ̀-H mà bî ndáwɔ̀ dé tù \\
         3 be-R COMPL[Kwasio] 1\textsc{pl}.NSBJ $\emptyset$9.house LOC inside   \\
    \trans `That it was already in our houses!'
\end{exe}

\paragraph{Negation with {\itshape bɛ̀}}
{\itshape bɛ̀} is the only copula type that can be used in negated copula constructions. This holds for all predication types as well as for all tenseood categories, including the \textsc{present}. Thus, the negated form {\itshape bɛ́lɛ́} is used in the \textsc{present}, for instance with a nominal predicate, as in (\ref{bele1}).

\begin{exe} 
\ex\label{bele1}
  \glll  mɛ̀ɛ́ {\bfseries bɛ́lɛ́} mùdì wà lèkɛ́lɛ̀ \\
         mɛ̀ɛ́ bɛ́-lɛ́ m-ùdì wà le-kɛ́lɛ̀ \\
          1\textsc{sg}.PRES.NEG be-NEG N1-person 1:ATT le5-word  \\
    \trans `I'm not a person of many words.'
\end{exe}

\noindent The same construction is used with adjectival predicates, as in (\ref{bele2}).

\begin{exe} 
\ex\label{bele2}
  \glll nkwànò wú {\bfseries bɛ́lɛ́} mpà \\
       nkwànò wu-H bɛ̀-lɛ mpà \\
       $\emptyset$3.honey 3-PRES be-NEG good  \\
    \trans `The honey is not good.'
\end{exe}

\noindent Also deictic predicates have been found with a negated copula {\itshape bɛ́lɛ́}, as in (\ref{bele3}).


\begin{exe} 
\ex\label{bele3}
  \glll nyɛ̀ nâ mɛ̀ɛ́ {\bfseries bɛ́lɛ́} wû \\
       nyɛ nâ mɛ̀ɛ́ bɛ̀-lɛ wû \\
        1 COMP 1\textsc{sg}.PRES.NEG be-NEG there \\
    \trans `He [says]: `I'm not there.''
\end{exe}

\noindent Finally, there are a few constructions which lack a predicate, parallel to what has been described for the STAMP copula in \sectref{sec:COP}. In (\ref{bele4}), the negated copula expresses a negative existential clause: the person is not there. While in English, the use of `there' is obligatory in these constructions, in Gyeli, the occurrence of the deictic as in (\ref{bele3}) is optional. In (\ref{bele4}), the deictic does not appear so that the negated form of `be' serves as predicate in this case.

\begin{exe} 
\ex\label{bele4}
  \glll  mùdì nú {\bfseries bɛ́lɛ́}   \\
        m-ùdì nú bɛ́-lɛ́ \\
         N1-person 1.DEM.DIST be-NEG        \\
    \trans `This person is not there.'
\end{exe}

\paragraph{Predicate possession with {\itshape bɛ̀ nà}}
The verbal copula {\itshape bɛ̀} `be' in conjunction with the comitative marker {\itshape nà} express predicate possession. Typically, the predicate is nominal in these cases. Predicate possession with {\itshape bɛ̀ nà} can be used in all tense-mood categories. I provide examples for some of them in (\ref{bena}), namely for the \textsc{present}, the \textsc{recent past}, and the \textsc{future}.

\begin{exe} 
\ex\label{bena}
\begin{xlist}
\ex\label{bena1}
  \glll  mɛ́ {\bfseries bɛ́} {\bfseries nà} nkwànò  \\
        mɛ-H bɛ̀-H nà nkwànò \\
           1-PRES be-R COM $\emptyset$3.honey      \\
    \trans `I have honey.'
\ex\label{bena2}
  \glll  mɛ̀ {\bfseries bɛ́} {\bfseries nà} nkwànò  \\
        mɛ bɛ̀-H nà nkwànò \\
           1.PST1 be-R COM $\emptyset$3.honey      \\
    \trans `I had honey.'
\ex\label{bena3}
  \glll  mɛ̀ɛ̀ {\bfseries bɛ̀} {\bfseries nà} nkwànò  \\
        mɛ̀ɛ̀ bɛ̀ nà nkwànò \\
           1.FUT be COM $\emptyset$3.honey      \\
    \trans `I will have honey.'
\end{xlist}
\end{exe}

Encoding of predicate possession in the \textsc{present} is special in that it can also take other forms to express the meaning of `have'. While the verbal copula plus comitative marker as in (\ref{bena1}) is one option, the copula can also be omitted in the \textsc{present} so that only the comitative marker surfaces, as in (\ref{bena1a}).

\begin{exe}
\ex\label{bena1a}
  \glll  mɛ́ {\bfseries nà} nkwànò  \\
        mɛ-H nà nkwànò \\
           1-PRES COM $\emptyset$3.honey      \\
    \trans `I have honey.'
\end{exe}

\noindent Further, another verbal copula, {\itshape bùdɛ́}, can be used, as discussed in \sectref{sec:COPbude}.

{\itshape bɛ̀ nà} can be used for affirmative clauses, but also in negation, thus expressing negative possession. Negation of {\itshape bɛ̀ nà} constructions is achieved by regular negation patterns for the different tense-mood categories. In the \textsc{present}, two construction types are possible. One involves the negation suffix -{\itshape lɛ}, as in (\ref{beNEG1}).


\begin{exe} 
\ex\label{beNEG1}
  \glll mɛ̀ɛ́ {\bfseries bɛ́lɛ́} {\bfseries nà} nkwànò \\
       mɛ̀ɛ́ bɛ̀-lɛ nà nkwànò \\
       1\textsc{sg}.PRES.NEG be-NEG COM $\emptyset$3.honey  \\
    \trans `I don't have any honey.'
\end{exe}

\noindent The second possible negation construction involves the negation particle {\itshape tí}, or, as in (\ref{beNEG2}), the Mabi form {\itshape kí} which is often used in code-switching.

\begin{exe} 
\ex\label{beNEG2}
  \glll mɛ̀ {\bfseries kí} {\bfseries bɛ̀} {\bfseries nà} tsídí \\
       mɛ kí bɛ̀ nà tsídí \\
       1\textsc{sg} NEG[Kwasio] be COM $\emptyset$1.meat  \\
    \trans `I don't have any meat.'
\end{exe}

Also for \textsc{past} negation, both negation words, {\itshape sàlɛ́} and {\itshape pálɛ́} can be used, as (\ref{besale}) and (\ref{bepale}) show. The negation words precede {\itshape bɛ̀ nà} as they would with any other verb.

\begin{exe} 
\ex\label{besale}
  \glll ɛ́kɛ̀! Nzàmbí wà nú áà {\bfseries sàlɛ́} {\bfseries bɛ̀} {\bfseries nà} bã̂ líná-á pámò \\
      ɛ́kɛ̀! Nzàmbí wà nú áà sàlɛ́ bɛ̀ nà bã̂ líná a-H pámo \\
        EXCL $\emptyset$1.PN 1:ATT 1.DEM.DIST 1.PST2 NEG.PST be COM $\emptyset$7.word when 1-PRES arrive  \\
    \trans `Oh! That Nzambi had no words as soon as he arrives.'
\end{exe}


\begin{exe} 
\ex\label{bepale}
  \glll  yà {\bfseries pálɛ́} {\bfseries bɛ̀} {\bfseries nà} bùdã̂ \\
      ya pálɛ́ bɛ̀ nà b-ùdã̂ \\
        1\textsc{pl} NEG.PST be COM ba2-woman  \\
    \trans `We did not have any women.'
\end{exe}

\noindent Accordingly, negation of predicate possession in the \textsc{future} is achieved with the \textsc{future} negation word {\itshape kálɛ̀}, as shown in (\ref{bekale}).


\begin{exe} 
\ex\label{bekale}
  \glll  mɛ̀ɛ̀ {\bfseries kálɛ̀} ná {\bfseries bɛ̀} {\bfseries nà} jí ɛ́ vâ \\
        mɛ̀ɛ̀ kálɛ̀ ná bɛ̀ nà jí ɛ́ vâ \\
           1\textsc{sg}.FUT NEG.FUT anymore be COM $\emptyset$7.place LOC here  \\
    \trans `I won't have a place here anymore.'
\end{exe}









\subsection{Verbal copula {\itshape múà} `be almost'}
\label{sec:COPmua}

The verbal copula {\itshape múà} seems to be a special variety for expressing copular clauses in the \textsc{recent past}. As such, its use is very limited as well as its occurrence in the corpus. While the general verbal copula {\itshape bɛ̀} constitutes 24.1\% of all copula occurrences in the corpus, {\itshape múà} only constitutes 5.4\%.  Also, the use of {\itshape múà} as a copula seems to depend on speaker preference. Only one of the speakers chose {\itshape múà} over {\itshape bɛ̀} while other speakers only used {\itshape múà} as \textsc{prospective} marker (see \sectref{sec:PROSP}). Therefore, in all copular clauses with {\itshape múà}, {\itshape múà} could be replaced by the more general verbal copula {\itshape bɛ̀}. Examples from the corpus with {\itshape múà} as copula are given in (\ref{bemua1}) and (\ref{bemua2}).


\begin{exe} 
\ex\label{bemua1}
  \glll  à {\bfseries múà} mɛ́dɛ́ nyá mùdì \\
          a múà mɛ́dɛ́ nyá m-ùdì   \\
         1 be self real N1-person    \\
    \trans `He was about to be a real (old) man.'
\end{exe}


\begin{exe} 
\ex\label{bemua2}
  \glll mɛ̀ {\bfseries múà} pɔ́nɛ́ wá yìmbá ntɛ́ wû \\
          mɛ múà pɔ́nɛ́ wá yìmbá ntɛ́ wû       \\
         1\textsc{sg} be $\emptyset$7.truth 3:ATT $\emptyset$7.age $\emptyset$3.size there \\
    \trans `I was really about the age of this size there [makes a gesture with hand showing his height].'
\end{exe}

\noindent {\itshape múà} as a copular verb is, however, more restricted than {\itshape bɛ̀} in that is can only occur in the \textsc{recent past}. Also, negation is not possible with {\itshape múà}.


\paragraph{Predicate possession with {\itshape múà nà}}
The expression of predicate possession is also possible with {\itshape múà} in conjunction with the comitative marker {\itshape nà}. Again, this is restricted to the \textsc{recent past}, as (\ref{muana}) shows.


\begin{exe} 
\ex\label{muana} 
  \glll  gbĩ́ gbĩ̀ gbĩ́ gbĩ̀ gbĩ́   à {\bfseries múà} {\bfseries nà} bábɛ̀ tí wúmbɛ̀ wɛ̀ \\
            gbĩ́-gbĩ̀-gbĩ́-gbĩ̀-gbĩ́  a múà nà bábɛ̀ tí wúmbɛ wɛ̀   \\
         IDEO:roaming 1 PROSP COM $\emptyset$7.illness NEG want-R die \\
    \trans `[imitation of the disease roaming in his body] He was about to be sick, without wanting to die.'
\end{exe}

\noindent {\itshape múà nà} cannot be directly negated, but requires the \textsc{past} negation words {\itshape sàlɛ́} or {\itshape pálɛ́} as in (\ref{besale}) and (\ref{bepale}).









\subsection{Verbal copula {\itshape bùdɛ́} `have'}
\label{sec:COPbude}

The verbal copula {\itshape bùdɛ́} `have' only expresses predicate possession. It is interchangeable with {\itshape bɛ̀} plus comitative marker {\itshape nà}, as (\ref{bude}) shows. 

\begin{exe} 
\ex\label{bude}
\begin{xlist} 
\ex\label{budea}
  \glll  bá {\bfseries bɛ́} {\bfseries nà} bvúbvù \\
        ba-H bɛ̀-H nà bvúbvù \\
         2-PRES be-R COM lots   \\
    \trans `They have lots.'
\ex\label{budeb}
  \glll  bá {\bfseries bùdɛ́} bvúbvù \\
        ba-H bùdɛ́ bvúbvù \\
         2-PRES have.R lots   \\
    \trans `They have lots.'
\end{xlist}
\end{exe}

{\itshape bùdɛ́} occurs 17 times in the corpus which equals 15.2\% of all copula occurrences. Out of 27 instances of {\itshape bɛ̀} as a copula, 10 occur with the comitative marker {\itshape nà}. Thus, {\itshape bɛ̀ nà} constructions only constitute 11.2\% of the copula constructions and are thus less frequent than predicate possession constructions with {\itshape bùdɛ́}. Given the relatively few instances in the corpus of both constructions, it is not yet possible to determine distributional and/or semantic differences. Speakers generally state that both constructions mean the same and both can be used interchangeably.

{\itshape bùdɛ́} differs from other verbs including the copula {\itshape bɛ̀} in its tonal behavior on the STAMP marker. Comparable to, for instance, the \textsc{future} tense-mood category, the first and second person singular and the STAMP marker of class 1 have a different tonal pattern, namely a L tone, than the STAMP markers of the other agreement classes which have a H tone, as in (\ref{budeb}). As to the tonal shape of the verb {\itshape bùdɛ́}, it always ends in a H tone which suggests that it belongs to the realis mood, as discussed in \sectref{sec:SynH}. Since {\itshape bùdɛ́} never occurs phrase-finally, however, it is not possible to prove that its final TBU is underlyingly L. I therefore gloss the realis H tone as being inherent to the verb.

The predicates in constructions with {\itshape bùdɛ́} are all nominal or extended nominal noun phrases, as examples (\ref{bude1}) through (\ref{bude3}) show. In (\ref{bude1}), the predicate is a noun plus a numeral.

\begin{exe} 
\ex\label{bude1} 
  \glll  mɛ̀ {\bfseries bùdɛ́} bwánɔ̀ bábáà \\
        mɛ̀ bùdɛ́ b-wánɔ̀ bá-báà \\
         1\textsc{sg}.SBJ have.R ba2-child 2-two   \\
    \trans `I have two children.'
\end{exe}

\noindent In (\ref{bude2}), the predicate is nominal as well, followed by a comitative construction which literally translates as `the Bulu has anger with me.'

\begin{exe} 
\ex\label{bude2} 
  \glll pílì wɛ́ kɛ́ nâ wɛ́ kɛ́ tɔ́kɛ̀ mwánɔ̀ sáyà bvúlɛ̀ à {\bfseries bùdɛ́} lébvúú nà mɛ̂ \\
        pílì wɛ-H kɛ̀-H nâ wɛ-H kɛ̀-H tɔ́kɛ m-wánɔ̀ sáyà bvúlɛ̀ a bùdɛ́ H-le-bvúú nà mɛ̂ \\
           when 2\textsc{sg}-PRES go-R COMP 2\textsc{sg}-PRES go-R collect N1-child $\emptyset$7.thing ba2.Bulu 1 have.R OBJ.LINK-le5-anger COM 1\textsc{sg}.NSBJ \\
    \trans `When you go to go gather a small thing, the Bulu is angry with me.'
\end{exe}

\noindent {\itshape bùdɛ́} can also occur in relative clauses, as (\ref{bude3}) shows. Here, the relative clause modifies the object noun phrase {\itshape mwánɔ̀ wɔ́ɔ̀}. The demonstrative following {\itshape bùdɛ́} is coreferential with this object noun phrase.

\begin{exe} 
\ex\label{bude3}
  \glll  vɛ̂ mɛ̂ sâ mwánɔ̀ wɔ́ɔ̀ [wà wɛ̀ {\bfseries bùdɛ́} nû]\textsubscript{REL} \\
         vɛ̂ mɛ̂ sâ m-wánɔ̀ w-ɔ́ɔ̀ wà wɛ bùdɛ́ nû \\
          give.IMP 1\textsc{sg}.NSBJ only N1-child 1-POSS.2\textsc{sg} 1:ATT 2\textsc{sg} have.R 1:DEM.PROX \\
    \trans `Give me only your child that you have here.'
\end{exe}


The distribution of {\itshape bùdɛ́} seems to be restricted to the \textsc{present} tense-mood category. Given the special tonal pattern of the STAMP marker which differs from the general \textsc{present} tonal pattern, tense-mood category affiliation cannot be determined by the default tonal shape. Speakers consistently translate clauses with {\itshape bùdɛ́} with the \textsc{present}. The same is true for the special construction involving the Kwasio loan form of the \textsc{absolute completive} marker {\itshape mà}. As discussed in \sectref{sec:COMPL}, the Gyeli completive marker {\itshape mɔ̀/-Ṽ} is restricted to the \textsc{recent past}. In (\ref{budema}), however, it occurs with {\itshape bùdɛ́} and speakers translate the sentence in the \textsc{present} into French as {\itshape Il a déjà une femme.} 


\begin{exe} 
\ex\label{budema}  
  \glll à {\bfseries bùdɛ́} {\bfseries mà} mùdã̂ \\
        a bùdɛ́ mà m-ùdã̂ \\
          1 have.R COMPL[Kwasio] N1-woman  \\
    \trans `He already has a wife.'
\end{exe}

Two explanations are possible. One could propose that {\itshape bùdɛ́} does not belong to the \textsc{present} tense-mood category and constitutes a general exception. As such, it can combine with the \textsc{absolute completive} marker {\itshape mà}. Semantically, it encodes a present perfect reading, comparable to English {\itshape have got} constructions. Alternately, one could propose that {\itshape bùdɛ́} belongs to the \textsc{present} tense-mood category, despite the special tonal pattern of the STAMP marker. The co-occurrence with {\itshape mà}, which is only expected to occur in the \textsc{recent past}, can be explained by the potential grammaticalization of {\itshape mà} into an adverb. It is noteworthy that {\itshape bùdɛ́} only co-occurs with the Kwasio loan form of {\itshape mà}, but never with its own \textsc{absolute completive} marker {\itshape mɔ̀/-Ṽ}. At the same time, speakers consistently translate {\itshape mà} as {\itshape déjà} `already'. It is thus possible that {\itshape mà} functions as an adverb rather than an aspect marker which would explain why {\itshape mà} is not restricted to the \textsc{recent past}.


Finally, {\itshape bùdɛ́} is also used in the quotative index of reported speech (see \sectref{sec:RD} for more information), as shown in (\ref{budeQI1}) and (\ref{budeQI2}). Generally, there seems to be a tendency that {\itshape bùdɛ́} as verb in a quotative index indicates some kind of wish or order, as both examples illustrate.

\begin{exe} 
\ex\label{budeQI1} 
  \glll  mais mɛ̀ {\bfseries bùdɛ́} nâ ɛ́ pɛ̀ ɛ́ wû bèyá lwɔ̃́ kwádɔ́ yã̂ ɛ́ wû \\
       mais mɛ bùdɛ́ nâ ɛ́ pɛ̀ ɛ́ wû bèya-H lwɔ̃̂-H kwádɔ́ y-ã̂ ɛ́ wû \\
         but[French] 1\textsc{sg} have.R COMP LOC over.there LOC there 2\textsc{pl}[Kwasio]-PRES build-R $\emptyset$7.village 7-POSS.1\textsc{sg} LOC there\\
    \trans `But I say that over there, there you (pl) build my village over there.'
\end{exe}


\begin{exe} 
\ex\label{budeQI2} 
  \glll mɛ̀ {\bfseries bùdɛ́} nâ á lwɔ́ngɔ́ mɛ̂ màndáwɔ̀ \\
        mɛ bùdɛ́ nâ a-H lwɔ́ngɔ-H mɛ̂ ma-ndáwɔ̀ \\
       1\textsc{sg} have.R COMP 1-PRES build[Kwasio]-R 1\textsc{sg}.NSBJ ma6-house   \\
    \trans `I say that she [Nadine] builds me houses,'
\end{exe}

\noindent Having outlined constructions with non-verbal predicates, I now turn to constructions with verbal predicates as well as a general discussion of grammatical relations in Gyeli.











\section{Verbal clauses and grammatical relations}
\label{sec:verbalC}

In this section, I first discuss the different grammatical relations found in Gyeli before describing basic clause types. I also address sentential modifiers.




\subsection[Grammatical relations]{Grammatical relations: definitions and diagnostics}
\label{sec:GR}
\sectionmark{Grammatical relations}

In this section, I describe the grammatical relations in Gyeli. In doing so, I follow \citet{dryer97} who argues against grammatical relations, such as {\itshape subject} and {\itshape object}, as cross-linguistic notions, but emphasizes that grammatical relations are fundamentally language-specific.
I therefore use a range of language specific formal criteria in order to determine the grammatical relations in Gyeli. These include word order, agreement, and suprasegmental noun phrase marking. Based on these criteria, I distinguish subjects, objects, and obliques in Gyeli, which I will discuss in turn.





\subsubsection{Subjects}
\label{sec:SBJ}

Subjects in Gyeli are formally characterized by their preverbal position in basic word order, as shown in (\ref{SBJ1}) and (\ref{SBJ2}), and by agreement of the STAMP marker, a portmanteau morpheme encoding subject agreement and other clause information such as tense-mood and negation (\sectref{sec:SCOP}). Also, pronouns can serve as a subject diagnostic since subject pronouns differ in their shape from non-subject pronouns.

\begin{exe} 
\ex\label{SBJ1}
  \glll yɔ́ɔ̀ [mùdã̂]\textsubscript{SBJ} á kɛ̀ \hfill [intransitive] \\
        yɔ́ɔ̀ m-ùdã̂ a-H kɛ̀ \\
      so N1-woman 1-PRES go \\
    \trans `So the woman goes.'
\end{exe}

\begin{exe} 
\ex\label{SBJ2}
  \glll  [Nzàmbí]\textsubscript{SBJ} à bwã̀ã́ mwánɔ̀ \hfill [transitive] \\
          Nzàmbí a bwã̀ã-H m-wánɔ̀ \\
             $\emptyset$1.PN 1.PST1 give.birth-R N1-child \\
    \trans `Nzambi has given birth to a child.'
\end{exe}

\noindent As visible in these two examples, the subject has the same characteristics for intransitive and transitive verbs, both in terms of word order and agreement behavior. 

The STAMP marker, {\itshape á} in (\ref{SBJ1}) and {\itshape à} in (\ref{SBJ2}), is a free grammatical morpheme rather than a prefix since it can optionally be omitted in certain contexts (\sectref{sec:SCOP}).
Still, the STAMP marker is a valid diagnostic for subjecthood since it can always be added to a nominal subject. The STAMP marker as subject agreement marker suffices as subject expression in cases where the subject noun phrase is zero expressed (before the verbal predicate in square brackets), as in (\ref{SBJ1a}) and (\ref{SBJ2a}) for intrasitive and transitive verbs, respectively. 

\begin{exe} 
\ex\label{SBJ1a}
  \glll yɔ́ɔ̀ [á kɛ̀] \hfill [intransitive] \\
        yɔ́ɔ̀ a-H kɛ̀ \\
      so 1-PRES go \\
    \trans `So she goes.'
\end{exe}

\begin{exe} 
\ex\label{SBJ2a}
  \glll  [à bwã̀ã́] mwánɔ̀ \hfill [transitive] \\
           a bwã̀ã-H m-wánɔ̀ \\
             1.PST1 give.birth-R N1-child \\
    \trans `He has given birth to a child.'
\end{exe}



Another diagnostic is the form of subject pronouns which differs from non-subject pronouns (\sectref{sec:SBJPRO} and \ref{sec:OBJPRO}), as illustrated in (\ref{SBJSCOP}) with the subject and non-subject pronouns for agreement class 6.


\begin{exe} 
\ex\label{SBJSCOP}
\begin{xlist}
\ex\label{SBJSCOP1}
  \glll  [{\bfseries má}]\textsubscript{SBJ} má kwé mímpìndí \hfill [subject] \\
         má  ma-H kwè-H H-mi-pìndí \\
         6.SBJ 6-PRES fall-R OBJ.LINK-mi4-non.ripe \\
    \trans `They [= the bread fruit] fall non ripe.'
\ex\label{SBJSCOP2}
  \glll  mɛ́ nyɛ́ [{\bfseries mɔ̂}]\textsubscript{OBJ} \hfill [object] \\
           mɛ-H nyɛ̂-H mɔ̂ \\
             1\textsc{sg}-PRES see-R 6.NSBJ \\
    \trans `I see them [= the bread fruit].'
\ex\label{SBJSCOP3}
  \glll  mɛ́ njí [nà {\bfseries mɔ̂}]\textsubscript{OBL} \hfill [oblique] \\
           mɛ-H njì-H nà mɔ̂ \\
             1\textsc{sg}-PRES come-R COM 6.NSBJ \\
    \trans `I bring them [= the bread fruit].'
\end{xlist}
\end{exe}

\noindent (\ref{SBJSCOP1}) shows the subject pronoun {\itshape má}  which precedes the STAMP marker. In (\ref{SBJSCOP2}), the agreement class 6 pronoun is in object position and takes the shape {\itshape mɔ̂}. This is the same form as the pronoun takes in obliques with the comitative marker {\itshape nà}, as in (\ref{SBJSCOP3}).









\subsubsection{Objects}
\label{sec:HLinker}

While subjects can uncontroversially be recognized as a grammatical relation, it is more challenging to distinguish objects from obliques. This seems to be particularly common in northwestern Bantu. For instance, \citet[287]{velde2008} only distinguishes subjects from non-subjects in Eton (A71) since ``there are no clear syntactic arguments to define grammatical relations other than subject.'' This correponds to \posscitet{schadeberg95} observation that 
\begin{quote}
``Bantu languages recognize a type of syntactic relationship which is wider than our traditional category of object, including some but not all of our category of adjunct.'' (p. 179) \end{quote}
 In Gyeli, however, there are means to distinguish objects from obliques, even though they differ from the typical diagnostics used in Bantu languages.


Some of the typical object diagnostics for Bantu languages such as object prefixes on the verb or passivization, as suggested by \citet{schadeberg95}, do not work in Gyeli.  In Gyeli, objects are generally not cross-referenced on verbs.  (\ref{obj}) shows that the verb does not take any object marking prefixes, no matter whether the object is expressed by a lexical noun phrase, as in (\ref{obj1}), or a pronoun, as in (\ref{obj2}).

\begin{exe} 
\ex\label{obj}
\begin{xlist}
\ex\label{obj1}
  \glll    mɛ́ bìyɔ́ {\bfseries Màmbì} \hfill S V O\textsubscript{N} \\
		mɛ-H bìyɔ-H Màmbì   \\
             1\textsc{sg}-PRES beat-R $\emptyset$1.PN    \\
    \trans `I beat Mambi.'
\ex\label{obj2}
  \glll    mɛ́ bìyɔ́ {\bfseries nyɛ̂} \hfill S V O\textsubscript{PRO} \\
	mɛ-H bìyɔ-H nyɛ̂     \\
              1\textsc{sg}-PRES beat-R 1.NSBJ  \\
    \trans `I beat him.'
\end{xlist}
\end{exe}

\noindent In contrast to pre-verbal object prefixes, post-verbal object marking is more difficult to analyze. This is because, according to \citet[239]{marten2012}, post-verbal object markers \begin{quote}  ``may in fact be normal pronouns, or pronouns in some special position with respect to the verb, or clitic pronouns with special phonological or morphological characteristics.'' \end{quote} In Gyeli, I consider them ``normal'' pronouns. As such, they do not qualify as object diagnostics.

Another diagnostic that is often used in determining objects in Bantu is passivization. In Gyeli, passivization seems, however, to be a rare process that mostly shows up in elicitations, but not in natural speech. I therefore do not consider passivization as a good diagnostic for objecthood, even though simple constructions such as in (\ref{passdefa}) yield the expected results. As described in \sectref{sec:PASS}, the object of an active construction as in (\ref{passdef1a}) corresponds to the subject of a passive construction as in (\ref{passdef2a}), while the subject of an active construction can optionally be expressed as an oblique in the passive construction.

\begin{exe}
\ex\label{passdefa}
\begin{xlist}
\ex \label{passdef1a}
  \glll  [bùdì bá]\textsubscript{SBJ} tsìlɔ́ [békálàdɛ̀]\textsubscript{OBJ}\\
	b-ùdì ba-H tsìlɔ-H H-be-kálàdɛ̀ \\
             ba2-person 2-PRES write-R OBJ.LINK-be8-book  \\
    \trans `People write books.'
\ex\label{passdef2a}
 \glll  [bèkálàdɛ̀ bé]\textsubscript{SBJ} tsìl{\bfseries á} [(nà bùdì)]\textsubscript{OBL} \\
	be-kálàdɛ̀ be-H tsìl-a-H nà b-ùdì \\
         be8-book 8-PRES write-PASS-R COM ba2-person  \\
    \trans `Books are written (by people).'
\end{xlist}
\end{exe}

Passivization as object diagnostic in Gyeli is limited, however. First, passivization is a restricted morphological process, given that the possibility to form passives is lexically determined by the verb. Thus, many verbs that semantically would be expected to have a passive form, do not. Speakers generally prefer active constructions with unspecified agents expressed by the agreement class 2 STAMP marker {\itshape ba}. Second, while passivization might work as a diagnostic for single objects, it does not for double object constructions. The attempt to passivize both objects in a double object construction in elicitation proved to be an unnatural process and yielded dubious results.

Having ruled out some typical Bantu object diagnostics for Gyeli, I now turn to the two formal criteria that actually characterize objects in this language. These include suprasegmental marking of the object noun phrase, which I call an ``object linking H tone'', and word order. I will discuss both in turn.  



%Bresnan \& Moshi 1990: Object asymmetries
%asymmetrical object type:  only one of the postverbal NPs exhibits "primary object" syntactic properties of passivizability, object agreement, adjacency to the verb, and the like
%symmetrical object type: more than one NP can display "primary object" syntactic properties
%- ``even in symmetrical object languages some syntactic processes distinguish objects from indirect objects.'' (p. 157) (basic word order constraints, e.g. IO first, the OBJ in Chichewa)

%Riedel and Marten 2012: 278
%``typically, Bantu languages have a very small set of underived ditransitive verbs, there is no overt case marking, and objects are not overtly distinguished from adjuncts by any other kind of special morphology.''


\paragraph{Object linking H tone}
Objects in Gyeli are marked by a syntactic H tone that attaches to underlyingly toneless tone bearing units of the object noun, namely to CV- noun class prefixes.\footnote{There is one other toneless element that the syntactic object linking H tone can be realized on, namely the verbal plural particle {\itshape nga} (\sectref{sec:nga}), which seems to `steal' the object linking H tone from the object.} I gloss this object linking H tone as ``OBJ.LINK.''  Thus, in (\ref{LINK1}), the object receives a H tone, attaching to the noun class prefix which is underlyingly toneless.

\begin{exe} 
\ex\label{LINK1} 
  \glll   wɛ̀ nzíí bàlɛ̀ [{\bfseries bé}bã́ã̀]\textsubscript{OBJ} \\
      wɛ nzíí bàlɛ H-be-bã́ã̀ \\
            2\textsc{sg} PROG.PRES keep OBJ.LINK-be8-word \\
    \trans `You are keeping the words.'
\end{exe}


\noindent In contrast, in (\ref{notone}), the noun phrase following the verb is not marked with a H tone, indicating its status as an oblique.

\begin{exe}
\ex\label{notone}
  \glll mɛ̀ pàlɛ́ kɛ̀ dyɔ̂ [{\bfseries mà}fú málálɛ̀]\textsubscript{OBL} \\
       mɛ pàlɛ́ kɛ̀ dyɔ̂ ma-fú má-lálɛ̀  \\
         1\textsc{sg} NEG.PST go sleep ma6-day 6-three  \\
    \trans `I haven't slept in three days.'
\end{exe}

\noindent Since the appearance of the object linking H tone is restricted to toneless tone bearing units, namely CV- noun class prefixes, nominal objects that have no CV- prefix or pronominal objects are not marked for their object status suprasegmentally. Only a substitution test, substituting a tonally unmarked noun phrase with a noun that has a  CV- noun class prefix, ultimately determines whether the noun phrase is an object or an oblique. This, however, is subject to further restrictions. As we will see below, in double object constructions, only the object that is closest to the verb is tonally marked as an object.


%Before investigating double object constructions and multiple post-verbal noun phrases, I first turn to discussing the origin of the object linking H tone and its appearance in related languages of the area.
%A H tone on the object's noun class prefix also occurs in other languages of the area, for instance in Abo (A42). In Abo, however, the H tone on the lexical object noun phrase is phonologically conditioned as a result of HTS. As \citet[171]{hyman2011} show, the tone on the object prefix is tied to metatony. Thus, the object prefix following a verb which takes a grammatical H tone also surface H, as in (\ref{AB1}). In contrast, if the verb occurs without metatony, as in (\ref{AB2}), the object prefix surfaces L.\footnote{There are special tense-mood categories where the verb ends in a H tone, but HTS is blocked and the object surfaces with a L tone prefix. This is the case for imperatives, statives, and subjunctives.}

%\begin{exe}
%\ex\label{AB} Abo (A42)
%\begin{xlist}
%\ex \label{AB1}
%  \gll  ǎ pɔ̀ŋ{\bfseries ɔ́} {\bfseries bí}-támbé.    (with metatony) \\
%         1 make be8-shoe \\
 %   \trans `He is making shoes.'
%\ex\label{AB2}
 % \gll  à káà pɔ̀ŋ{\bfseries ɔ̀} {\bfseries bì}-támbé. (without metatony) \\
 %          1 FUT make be8-shoe \\
 %   \trans `He will make shoes.'
%\end{xlist}
%\end{exe}


In Gyeli, I argue for two distinct tones, a grammatical realis marking H tone on the verb (\sectref{sec:SynH}), and an object linking H tone on the CV- noun class prefix of an object. While it is possible that the object linking H tone has its origin in high tone spreading from the realis marking H tone on the verb, synchronically, these two tones are distinct, as (\ref{meta2}) shows. The object linking H tone shows up in conjunction with the realis marking H tone, as in (\ref{meta2a}), but also without metatony, as in (\ref{meta2b}). The latter case makes clear that HTS is not an explanation for the H tone on the object.

\begin{exe} 
\ex\label{meta2}
\begin{xlist}
\ex\label{meta2a} 
  \glll     mɛ́ gyám{\bfseries bɔ́} {\bfseries bé}lɔ̀lɔ̀ \hfill (with realis H tone)  \\
          mɛ-H gyámbɔ-H H-be-lɔ̀lɔ \\
              1\textsc{sg}-PRES cook-R OBJ.LINK-be8-duck   \\
    \trans `I cook ducks.'
\ex\label{meta2b}
  \glll     mɛ̀ɛ̀  gyám{\bfseries bɔ̀} {\bfseries bé}lɔ̀lɔ̀ \hfill (without realis H tone) \\
            mɛ̀ɛ̀  gyámbɔ H-be-lɔ̀lɔ \\
              1\textsc{sg}.FUT cook OBJ.LINK-be8-duck   \\
    \trans `I will cook ducks.'
\end{xlist}
\end{exe}

Other evidence that the H tone on the object prefix cannot stem from hight tone spreading comes from examples where multiple verbs occur between the realis marking H tone and the object H tone, as in (\ref{me1}).

\begin{exe} 
\ex\label{me1} 
\glll à nzí{\bfseries í} tálɛ̀ sɛ́lɔ̀ [{\bfseries bé}ntùgú]\textsubscript{OBJ} \\
     a nzíí tálɛ sɛ́lɔ H-be-ntùgú \\
    1 PROG.PRES.R begin peel OBJ.LINK-be8-potato \\
\trans `S/he is starting to peel potatoes.'
\end{exe}

\noindent The same is true when other parts-of-speech than verbs stand between the finite verb and the object, as for instance the adverb in (\ref{me2}).

\begin{exe}
\ex\label{me2}
\glll mɛ́ kwà{\bfseries lɛ́}  kɔ́ɔ̀ [{\bfseries bá}bwálɛ̀ bã́ã̀]\textsubscript{OBJ} \\
        mɛ-H kwàlɛ-H  kɔ́ɔ̀ H-ba-bwálɛ̀ b-ã́ã̀ \\
    1\textsc{sg}-PRES love-R always OBJ.LINK-ba2-parent 2-1\textsc{sg}.POSS\\
\trans `I always love my parents.'
\end{exe}

\paragraph{Double objects and the linking H tone}
The function of the linking H tone is to mark the object that is closest to the verb. This becomes apparent in constructions involving two objects. As (\ref{ncpre}) shows, a verb can be followed by two object noun phrases. \citet[279]{riedel2012} point out that indirect objects generally precede direct objects in Bantu languages. In Gyeli, however, there is no word order restriction as to which object is closer to the verb.  (\ref{ncpre2}) illustrates that also the direct object can precede the indirect object. Further, there are no formal criteria to distinguish what is generally called a direct object from an indirect object. Therefore, I will rather refer to multiple objects as the first object, i.e.\ the object closer to the verb, and the second object. The crucial point is that, in Gyeli, the object that is closer to the verb is marked by the linking H tone, but not the second object.  

\begin{exe}
\ex\label{ncpre}
\begin{xlist}
\ex\label{ncpre1}
  \glll  mɛ́ vɛ́ {\bfseries bá}bwálɛ̀ bèfùmbí \hfill S V O\textsubscript{1} O\textsubscript{2}  \\
        mɛ-H vɛ̂-H H-ba-bwálɛ̀ be-fùmbí \\
           1\textsc{sg}-PRES give-R OBJ.LINK-ba2-parent be8-orange  \\
    \trans `I give the parents oranges.'
\ex \label{ncpre2}
  \glll  mɛ́ vɛ́ {\bfseries bé}fùmbí bàbwálɛ̀ \hfill S V O\textsubscript{1} O\textsubscript{2}  \\
       mɛ-H vɛ̂-H H-be-fùmbí ba-bwálɛ̀ \\
         1\textsc{sg}-PRES give-R OBJ.LINK-be8-orange ba2-parent   \\
    \trans `I give oranges to the parents.'
\end{xlist}
\end{exe} 

\noindent Thus, tonally, the second object cannot be distinguished from an oblique noun phrase as in (\ref{notone}) where the noun class prefix also surfaces with a L tone. In order to distinguish objects from obliques, another diagnostic is needed, namely word order.


\paragraph{Word order}

\citet[279]{riedel2012} state that \begin{quote}
``The clearest way to distinguish adjuncts from objects in Bantu languages appears to be word order. Bantu languages generally have the word order S V O X or rather S V IO DO X, where locatives usually follow any objects, and high adjuncts, such as temporal modifiers, also follow the objects.'' \end{quote}

\noindent This generalization broadly applies to Gyeli as well, except that indirect and direct objects cannot be clearly distinguished, as noted above. Thus, it seems more accurate for Gyeli to suggest a general order of S V O\textsubscript{1} O\textsubscript{2} X\textsubscript{n}. The object slot can host any number of objects from none to two. Also the oblique position X can be filled by multiple adjuncts. Within the object slot, the order of the two objects is free. Similarly,  adjuncts are also  free in their relative order. Generally, however, objects are restricted to the object slot and obliques to the final X slot. This word order ultimately distinguishes objects from obliques and is illustrated in (\ref{objo}). 

\begin{exe}
\ex\label{objo}
\begin{xlist}
\ex\label{objo1}
  \glll  mɛ̀ vɛ́ [{\bfseries bá}bwálɛ̀]\textsubscript{OBJ1} [bèfùmbí]\textsubscript{OBJ2} [màfú málálɛ̀ dẽ̂]\textsubscript{X1} [ɛ́ tísònì]\textsubscript{X2}  \\
        mɛ vɛ̂-H H-ba-bwálɛ̀ be-fùmbí ma-fú má-lálɛ̀ dẽ̂ ɛ́ tísònì\\
           1\textsc{sg}.PST1 give-R OBJ.LINK-ba2-parent be8-orange ma6-day 6-three today LOC $\emptyset$7.town  \\
    \trans `I gave the parents oranges three days ago in town.'
\ex \label{objo2}
  \glll  mɛ́ vɛ́ [{\bfseries bé}fùmbí]\textsubscript{OBJ1} [bàbwálɛ̀]\textsubscript{OBJ2} [ɛ́ tísònì]\textsubscript{X1} [màfú málálɛ̀ dẽ̂]\textsubscript{X2}  \\
       mɛ-H vɛ̂-H H-be-fùmbí ba-bwálɛ̀ ɛ́ tísònì ma-fú má-lálɛ̀ dẽ̂ \\
         1\textsc{sg}-PRES give-R OBJ.LINK-be8-orange ba2-parent LOC $\emptyset$7.town ma6-day 6-three today \\
    \trans `I gave oranges to the parents in town three days ago.'
\ex\label{objo3}
  \glll  *mɛ̀ vɛ́ [{\bfseries bá}bwálɛ̀]\textsubscript{OBJ1}  [màfú málálɛ̀ dẽ̂]\textsubscript{X1} [bèfùmbí]\textsubscript{OBJ2} [ɛ́ tísònì]\textsubscript{X2} \\
        mɛ vɛ̂-H H-ba-bwálɛ̀ ma-fú má-lálɛ̀ dẽ̂ be-fùmbí ɛ́ tísònì \\
           1\textsc{sg}.PST1 give-R OBJ.LINK-ba2-parent ma6-day 6-three today be8-orange LOC $\emptyset$7.town \\
    \trans `I gave the parents three days ago oranges in town.'
\end{xlist}
\end{exe} 

\noindent In (\ref{objo1}) and (\ref{objo2}), the relative order of objects and obliques is reversed within the object and oblique slot, respectively. While this is permissible, moving an oblique into an object position or an object into the oblique slot, mixing objects and obliques, as in (\ref{objo3}), is prohibited. Thus, word order principles characterize a second object such as {\itshape bèfùmbí} `oranges' in (\ref{objo1}) as an object in comparison to the following oblique noun phrase {\itshape màfú málálɛ̀} `three days'. Both noun phrases carry a L tone on the noun class prefix since only the first object is marked by the object linking H tone. The second object, however, can be promoted to the first object position while the oblique noun phrase can only be reversed in order with another oblique.

%Transitivity

%intransitive 

%\begin{exe} 
%\ex\label{26} 
%  \glll  má dvúmɔ́lɛ́ mbvú mbì mbvû. \\
%        ma-H dvúmɔ́-lɛ́ mbvú mbì mbvû \\
%           6-PRES produce-NEG  $\emptyset$3.year like[Kwasio] $\emptyset$3.year \\
%    \trans `They don't produce [fruit] every year.'
%\end{exe}

%transitive 

%\begin{exe} 
%\ex\label{27} 
%  \glll  màlɛ́ndí máà vɛ̀ɛ̀  kwè mípìndí.  \\
 %        ma-lɛ́ndí máà vɛ̀ɛ̀  kwè H-mi-pìndí \\
%           ma6-palm.tree 6.DEM.PROX only fall OBJ.LINK-mi4-non.ripe \\
%    \trans `These palm trees only fall non-ripe [fruit].'
%\end{exe}







\subsubsection{Obliques}
\label{sec:OBL}

In the previous section, I explained the formal distinction between objects and obliques which is related to an object linking H tone and word order.
In this section, I present different types of obliques, following
 \posscitet{dryer2013}  definition of ``oblique'':
\begin{quote}
``An oblique phrase is a noun phrase or adpositional phrase (prepositional or postpositional) that functions as an adverbial modifier (or ``adjunct'') of the verb.'' \end{quote}

\noindent (\ref{OBL1}) provides an example with multiple obliques, all of which represent different types of oblique phrases. As described in the previous section already, the order of the oblique phrases can be freely varied, provided that the obliques remain within the oblique slot and do not move to the objects' position. 

\begin{exe}
\ex\label{OBL1}  S V O X1 X2 X3
  \glll  [bùdì bɔ́gà bá]\textsubscript{SBJ} gyámbó [bédéwɔ̀]\textsubscript{OBJ} [púù yá bwánɔ̀]\textsubscript{X1} [kìsínì dé tù]\textsubscript{X2} [nà màsɔ̀sí]\textsubscript{X3}\\
         b-ùdì bɔ́-gà ba-H gyámbɔ-H H-be-déwɔ̀ púù yá b-wánɔ̀ kìsínì dé tù  nà ma-sɔ̀sí\\
        ba2-person 2-other 2\textsc{pl}-PRES prepare-R OBJ.LINK-be8-food $\emptyset$7.reason 7:ATT ba2-child $\emptyset$7.kitchen LOC inside COM ma6-joy\\
    \trans `Other people prepare food for the children in the kitchen with joy.'
\end{exe}

\noindent X1 is an instance of a noun + noun construction expressing a benefactive oblique. X2 constitutes an adpositional noun phrase with the postposition {\itshape dé}, and X3 is a comitative phrase. I will describe different oblique phrase types in turn.

\paragraph{Bare noun phrases}

An oblique can have the structure of a bare noun phrase, i.e.\  a noun phrase without any adposition or other grammatical marker such as  the comitative. A similar example of a temporal oblique is given in (\ref{OBLbare}) (see also (\ref{notone})).

\begin{exe} 
\ex\label{OBLbare} 
  \glll mɛ̀gà mɛ́ɛ̀ dyúwɔ́ nzã́ã̀ [dúwɔ̀ lé tè]\textsubscript{X} \\
       mɛ-gà mɛ́ɛ̀ dyúwɔ-H nzã́ã̀ d-úwɔ̀ lé tè \\
         1\textsc{sg}-CONTR 1\textsc{sg}.PST2 feel-R $\emptyset$7.appetite le5-day 5:ATT there \\
    \trans `As for me, I felt appetite that day.'
\end{exe}

Bare noun phrases can also encode other types of obliques, as in (\ref{notone2}). Here, the first oblique, {\itshape bàgyɛ̃̂} `guest', serves as a secondary predication relating to the subject. The second oblique is introduced by the associative plural marker and discussed below.

\begin{exe} 
\ex\label{notone2}
  \glll mɛ́ lɔ́ njì [{\bfseries bàgyɛ̃̂}]\textsubscript{X1} [bà wɛ̂]\textsubscript{X2} \\
       mɛ-H lɔ́ njì ba-gyɛ̃̂ bà wɛ̂ \\
       1\textsc{sg}-PRES RETRO come ba2-guest AP 2\textsc{sg}  \\
    \trans `I just came as a guest to you.'
\end{exe}

\noindent The oblique nouns in both (\ref{notone}) and (\ref{notone2}) can clearly be identified as such since they surface with a L tone on their noun class prefix. If they were object arguments, they would surface with an object linking H tone.




\paragraph{Purpose/benefactive {\itshape púù yá} `reason of'}
Some nouns are consistently used in obliques. This is, for instance, the case with {\itshape púù} `reason' which is used in benefactive obliques, as shown in (\ref{BEN1}).

\begin{exe} 
\ex\label{BEN1}
  \glll á gyàgá mántúà [{\bfseries púù} yá bwánɔ̀]\textsubscript{X}\\
        a-H gyàga-H H-ma-ntúà púù yá b-wánɔ̀\\
        1-PRES buy-R OBJ.LINK-ma6-mango $\emptyset$7.reason 7:ATT ba2-child\\
    \trans `He buys mangoes for the children.'
\end{exe}

\noindent {\itshape púù yá} obliques also express purpose, as illustrated in (\ref{BEN2}).

\begin{exe} 
\ex\label{BEN2}
  \glll mɛ́ lɔ́ nɔ́ɔ̀ mwánɔ̀ [{\bfseries púù} yá mábɔ́'ɔ̀ mâ]\textsubscript{X} \\
        mɛ-H lɔ́ nɔ́ɔ̀ m-wánɔ̀ púù yá ma-bɔ́'ɔ̀ mâ \\
        1\textsc{sg}-PRES RETRO take N1-child $\emptyset$7.reason 7:ATT ma6-bread.fruit 6.DEM.PROX\\
    \trans `I have just taken the child for these bread fruit.'
\end{exe}

\paragraph{Manner/benefactive {\itshape mpá'à wá} `side of'}
While {\itshape púù} `reason' seems to be the default noun for benefactive obliques, also {\itshape mpá'à} `side' can be used for this function, as (\ref{BEN3}) shows.

\begin{exe} 
\ex\label{BEN3}
  \glll á gyàgá mántúà [{\bfseries mpá'à} wá bwánɔ̀]\textsubscript{X}\\
        a-H gyàga-H H-ma-ntúà mpá'à wá b-wánɔ̀\\
        1-PRES buy-R OBJ.LINK-ma6-mango $\emptyset$3.side 3:ATT 2ba-child\\ 
    \trans `He buys mangoes for the children.'
\end{exe}

While speakers state that both nouns can be used interchangeably for benefactive obliques, there seems to be a tendency that {\itshape mpá'à} `side' is used if the benefactor is expressed pronominally, as in (\ref{BEN4}), even though also pronominal benefactors are allowed with {\itshape púù} `reason'.

\begin{exe} 
\ex\label{BEN4}
  \glll á gyàgá mántúà [{\bfseries mpá'à} wã̂]\textsubscript{X} \\
        a-H gyàga-H H-ma-ntúà mpá'à w-ã̂ \\
        1-PRES buy-R OBJ.LINK-ma6-mango $\emptyset$3.side 3-POSS.1\textsc{sg} \\
    \trans `He buys mangoes for me.'
\end{exe}

\noindent Further, {\itshape mpá'à} `side' is used in manner obliques, as in (\ref{MAN1}).

\begin{exe} 
\ex\label{MAN1}
  \glll   bí bɔ́ɔ̀ yá bígɛ́ [{\bfseries mpá'à} wá vɛ́]\textsubscript{X} \\
           bí b-ɔ́ɔ̀ ya-H bígɛ-H mpá'à wá vɛ́ \\
           1\textsc{pl}.EMPH 2-other 1\textsc{pl}-PRES develop-R $\emptyset$3.side 3:ATT which  \\
    \trans `How will we others develop?'
\end{exe}



\paragraph{Obliques with the associative plural marker {\itshape bà}}
Another type of oblique phrase is introduced by the associative plural marker {\itshape bà} and its functional extensions (\sectref{sec:AP}) and expresses usually location, as in (\ref{chez1}) and (\ref{chez2}).\footnote{While associative plurals canonically co-occur with nouns whose referents are typically human, as stated by \citet{daniel2013}, the associative plural morpheme {\itshape bà} also extends to pronouns in Gyeli. Other than expressing association with the nominal referent, the associative plural can also express location at the referent's place which is systematically translated by the preposition {\itshape chez} `at somebody's place' into French.}

\begin{exe} 
\ex\label{chez1}
  \glll bèdéwò béndɛ̀ byɔ̀ mɛ́ lɔ́ njì lɛ́bɛ̀lɛ̀ bédéwò [{\bfseries bà} wɛ̂]\textsubscript{X} \\
        be-déwò bé-ndɛ̀ byɔ̀ mɛ-H lɔ́ njì lɛ́bɛlɛ H-be-déwò bà wɛ̂ \\
           be8-food 8-ANA 8.EMPH 1-PRES RETRO come  follow be8-food AP 2\textsc{sg}.NSBJ  \\
    \trans `It is that food that I have come to look for at your place.'
\end{exe}

\begin{exe} 
\ex\label{chez2}
  \glll mùdì á sɔ́mɔ́nɛ́ mùdã̂ [{\bfseries bà} kfúmá wà kwádɔ́]\textsubscript{X} \\
        m-ùdì a-H sɔ́mɔnɛ-H m-ùdã̂ [{\bfseries bà} kfúmá wà kwádɔ́ \\
           N1-person 1-PRES complain-R N1-woman AP $\emptyset$1.chief 1:ATT $\emptyset$7.village  \\
    \trans `The person complains about the woman at the chief of the village's place.'
\end{exe}

\noindent The associative plural corresponds to the French preposition {\itshape chez} `at' and is consistently translated as such.



\paragraph{Adpositional obliques}
Adpositional obliques express location. They come in two types, namely with i) the preposition {\itshape ɛ́} and ii) the postposition {\itshape dé}, as described in \sectref{sec:LOCe} and \ref{sec:LOCde}, respectively. The oblique including the preposition {\itshape ɛ́} in (\ref{LOBL1}) refers to some general location, corresponding to {\itshape at} in English.

\begin{exe} 
\ex\label{LOBL1}
  \glll nyàá sùbɔ̀ èsã̂s [{\bfseries ɛ́} dyúwɔ̀]\textsubscript{X} \\
        nyàá sùbɔ èsã̂s ɛ́ dyúwɔ̀ \\
       1.INCH pour $\emptyset$1.fuel LOC $\emptyset$7.top  \\
    \trans `He starts pouring fuel on top.'
\end{exe}

\noindent In contrast, the postpositional oblique in (\ref{LOBL2}) rather refers to containment, i.e.\ a location inside the locative noun.

\begin{exe} 
\ex\label{LOBL2} 
  \glll  bùdì bɛ́sɛ̀ bà nzíí kɛ̀ nà kɛ̀ dẽ́ [bèjìí {\bfseries dé} tù]\textsubscript{X} \\
       b-ùdì b-ɛ́sɛ̀ ba nzíí kɛ̀ nà kɛ̀ dẽ́ be-jìí dé tù \\
         ba2-person 2-all 2 PROG.PRES go COM go today be8-forest LOC inside   \\
    \trans `All the people are going into the forest today.'
\end{exe}


\paragraph{Locative obliques and the H tone}
Noun phrases that appear bare on the surface and expressing location and/or direction can also serve as obliques.%\footnote{Alternately, one could propose that the linking H tone generally picks out arguments, i.e.\ constituents that are required by the verb's valency, in contrast to adjuncts. Since arguments other than objects are rare in Gyeli, however, and those that allow a potential H tone to surface on a CV- prefix are even rarer, this is difficult to prove at the moment.} 
In (\ref{Lobj}), the verb {\itshape kɛ̀} `go', which is mostly intransitive, is followed by the location oblique {\itshape mánkɛ̃̂} `fields'. I propose that the H tone on {\itshape mánkɛ̃̂} `fields' stems from an assimilated locative preposition {\itshape ɛ́} (\sectref{sec:LOCe}), whose H tone survives on the noun class prefix.


\begin{exe} 
\ex\label{Lobj}
  \glll   wɛ̀ mɛ́dɛ́ pã́ lígɛ̀ yá nà nyɛ̀ yá kɛ́ [{\bfseries má}nkɛ̃̂]\textsubscript{OBL}  \\
         wɛ mɛ́dɛ́ pã̂-H lígɛ ya-H nà nyɛ ya-H kɛ̀-H ɛ́?-ma-nkɛ̃̂  \\
           2\textsc{sg}.EMPH self start-R stay 1\textsc{pl}-PRES COM 1  1\textsc{pl}-PRES go-R LOC?-6-field \\
    \trans `You [= his wife] stay first, we and her, we go to the fields.'
\end{exe}

One might assume that the H tone on {\itshape mánkɛ̂̃} `fields' could also be an object linking H tone, since, in Gyeli, the verb {\itshape kɛ̀} `go' might require a location argument. This possibility can, however, be excluded on the grounds that the location noun phrase clearly appears in an oblique position. In (\ref{Lobj2}), the location oblique {\itshape mánkɛ̃̂} `fields' follows another oblique noun phrase. Arguments, however, cannot appear after obliques.

\begin{exe} 
\ex\label{Lobj2}
  \glll mùdã̂ kɛ́ [nà nyɛ̀]\textsubscript{OBL} [mánkɛ̃̂]\textsubscript{OBL} \\
       m-ùdã̂ kɛ̀-H nà nyɛ̀ ɛ́-ma-nkɛ̃̂ \\
         N1-woman go-R COM 1 LOC?-ma6-field  \\
    \trans `The woman [his wife] shall go with him to the fields,'
\end{exe}




\paragraph{Comitative obliques}

A lot of oblique phrases contain the comitative marker {\itshape nà} `and/with'. The notion `comitative', as used in the Bantuist tradition, should however, not lead to any terminological confusion in assuming that it has only the use of accompaniment, for it shows a broad range of uses, as I will show in the following.

One salient function of comitative obliques is accompaniment, as shown in (\ref{naACC1}) and (\ref{naACC2}). In (\ref{naACC1}), the intransitive verb {\itshape njì} `come' is followed by the comitative phrase. This construction of `come with' is systematically used to express `bring' in English.

\begin{exe} 
\ex\label{naACC1} 
  \glll   ɛ́ pɛ̀ nâ á njíyɛ̀ mɛ̂ [{\bfseries nà} yɔ̂]\textsubscript{X} \\
         ɛ́ pɛ̀ nâ a-H njíyɛ mɛ̂ nà y-ɔ̂ \\
         LOC there COMP 1-PRES come.SBJV 1\textsc{sg}.NSBJ COM 7-NSBJ  \\
    \trans `So that she bring me that [food].'
\end{exe}

\noindent In (\ref{naACC2}), the comitative oblique {\itshape nà màbɔ́ɔ̀} `with bread fruit' is the accompaniment to the verb {\itshape dè} `eat'.

\begin{exe} 
\ex\label{naACC2}
  \glll nyɛ̀ nâ mɛ́ɛ̀ dé pɔ́nɛ́ [{\bfseries nà} màbɔ́'ɔ̀]\textsubscript{X} \\
        nyɛ nâ mɛ́ɛ̀ dè-H pɔ́nɛ́ nà ma-bɔ́'ɔ̀ \\
      1 COMP 1\textsc{sg}.PST2 eat-R $\emptyset$7.truth COM ma6-bread.fruit   \\
    \trans `He [says]: I really ate [it] with bread fruit.'
\end{exe}

\noindent The comitative oblique phrase can also have an instrumental function, as in (\ref{naINS1a}).

\begin{exe} 
\ex\label{naINS1a}
  \glll  á kɛ́ sɔ́lɛ̀gà ngùndyá [{\bfseries nà} nkwálá]\textsubscript{X}  \\
          a-H kɛ̀-H sɔ́lɛga ngùndyá nà nkwálá   \\
      1-PRES go-R chop $\emptyset$9.raffia COM $\emptyset$3.machete \\
    \trans `He goes to cut the raffia with the machete.'
\end{exe}

Instrumental meaning can extend to contexts which are expressed by locatives in English. In (\ref{naINS1}), the speaker chooses to employ a comitative oblique rather than a locative oblique with the preposition {\itshape ɛ́}. This gives more of an instrumental than locative reading.

\begin{exe} 
\ex\label{naINS1}
  \glll  á kɛ́ jìí dé tù [{\bfseries nà} ndzǐ gyâ]\textsubscript{X}  \\
          a-H kɛ̀-H jìí dé tù nà ndzǐ gyâ      \\
      1-PRES go-R $\emptyset$7.forest LOC inside COM $\emptyset$9.path $\emptyset$7.length \\
    \trans `He goes into the forest using the long path.'
\end{exe}

\noindent Another function of the oblique phrase is to express the agent role in a passive construction, as in (\ref{naPASS}).

\begin{exe} 
\ex\label{naPASS}
  \glll lé yí lɛ̀yá [{\bfseries nà} mpɛ̀wɔ́]\textsubscript{X}  \\
        lé yi-H lɛ̀ya-H nà mpɛ̀wɔ́ \\
        $\emptyset$7.tree 7-PRES uproot:PASS-R COM $\emptyset$3.wind   \\
    \trans `The tree is uprooted by the wind.'
\end{exe}

\noindent This structure is parallel to many verb constructions which synchronically cannot be transparently recognized as passive forms since they lack another underived form which does not end in -{\itshape a}.\footnote{See \sectref{sec:PASS} for more information on passive formation.}
In these instances, the oblique expresses some kind of source which is usually encoded by a prepositional phrase with {\itshape from} in English. In (\ref{naSOU1}), the source of the suffering is the raffia and bamboo.

\begin{exe} 
\ex\label{naSOU1}
  \glll     yá tfúgá [{\bfseries nà} ngùndyá mpángì]\textsubscript{X} \\
            ya-H tfúga-H nà ngùndyá mpángì \\
              1\textsc{pl}-PRES suffer-R COM $\emptyset$9.raffia $\emptyset$7.bamboo \\
    \trans `We suffer from the straw, the bamboo.'
\end{exe}

\noindent In (\ref{naSOU2}), the source of death is hunger.

\begin{exe} 
\ex\label{naSOU2}
  \glll  mɛ̀ múà wɛ̀ [{\bfseries nà} nzà]\textsubscript{X} \\
        mɛ múà wɛ̀ nà nzà \\
          1\textsc{sg} PROSP die COM $\emptyset$9.hunger  \\
    \trans `I'm about to die from hunger.'
\end{exe}

\noindent Another example where the comitative oblique expresses the source is given in (\ref{SOU3}).

\begin{exe} 
\ex\label{SOU3}
  \glll nyɛ̀gà váà nyɛ̀gá tsíyɛ́ sáà [{\bfseries nà} màlɛ́ndí]\textsubscript{X} màlɛ́ndí máà mɔ́gà \\
         nyɛ-gà váà nyɛ-gá tsíyɛ́ sáà nà ma-lɛ́ndí ma-lɛ́ndí máà mɔ́-gà \\
          1-CONTR here 1-CONTR live-R only COM 6-palm.tree 6-palm.tree 6:DEM 6-CONTR \\
    \trans `Him here, he lives only from palm trees, these palm trees.'
\end{exe}

Certain verbs such as {\itshape dílɛsɛ} `feed' in (\ref{SOU4}), also require a comitative oblique phrase rather than taking a noun phrase object. In such instances, one can think of the comitative's function either as manner or instrumental.

\begin{exe} 
\ex\label{SOU4}
  \glll Màmbì à nzí dílɛ̀sɛ̀ Àdà [{\bfseries nà} ntúà]\textsubscript{X}  \\
        Màmbì a nzí dílɛsɛ Àdà nà ntúà   \\
        $\emptyset$1.PN 1 PROG.PST feed $\emptyset$1.PN COM $\emptyset$7.mango   \\
    \trans `Mambi feeds Ada a mango.'
\end{exe}

\noindent Comitative obliques may encode a stimulus, as in (\ref{SOU5}) where the snake causes fear.

\begin{exe} 
\ex\label{SOU5}
  \glll Àdà á sàgá [{\bfseries nà} nyùà]\textsubscript{X}  \\
        Àdà a-H sàga nà nyùà  \\
        $\emptyset$1.PN 1-PRES be.scared-R COM $\emptyset$1.snake  \\
    \trans `Ada is scared of the snake.'
\end{exe}

\noindent These sentences provide a few examples of the functional range of comitative obliques. While they seem to cover the most frequent functions, they most likely do not constitute an exhaustive list.






\subsection{Basic word order}
\label{sec:BasicCT}

Based on the grammatical relations that I established for Gyeli in the previous section, I now discuss the  basic word order in this language.%\footnote{I will not discuss valency as such. It should suffice here to say that I do not believe in a simple valency classification that is inherent to a verb.} 
\ According to \citet[73-76]{dryer2007b}, basic word order can be identified through a number of criteria, such as:
\begin{enumerate}
\itshapeem Frequency
\itshapeem Pragmatic neutrality
\itshapeem Possible restrictions in distribution
\end{enumerate}
For Gyeli, I will mostly consider frequency as determining the basic word order. Pragmatic neutrality ties in with this factor since those constructions that are not pragmatically neutral, i.e.\ which take over some special topic or focus function, as discussed in \sectref{sec:IS}, are naturally less frequent. As to possible restrictions in distribution, we will see in \chapref{sec:CC} that Gyeli generally keeps the basic word order of simple, main clauses also in dependent clauses.


Table \ref{Tab:BCT} summarizes the frequency of each basic clause type relating to word order as found in the Gyeli corpus. `Basic clause type' includes all simple, non-dependent clauses with a verbal predicate. 
By definition, other clause types are excluded from this count, namely complex clauses, such as relative clauses and coordination,  and clauses with non-verbal predicates. I also do not consider unfinished sentences that obviously occur in natural speech. Repeated clauses are only listed once to not artificially enlarge the corpus with one construction type. Subjects and objects include both instances of lexical noun phrases and bare STAMP markers or pronominal objects.


\begin{table}[!h]
\centering
%\scalebox{0.9}{
\begin{tabular}{l|lll}
 \midrule
\multirow{3}{*}{Basic word order} & S V (X\textsubscript{n}) & 104 & (48.8\%) \\
& S V O (X\textsubscript{n}) & 74 & (34.7\%)\\
& S V O\textsubscript{1} O\textsubscript{2} (X\textsubscript{n}) & 3 & (1.4\%) \\
 \midrule 
\multirow{2}{*}{Imperatives} &  $\emptyset$ V (X\textsubscript{n}) & 5 & (2.3\%) \\
 & $\emptyset$ V O (X\textsubscript{n}) & 3 & (1.4\%)  \\ 
 \midrule
\multirow{3}{*}{Special object position}  & S V X LO & 1 & (0.5\%) \\
 & Object fronting & 17 & (8\%)  \\
 & Left dislocation & 6  & (2.8\%) \\
 \midrule
Total & &  213 \\
 \midrule
\end{tabular}
\caption{Word order in simple clauses}
\label{Tab:BCT}
\end{table}

As Table \ref{Tab:BCT} shows, the most frequent word order patterns in Gyeli are S V (48.8\%) and S V O (34.7\%). Intransitive constructions are more frequent than those containing an object, while double object constructions are rather rare in the corpus, representing only 1.4\% of the basic verbal clauses.\footnote{Note that `V' generally represents the predicate without specifying whether the predicate is simple or complex. Thus, `V' may be comprised of 1-3 verbs; complex predicates are discussed in \sectref{sec:CompPred}.} Every construction type can be followed by one or more oblique phrases.  As outlined in \sectref{sec:OBL}, obliques generally follow the object slot. This is also true for special word order patterns such as object fronting and left dislocation. The only exception concerns locative objects with the verb {\itshape kɛ̀} `go' where a comitative oblique can precede the object noun phrase expressing a goal or direction.

Imperatives and special object positions in Table \ref{Tab:BCT} list exceptional patterns. First, imperative forms, except for the first person plural, lack STAMP marking. Therefore, both intransitive and transitive imperative constructions do not contain a subject, while maintaining the general word order of verb before object.

Object positions can be exceptional in various ways. The first construction type of S V X LO is special in that the oblique precedes the object. This, as confirmed in elicitations and further discussed in \sectref{sec:HLinker}, only works with locative objects. Object fronting and left dislocation are pragmatically non-neutral constructions and relate to information structure. Both are discussed in more detail in \sectref{sec:IS}. Object fronting subsumes all instances where a pronominal object precedes the simple verb or part of a multi-verb construction. In addition to the basic word order criterion of being pragmatically neutral, object fronting is further restricted in its distribution since only pronominal objects can be fronted. As such, object fronting cannot be considered a basic word order type. The same is true for left dislocation where the lexical object noun phrase precedes the subject noun phrase (and is then pronominally taken up again in situ). These construction types are non-basic due to their low frequency.

%\begin{table}[!h]
%\centering
%\scalebox{0.9}{
%\begin{tabular}{l|lll}
% \midrule
%\multirow{5}{*}{Main patterns} & S V & 84 & (39.4\%) \\
% & S V X (X) & 20 & (9.4\%)\\
% & S V O & 58 & (27.2\%)\\
% & S V O X (X) & 16 & (7.5\%) \\
% & S V O\textsubscript{1} O\textsubscript{2} & 3 & (1.4\%) \\
%  \midrule
% \multirow{7}{*}{Special object position}  & S V X LO & 1 & (0.5\%) \\

% & S O V & 1 & (0.5\%)  \\
% & S V\textsubscript{1} O  V\textsubscript{2} (X) & 13 & (6.1\%) \\
% & S V\textsubscript{1} O\textsubscript{1} V\textsubscript{2} O\textsubscript{2} & 3  & (1.4\%) \\
% & O\textsubscript{1}, S V O\textsubscript{1} (X) & 2  & (0.9\%) \\
% & O\textsubscript{1}, S V\textsubscript{1} O\textsubscript{1} V\textsubscript{2} & 2 & (0.9\%) \\
% & O\textsubscript{1}, S O\textsubscript{1} V (X) & 2 & (0.9\%) \\
%  \midrule
% \end{tabular}
% \caption{Frequency of obliques in simple clauses}
% \label{Tab:OBLF}
% \end{table}

Having investigated the basic word order of all grammatical relations, I now briefly discuss the relation between pairs, namely the order of subject to verb, verb to object, and object to subject. These dual relations confirm the findings of a general S V O (X) word order in Gyeli.

Table \ref{Tab:dualGram} summarizes the relative order of only two grammatical relations. The first column states the grammatical relations whose order are investigated, followed by the total number of occurrences in the corpus. For instance, there are 205 simple verbal clauses which contain a subject and a verb.\footnote{This number can also be deduced from Table \ref{Tab:BCT} where every construction type involves a subject and an object except for the imperative constructions.} Given that there are transitive and intransitive simple verbal clauses, this total number changes for the relation between verb and object which only has 104 occurrences in the corpus; subject to object order can be investigated for 101 instances.


\begin{table}[!h]
\centering
%\scalebox{0.9}{
\begin{tabular}{l|lll}
 \midrule
Grammatical relations & Word order & Frequency &  \\ 
 \midrule
S - V        (205)                  &  S V & 205 & (100\%) \\  \midrule
\multirow{2}{*}{V - O (104)} &  V O & 81 & (77.9\%)  \\
                                  & O V & 23  & (22.1\%) \\  \midrule
\multirow{2}{*}{S - O (101)} &  S O &  95 & (94.1\%) \\
                                  & O S &  6     & (5.9\%) \\
 \midrule
\end{tabular}
\caption{Order of dual grammatical relations}
\label{Tab:dualGram}
\end{table}

In all instances, the subject precedes the verb. In relations between the verb and the object, there are two options for the relative order. In verb - object relations, the verb canonically precedes the object. This is the case for 77.9\% or all verb - object relations. There are a few exceptions, however, where the object precedes the verb. This is the case in left dislocation where the nominal object noun phrase appears even before the subject and in pronominal object fronting. Due to its low frequency and special pragmatic function in terms of information structure, O V order should be considered as non-basic. In addition to this, \citet[80]{dryer2007b} suggests to identify basic word order based on nominal noun phrases rather than pronominal ones. The fact that nominal objects cannot be fronted further indicates the special, rather than basic, order of O V.
Finally, also the relation between subject and object clearly shows that subjects generally precede objects, as in 94.1\% of all subject - object co-occurrences. Again, the only exception to this basic order is related to left dislocation.

In the following subsections, I will give examples of the basic word order types, namely S V, S V O, and S V O\textsubscript{1} O\textsubscript{2}. Note that obliques have been discussed in \sectref{sec:OBL} and will not be subject to further investigation here.

\subsubsection{S V word order}
\label{sec:SV}

Intransitive S V clauses constitute the most frequent construction type in Gyeli simple verbal clauses.
In the most simple case, as in (\ref{SV1}), the clause minimally consists of a zero expressed subject noun phrase and the simple predicate which contains the STAMP marker (with subject reference) and a verb.

\begin{exe}
\ex\label{SV1}
  \glll [$\emptyset$]\textsubscript{S} [á vòdà]\textsubscript{V} \\
	$\emptyset$ a-H vòda \\
        $\emptyset$  1-PRES rest   \\
    \trans `She rests.'
\end{exe}

S V clauses can be more complex than that. For instance, the subject can be expressed by a lexical noun phrase and the verb may be accompanied by aspect marking which, in (\ref{SV2}), appears postverbally.

\begin{exe} 
\ex\label{SV2}
  \glll   [bàNzàmbí bá tè]\textsubscript{S} [bá jìlɛ́ mà]\textsubscript{V} \\
          ba-Nzàmbí bá tè ba-H jìlɛ-H mà \\
              2-PN 2:ATT there 2-PRES sit-R COMPL[Kwasio]   \\
    \trans `The Nzambis there live there already.'
\end{exe}

Also, an S V clause can be expanded by an oblique noun phrase. In (\ref{SV3}), the oblique is a bare locative noun phrase. In addition to the oblique, the verb is also followed by the sentential modifier {\itshape sâ} `only'.\footnote{Sentential modification is discussed in \sectref{sec:SentMod}.}

\begin{exe} 
\ex\label{SV3} 
  \glll [$\emptyset$]\textsubscript{S}  [à tɛ́lɛ́]\textsubscript{V} sâ [dɛ́ndì témɔ́]\textsubscript{X} \\
       $\emptyset$  a tɛ́lɛ-H sâ d-ɛ́ndì témɔ́ \\
       $\emptyset$  1.PST1 stand-R only le5-courtyard middle \\
    \trans `He just stood in the middle of the courtyard.'
\end{exe}

An S V clause can further increase in complexity through auxiliary constructions ({sec:CompPred}), as in (\ref{SV4}).  In this example, the predicate consists of the \textsc{retrospective} aspectual verb {\itshape lɔ́} `come' and the non-finite verb {\itshape njì} `come'. 

\begin{exe} 
\ex\label{SV4}
  \glll [$\emptyset$]\textsubscript{S} [mɛ́ lɔ́ njì]\textsubscript{V} [bàgyɛ̃̂]\textsubscript{X1} [bà wɛ̂]\textsubscript{X2} \\
     $\emptyset$  mɛ-H lɔ́ njì ba-gyɛ̃̂ bà wɛ̂ \\
     $\emptyset$  1\textsc{sg}-PRES RETRO come ba2-stranger AP 2\textsc{sg}  \\
    \trans `I just came as a guest to you.'
\end{exe}

\noindent Also, the clause contains two oblique noun phrases, a bare noun phrase and one with associative plural marker {\itshape bà}.








\subsubsection{S V O word order}
\label{sec:SVO}


S V O word order is found in the corpus in 34.7\% of all simple verbal clauses. Just like S V clauses, their shape differs as well concerning complexity. The clause in (\ref{SVO1}) represents a relatively simple case with a lexical subject noun phrase, including the STAMP marker, a simple predicate, and a lexical object noun phrase.


\begin{exe}
\ex\label{SVO1}
  \glll  [Màmbì]\textsubscript{S} [à dé]\textsubscript{V} [mántúà]\textsubscript{O} \\
	Màmbì à dè-H H-ma-ntúà \\
         $\emptyset$1.PN 1.PST1 eat-PST OBJ.LINK-ma6-mango   \\
    \trans `Mambi ate mangoes.'
\end{exe}

Both subject and object can, however, be also expressed by non-lexical noun phrases. In (\ref{SVO2}), the subject is only expressed by the STAMP marker and the object by a pronoun. 

\begin{exe} 
\ex\label{SVO2}
  \glll  [$\emptyset$]\textsubscript{S} [bwáá lã́]\textsubscript{V} [bɔ̂]\textsubscript{O}  \\
         $\emptyset$ bwáa-H lã-H b-ɔ̂ \\
          $\emptyset$  2\textsc{pl}-PRES tell-R 2-NSBJ     \\
    \trans `You tell them!'
\end{exe}

\noindent (\ref{SVO3}) represents an example of a complex object noun phrase, containing a noun + noun genitive construction with a possessive pronoun.

\begin{exe} 
\ex\label{SVO3} 
  \glll [$\emptyset$]\textsubscript{S} [à nzí kɛ̀]\textsubscript{V} [létsíndɔ́ lé ntùmbà wã̂]\textsubscript{O} \\
      $\emptyset$  a nzí kɛ̀ H-le-tsíndɔ́ lé n-tùmbà w-ã̂ \\
       $\emptyset$  1 PROG.PST1 go OBJ.LINK-le5-funeral.ceremony 5:ATT N1-older.brother 1-POSS.1\textsc{sg}   \\
    \trans `She was going to my older brother's funeral ceremony.'
\end{exe}

S V O clauses can be complex in terms of their predicate. In (\ref{SVO4}), the verb is preceded by a \textsc{progressive} aspect.

\begin{exe} 
\ex\label{SVO4} 
  \glll  [$\emptyset$]\textsubscript{S} [wɛ̀ nzíí bàlɛ̀]\textsubscript{V} [bébã́ã̀]\textsubscript{O} \\
    $\emptyset$  wɛ nzíi-H bàlɛ H-be-bã́ã̀ \\
      $\emptyset$      2\textsc{sg} PROG-PRES keep OBJ.LINK-be8-word \\
    \trans `You are keeping the words.'
\end{exe}

Finally, S V O clauses can be increased in complexity through the addition of oblique noun phrases as with the comitative oblique in (\ref{SVO5}).


\begin{exe} 
\ex\label{SVO5}
  \glll  [mɛ̀gà]\textsubscript{S} [mɛ́ lígɛ́ dè]\textsubscript{V} [mwánɔ̀ wɔ́ɔ̀]\textsubscript{O} [nà màbɔ́'ɔ̀]\textsubscript{X}  \\
        mɛ̀-gà mɛ-H lígɛ-H dè m-wánɔ̀ w-ɔ́ɔ̀ nà ma-bɔ́'ɔ̀ \\
          1/SBJ-CONTR 1\textsc{sg}-PRES stay-R eat N1-child 1-POSS.2\textsc{sg} COM ma6-bread.fruit \\
    \trans `As for me, I stay and eat your child with bread fruit.'
\end{exe}








\subsubsection{S V O O word order}
\label{sec:SVOO}

Double object constructions are rather rare in the corpus with only three instances. As discussed in \sectref{sec:HLinker}, however, each object in a double object construction can occur as first or as second object. This is illustrated in example (\ref{SVOO1}).

\begin{exe}
\ex\label{SVOO1}
\begin{xlist}
\ex\label{SVOO1a}
  \glll  [Àdà]\textsubscript{S} [á líbɛ́lɛ́]\textsubscript{V} [Màmbì]\textsubscript{O1} [màtúà]\textsubscript{O2} \\
         Àdà à-H líbɛlɛ-H Màmbì màtúà \\
        $\emptyset$1.PN 1\textsc{sg}-PRES show-R $\emptyset$1.PN $\emptyset$1.car   \\ 
    \trans `Ada shows Mambi A/THE CAR.'
\ex\label{SVOO1b}
  \glll  [Àdà]\textsubscript{S} [á líbɛ́lɛ́]\textsubscript{V} [màtúà]\textsubscript{O1} [Màmbì]\textsubscript{O2} \\
         Àdà à-H líbɛlɛ-H màtúà Màmbì  \\
       $\emptyset$1. PN 1-PRES show-R $\emptyset$1.car $\emptyset$1.PN    \\ 
    \trans `Ada shows MAMBI a/the car.'
\end{xlist}
\end{exe}

\noindent Pragmatically, the second object position seems to be the focus position. Thus, the choice of which object appears first and which second is conditioned by the information structure of the clause. In (\ref{SVOO1a}), {\itshape màtúà} `car' is in focus, while in (\ref{SVOO1b}) it is the animate object {\itshape Màmbì}.\footnote{Note that I refrain from using the terminology of `direct' and `indirect' objects in Gyeli since they cannot be distinguished on formal grounds. As explained in \sectref{sec:HLinker}, the first object which is closer to the verb receives an object linking H tone if it has a CV- shape noun class prefix while the second does not. When changing positions, still the first object will receive the H tone, but not the second object.}

Just as lexical object noun phrases can appear in both object positions, as in (\ref{SVOO2}), also pronominal objects can occur either in the first or second object position, depending on which object is in focus.  

\begin{exe}
\ex\label{SVOO2}
\begin{xlist}
\ex\label{SVOO2a}
  \glll [$\emptyset$]\textsubscript{S} [mɛ̀ vɛ́]\textsubscript{V} [{\bfseries bá}bwálɛ̀]\textsubscript{O1} [bèfùmbí.]\textsubscript{O2} \\
     $\emptyset$   mɛ vɛ̂-H H-ba-bwálɛ̀ be-fùmbí\\
      $\emptyset$     1\textsc{sg}.PST1 give-R OBJ.LINK-ba2-parent be8-orange  \\
    \trans `I gave the parents ORANGES.'
\ex \label{SVOO2b}
  \glll [$\emptyset$]\textsubscript{S}  [mɛ́ vɛ́]\textsubscript{V} [{\bfseries bé}fùmbí]\textsubscript{O1} [bàbwálɛ̀]\textsubscript{O2} \\
    $\emptyset$   mɛ-H vɛ̂-H H-be-fùmbí ba-bwálɛ̀ \\
      $\emptyset$   1\textsc{sg}-PRES give-R OBJ.LINK-be8-orange ba2-parent  \\
    \trans `I gave THE PARENTS oranges.'
\end{xlist}
\end{exe} 

\noindent In (\ref{SVOO3}), the lexical object noun phrases of (\ref{SVOO2}) are replaced by pronouns. Each of them can occur in either the first or second object position. The second object position is, again, the focus position.

\begin{exe}
\ex\label{SVOO3}
\begin{xlist}
\ex\label{SVOO3a}
  \glll [$\emptyset$]\textsubscript{S} [mɛ̀ vɛ́]\textsubscript{V} [bɔ̂]\textsubscript{O1} [byɔ̂]\textsubscript{O2} \\
       $\emptyset$ mɛ vɛ̂-H b-ɔ̂ by-ɔ̂\\
        $\emptyset$   1\textsc{sg}.PST1 give-R 2-NSBJ 8-NSBJ  \\
    \trans `I gave them [the parents] THEM [the oranges].'
\ex \label{SVOO3b}
  \glll [$\emptyset$]\textsubscript{S}  [mɛ́ vɛ́]\textsubscript{V} [byɔ̂]\textsubscript{O1} [bɔ̂]\textsubscript{O2} \\
      $\emptyset$ mɛ-H vɛ̂-H b-yɔ̂ b-ɔ̂\\
      $\emptyset$   1\textsc{sg}-PRES give-R 8-NSBJ 2-NSBJ \\
    \trans `I gave THEM [the parents] them [the oranges].'
\end{xlist}
\end{exe} 



%\subsubsection{A Note on Valency}
%\label{sec:valency}

%\begin{exe} 
%\ex\label{24}
%\begin{xlist}
%\ex\label{28a}
%  \glll  [à]\textsubscript{S} [bwã̀ã̀.]\textsubscript{V} \\
%         a bwã̀ã̀. \\
%          1.PST1 give.birth \\
%    \trans `She has given birth.'
%\ex\label{28b}
 % \glll  [Nzàmbí à]\textsubscript{S} [bwã̀ã́]\textsubscript{V} [mwánɔ̀.]\textsubscript{O} \\
%          Nzàmbí a bwã̀ã-H m-wánɔ̀ \\
%             PN 1.PST1 give.birth-R N1-child \\
%    \trans `Nzambi has given birth to a child.'
%\end{xlist}
%\end{exe}




%\begin{exe} 
%\ex\label{01}
%\begin{xlist}
%\ex\label{01a}
%  \glll    yɔ́ɔ̀ [yá]\textsubscript{S} [táàlɛ̀.]\textsubscript{V}  \\
%          yɔ́ɔ̀ ya-H táàlɛ̀  \\
%            so 1\textsc{pl}-PRES begin     \\
%    \trans `So, we begin.'
%\ex\label{07}
%  \glll  [nyɛ̀]\textsubscript{S} [táàlɛ́]\textsubscript{V} [bábɛ̀.]\textsubscript{O}      \\
%          nyɛ táàlɛ-H bábɛ̀  \\
%          1.PST1 begin-R $\emptyset$7.illness  \\
%    \trans `He started to be sick.'
%\end{xlist}
%\end{exe}















\subsection{Sentential modification}
\label{sec:SentMod}

Gyeli has a range of sentential modifiers, listed in Table \ref{Tab:SentMod}. They are all monosyllabic and clearly not nouns. These modifiers are special instances of adverbs which, in contrast to adverbs discussed in {sec:ADV}, occur in a preverbal position. As such, they show greater variability in their possible positions. In terms of their function, they modify the action and/or state of the verb.

\begin{table}[!h]
\centering
%\scalebox{0.9}{
\begin{tabular}{llll}
 \midrule 
{\bfseries ndáà} & `also' & 21 & (37.5\%)\\
{\bfseries ná} & `again, still' & 13 & (23.2\%) \\
{\bfseries vɛ̀ɛ̀} & `only, still' & 8 & (14.3\%) \\
{\bfseries kɔ́ɔ̀} & `only, still' & 7 & (12.5\%) \\
{\bfseries sâ} & `only, just' &  5 & (8.9\%) \\
{\bfseries lìí}  & `not yet'    & 2  &  (3.6\%)   \\
 \midrule
Total & & 56 \\
 \midrule
\end{tabular}
\caption{Sentential modifiers}
\label{Tab:SentMod}
\end{table}

Sentential modifiers also play a role in information structure, relating to the expression in focus and affecting the presuppositions of the sentence. For instance, {\itshape ndáà} `also' as an additive particle is used to  ``express that the predication holds for at least one alternative of the expression in focus'' \citep[111]{krifka99}. In contrast,  exclusive particles such as {\itshape vɛ̀ɛ̀}, {\itshape kɔ́ɔ̀}, and {\itshape sâ} ``presuppose that the predication holds for the expression in focus, and assert that it does not hold for any alternative'' ({\itshape Ibid.}). 

{\itshape ná} `again, still' can be used for both verbs and other grammatical relations. Further, {\itshape vɛ̀ɛ̀} and {\itshape kɔ́ɔ̀} can introduce subordinate clauses, similar to the negation particle {\itshape tí}, acting as a sequential marker. These constructions are discussed in {sec:InfSub}. Finally, {\itshape lìí} `not yet' not only modifies verbs, but it is a negative polarity item. As such, it interacts with tense-mood and polarity categories which goes beyond just modifying a verb.  

The most frequent sentential modifier in the Gyeli corpus is {\itshape ndáà} `also', constituting 37.5\% of all sentential modifiers. Table \ref{Tab:SentMod} lists modifiers in decreasing frequency. Thus, the second most frequent modifier is {\itshape ná} `again, still' which is translated as {\itshape encore} into French. The modifiers {\itshape vɛ̀ɛ̀}, {\itshape kɔ́ɔ̀}, and {\itshape sâ} are about equally frequent. In terms of their semantics, they are difficult to distinguish. They definitely have some overlap and speakers often state that one can be used interchangeably for the other. Typically, they are translated as either {\itshape seulement} or {\itshape toujours} into Cameroonian French.
Examples of each sentential modifier and its range of use is given in the following.


\paragraph{{\itshape ndáà} `also'} The sentential modifier {\itshape ndáà} `also' generally serves to expand a grammatical relation in terms of information structure. It generally follows the constituent it refers to. Thus, in (\ref{ndaa1}), {\itshape ndáà} follows the lexical subject noun phrase, expanding the subject topic. 

\begin{exe} 
\ex\label{ndaa1} The woman ate mangoes.
   \glll nà [mwánɔ̀ mùdã̂]\textsubscript{S} {\bfseries ndáà} à nzí dè mántúà \\
      nà m-wánɔ̀ m-ùdã̂ ndáà a nzí dè H-ma-ntúà \\
       COM N1-child N1-woman also 1 PROG.PST eat OBJ.LINK-ma6-mango  \\
    \trans `And the girl also ate mangoes.'
\end{exe}

{\itshape ndáà} also occurs directly after verbs, as in (\ref{ndaa2}). In the previous clause, the speaker stated that the Bulu contest the Bagyeli's ownership of their village. Now he expands on what else the Bulu do, namely also bother them.

\begin{exe} 
\ex\label{ndaa2} 
  \glll bvúlɛ̀ bá ntɛ́gɛ́lɛ́ {\bfseries ndáà} bíyɛ̀ \\
       bvúlɛ̀ ba-H ntɛ́gɛlɛ-H ndáà bíyɛ̀ \\
         ba2.Bulu 2-PRES bother-R also 1\textsc{pl}.NSBJ   \\
    \trans `The Bulu bother us, too.'
\end{exe}

\noindent Further, {\itshape ndáà} is used under negation, as in (\ref{ndaa3}). 

\begin{exe} 
\ex\label{ndaa3}
  \glll ká wɛ̀ɛ́ wúmbɛ́lɛ́ {\bfseries ndáà} mɛ́ nɔ̀ɔ́ nkwɛ̂ wá mábɔ́'ɔ̀ \\
        ká wɛ̀ɛ́ wúmbɛ-lɛ́ ndáà mɛ-H nɔ̀ɔ̀-H nkwɛ̂ wá H-ma-bɔ́'ɔ̀ \\
         if 2\textsc{sg}.PRES.NEG want-NEG also 1\textsc{sg}-PRES take-R $\emptyset$3.basket 3:ATT OBJ.LINK-ma6-bread.fruit\\
    \trans `if you don't want [this] either, I take the basket with the bread fruit.'
\end{exe}

{\itshape ndáà} also occurs phrase-finally, as in (\ref{ndaa4}). Here, it modifies the copula complement {\itshape kùrã̂} `electricity', which is one of the things, among others, that the Bagyeli wish to obtain.

\begin{exe} 
\ex\label{ndaa4} 
  \glll  yá wúmbɛ́ {\bfseries ndáà} náà bí bɔ́gà yá pángɔ́ bɛ̀ nà kùrã̂ {\bfseries ndáà} \\
         ya-H wúmbɛ-H ndáà nâ bí b-ɔ́gà ya-H pángɔ-H bɛ̀ nà kùrã̂ ndáà \\
        1\textsc{pl}-PRES want-R also COMP 1\textsc{pl}.EMPH 2-other 1\textsc{pl}-PRES  PRIOR[Kwasio]-R be  COM $\emptyset$7.electricity also\\
    \trans `We also want that we others first have also electricity.'
\end{exe}


\paragraph{{\itshape ná}  `again'}
The sentential modifier {\itshape ná} is mostly translated as {\itshape encore} into Cameroonian French, but in some contexts also as {\itshape toujours}, roughly translating to `still' and `again' in English. {\itshape ná} mostly occurs directly after the verb. If the clause contains a complex predicate with an auxiliary, the sentential modifier occurs between the auxiliary and the main verb, as in (\ref{na1}) with a modal auxiliary and (\ref{na2}) with an aspectual auxiliary.

\begin{exe} 
\ex\label{na1} 
  \glll  wɛ́ yànɛ́ {\bfseries ná} gyàgà ndísì \\
     wɛ-H yànɛ-H ná gyàga ndísì \\
        2\textsc{sg}-PRES must-H again buy $\emptyset$3.rice  \\
    \trans `You must again buy rice.'
\end{exe}

\begin{exe} 
\ex\label{na2}
  \glll mɛ́ pã́ {\bfseries ná} kɛ̀ dígɛ̀ mùdì wà nû ɛ́ pɛ́ɛ́\\
        mɛ-H pã̂-H ná kɛ̀ dígɛ m-ùdì wà nû ɛ́ pɛ́-ɛ́ \\
        1\textsc{sg}-PRES PRIOR-H again go see N1-person 1:ATT 1.DEM.PROX LOC over.there.DIST \\
    \trans `I try again and go see this person over there.'
\end{exe}

When {\itshape ná} follows negation, as in (\ref{na3}), its meaning is `anymore'. Thus, comparable to {\itshape ndáà} under negation, no negative polarity item is required.

\begin{exe} 
\ex\label{na3}
  \glll  mɛ̀ɛ̀ kálɛ̀ {\bfseries ná} bɛ̀ nà jí ɛ́ vâ \\
        mɛ̀ɛ̀ kálɛ̀ ná bɛ̀ nà jí ɛ́ vâ \\
           1\textsc{sg}.FUT NEG.FUT anymore be COM $\emptyset$7.place LOC here  \\
    \trans `I won't have a place here anymore.'
\end{exe}


\noindent In non-verbal predicates, {\itshape ná} follows the STAMP copula, as in (\ref{na4}). 

\begin{exe} 
\ex\label{na4}
  \glll  bɔ́nɛ́gá báà {\bfseries ná} jìí dé tù \\
        b-ɔ́nɛ́gá báà ná jìí dé tù \\
          2-other 2.COP still $\emptyset$7.forest LOC inside  \\
    \trans `The others are still in the forest.'
\end{exe}

{\itshape ná} further occurs frequently at the end of a phrase. For example, in (\ref{na5}), {\itshape ná} follows the object rather than the verb. While the modifier could also appear after the auxiliary, the choice of a phrase-final position in this instance is most likely related to information structure, making {\itshape bényámɛ̀} more salient. This, however, requires further investigation.

\begin{exe} 
\ex\label{na5} 
  \glll  ónóò bí bɔ́gà yá pã́ jî bényámɛ̀ {\bfseries ná} \\
         ónóò bí b-ɔ́gà ya-H pã̂-H jî H-be-nyámɛ̀ ná \\
         EXCL 1\textsc{pl}.EMPH 2-other 1\textsc{pl}-PRES start-R stay OBJ.LINK-be8-poor still \\
    \trans `Ohhh, we other will first stay still poor.'
\end{exe}

{\itshape ná} can co-oocur with other sentential modifiers, such as {\itshape ndáà} `also'. In this case, {\itshape ná} follows {\itshape ndáà}, as shown in (\ref{na6}).

\begin{exe} 
\ex\label{na6} 
  \glll  bwánɔ̀ bá bùdã̂ bábáà èè nà mwánɔ̀ wà mùdã̂ nláálɛ̀ ndáà {\bfseries ná} \\
         b-wánɔ̀ bá b-ùdã̂ bá-báà èè nà m-wánɔ̀ wà m-ùdã̂ nláálɛ̀ ndáà ná \\
         ba2-child 2:ATT ba2-woman 2-two EXCL COM N1-child 1:ATT N1-woman three also again   \\
    \trans `Two girls, yes, and also again a third girl.'
\end{exe}


There are a few cases where {\itshape ná} appears twice in a clause. In (\ref{na7}), the modifiers occurs after the auxiliary as well as phrase-finally.

\begin{exe} 
\ex\label{na7}
  \glll     áà mɛ̀ nzíí {\bfseries ná} làwɔ̀ {\bfseries ná} \\
            áà mɛ nzíí ná làwɔ ná \\
              yes 1\textsc{sg} PROG.PRES still talk still \\
    \trans `Yes, I am still talking.'
\end{exe}


Finally, {\itshape ná} can also occur preverbally, as in (\ref{na8}). Here, it follows the subject {\itshape wɛ́} `you' (while the other instances of {\itshape ná} in the clause follow the verb.) 

\begin{exe} 
\ex\label{na8} 
  \glll wɛ́ {\bfseries ná} báàlá nà nyɛ́ fí nà wɛ́ ndyándyá {\bfseries ná} sálɛ́ ɛ́ pɛ̀ nà wɛ́ kòlá {\bfseries ná} mɔ̀nɛ́ nû \\
      wɛ-H ná báàla-H nà nyɛ̂-H fí nà wɛ-H ndyándya-H ná sálɛ́ ɛ́ pɛ̀ nà wɛ-H kòla-H ná mɔ̀nɛ́ nû \\
         2\textsc{sg}-PRES again repeat-R COM see-R different COM 2\textsc{sg}-PRES work-R again $\emptyset$7.work LOC over.there COM 2\textsc{sg}-PRES add-R again $\emptyset$1.money 1.DEM.PROX \\
    \trans `You repeat again and see differently [= find another work] and you do again work there and you add again this money [= same amount of 250 Francs].'
\end{exe}

\noindent Instances of {\itshape ná} following the STAMP marker seem to be rather rare, however, at least rarer than {\itshape ndáà} `also' modifying noun phrases. 



\paragraph{{\itshape vɛ̀ɛ̀} `only, still'} 
In contrast to {\itshape ndáà} `also' and {\itshape ná} `again', {\itshape vɛ̀ɛ̀} `only, still' generally has scope over the constituents that follow the modifier. This may either be a noun phrase, a verb, or the whole sentence. At the same time, {\itshape vɛ̀ɛ̀} seems to acquire different meanings in different contexts, as we will see below. Even though it is beyond the scope of this work to disentangle the entire semantic range of sentential modifiers, it seems that {\itshape vɛ̀ɛ̀} has a restrictive function when it has scope over single constituents of the sentence. In contrast, when it has scope over the whole sentence, it seems to rather function as a sequential marker connecting subsequent events and adding a dramaturgic aspect.


In (\ref{vee1}) and (\ref{vee2}), {\itshape vɛ̀ɛ̀} appears phrase-initially. In both cases, it has a restrictive meaning which can truly be translated as `only' in the sense of `nothing but'.

\begin{exe} 
\ex\label{vee1} 
  \glll  mɛ̀ nyɛ́ kwádɔ́ yî Kúndúkùndù {\bfseries vɛ̀ɛ̀} màndáwɔ̀ má zì mɔ̂ nà mɔ̂ \\
         mɛ nyɛ̂-H kwádɔ́ yî Kúndúkùndù vɛ̀ɛ̀ ma-ndáwɔ̀ má zì m-ɔ̂ nà m-ɔ̂ \\
            1\textsc{sg}.PST1 see-R $\emptyset$7.village 7.DEM.PROX $\emptyset$7.PN only ma6-house 6:ATT $\emptyset$7.tin 6-NSBJ COM 6-NSBJ  \\
    \trans `I saw this village, Kundukundu. Only tin (roofed) houses, each and each.'
\end{exe}

In (\ref{vee1}), the {\itshape vɛ̀ɛ̀} modifies {\itshape màndáwɔ̀ má zì} `tin houses' (in contrast to houses with raffia roofs). In (\ref{vee2}), it refers to {\itshape nàmɛ́nɔ́} `tomorrow'.

\begin{exe} 
\ex\label{vee2}
  \glll   {\bfseries vɛ̀ɛ̀} nàmɛ́nɔ́ nàmɛ́nɔ́ nà pámò dẽ̀ \\
       vɛ̀ɛ̀ nàmɛ́nɔ́ nàmɛ́nɔ́ nà pámo dẽ̀ \\
            only tomorrow tomorrow COM arrive today \\
    \trans `Only tomorrow, tomorrow, until today. [= only heard promises till today]'
\end{exe}

In (\ref{vee3}), the modifier also appears phrase-initially, but in this instance, it does not have a restrictive meaning and as such does not seem to modify the subject noun phrase. Instead, it seems to rather have scope over the whole sentence and function as a dramatic sequential marker which is best translated as `suddenly' or `unexpectedly'.\footnote{In Cameroonian French, {\itshape vɛ̀ɛ̀} is still translated as {\itshape seulement} `only', but the meaning of {\itshape seulement} in this case is far from being clear.}

\begin{exe} 
\ex\label{vee3}
  \glll   nâ bá dyúù nyɛ̂ {\bfseries vɛ̀ɛ̀} mùdì nyɛ̂ jã́ã̀sà \\
           nâ ba-H dyúù nyɛ̂ vɛ̀ɛ̀ m-ùdì nyɛ̂ jã́ã̀sà      \\
         COMP 2-PRES kill.SBJV 1.NSBJ only N1-person 1 disappear \\
    \trans `That they kill him. Suddenly the person disappears.'
\end{exe}

Another instance of a sequential function is given in (\ref{vee4}). Here, the Nzambi story (see Appendix \ref{sec:Nzambi}) reaches its climax where the protagonist locks his friend's family into a house, pours fuel over the house, takes a lighter and lights it. The phrase in (\ref{vee4}) is the last step in this chain of events, the sentential modifier {\itshape vɛ̀ɛ̀} serving as a sequential marker that seems to express a dramaturgic effect at the same time.

\begin{exe} 
\ex\label{vee4}
  \glll {\bfseries vɛ̀ɛ̀} bɛ́dɛ̀ \\
       vɛ̀ɛ̀ bɛ́dɛ \\
       only light  \\
    \trans `just light [the house].'
\end{exe}


{\itshape vɛ̀ɛ̀} can also precede adverbs which it modifies in a restrictive sense. This is the case for both (\ref{vee5}) and (\ref{vee6}).


\begin{exe} 
\ex\label{vee5}
  \glll ɛ́ vâ màkwɛ̀lɔ̀ má fúgɛ̀ {\bfseries vɛ̀ɛ̀} vâ \\
         ɛ́ vâ ma-kwɛ̀lɔ̀ ma-H fúgɛ vɛ̀ɛ̀ vâ \\
         LOC here ma6-felling 6-PRES end only here \\
    \trans `Here, the felling ends, only here.'
\end{exe}

\begin{exe} 
\ex\label{vee6}
  \glll yɔ́ɔ̀ pɔ̀nɛ̀ {\bfseries vɛ̀ɛ̀} mpù \\
        yɔ́ɔ̀ pɔ̀nɛ̀ vɛ̀ɛ̀ mpù \\
         7.COP $\emptyset$7.truth still like.this \\
    \trans `It is still true like this.'
\end{exe}

In some instances, the modifier seems to pick out a whole verb phrase (i.e.\ verb plus noun phrase) while actually restricting only the noun phrase. This is the case in (\ref{vee7}) where {\itshape vɛ̀ɛ̀} precedes the verb, but in terms of its meaning, it rather serves as a restriction to the object {\itshape mímpìndí} `non-ripe': in contrast to falling ripe, the palm nuts only fall non-ripe.


\begin{exe} 
\ex\label{vee7} 
  \glll  màlɛ́ndí máà {\bfseries vɛ̀ɛ̀} kwè mímpìndí \\
         ma-lɛ́ndí máà vɛ̀ɛ̀ kwè H-mi-mpìndí \\
           ma6-palm.tree 6.DEM.PROX only fall OBJ.LINK-mi4-non.ripe \\
    \trans `These palm trees only fall non-ripe [fruit].'
\end{exe}






\paragraph{{\itshape kɔ́ɔ̀} `still, just'}
The sentential modifier {\itshape kɔ́ɔ̀} has some functional and semantic overlap with both {\itshape vɛ̀ɛ̀} and {\itshape sâ}. Therefore, it is hard to distinguish the functional and semantic range of these three modifiers. {\itshape kɔ́ɔ̀} has in common with {\itshape vɛ̀ɛ̀} that both can be used as a sequential marker which have scope over a whole sentence rather than single constituents. This is the case, for instance, in (\ref{koo1}) where {\itshape kɔ́ɔ̀} links an event within a chain of events. Nzambi locks his friend's family into a house, pours fuel over the house and the takes a lighter---the following event is introduced with {\itshape vɛ̀ɛ̀} as explained in (\ref{vee4}).

\begin{exe} 
\ex\label{koo1}
  \glll {\bfseries kɔ́ɔ̀} nɔ̀ɔ̀ brìkɛ̂ wɛ̂ \\
        kɔ́ɔ̀ nɔ̀ɔ̀ brìkɛ̂ w-ɛ̂ \\
      just take $\emptyset$1.lighter[French] 1-POSS.3\textsc{sg} \\
    \trans `just takes his lighter,'
\end{exe}

In (\ref{koo2}), the speaker wraps up a conversation by stating that they were three people who spoke and then finished. As such, {\itshape kɔ́ɔ̀} again more serves as a sequential marker rather than a restrictive modifier.

\begin{exe} 
\ex\label{koo2} 
  \glll  {\bfseries kɔ́ɔ̀} sílɛ̀ \\
        kɔ́ɔ̀ sílɛ \\
       just finish   \\
    \trans `Just finish.'
\end{exe}

As a second function, {\itshape kɔ́ɔ̀} is also used for restricting information. Thus, the statement in (\ref{koo3}), `The woman bought oranges and beans for the children.' is corrected, noting that only oranges have been bought. In this case, the modifier precedes the constituent it modifies, namely {\itshape befùmbí} `oranges'. As (\ref{koo3a}) and (\ref{koo3b}) illustrate, the modifier always precedes the object noun phrase, no matter whether it occurs as first or second object.

\begin{exe} 
\ex\label{koo3} The woman bought oranges and beans for the children.
\begin{xlist}
\ex\label{koo3a}
   \glll tɔ̀sâ, à nzí gyàgà {\bfseries sâ/kɔ́ɔ̀} béfùmbí bwánɔ̀ \\
          tɔ̀sâ a nzí gyàga {\bfseries sâ/kɔ́ɔ̀} H-be-fùmbí b-wánɔ̀ \\
        no 1 PROG.PST buy only OBJ.LINK-be8-orange ba2-child \\
    \trans `No, she bought only oranges for the children.'
\ex \label{koo3b}
   \glll tɔ̀sâ, à nzí gyàgà b-wánɔ̀ {\bfseries sâ/kɔ́ɔ̀} bè-fùmbí \\
     tɔ̀sâ a nzí gyàga b-wánɔ̀ {\bfseries sâ/kɔ́ɔ̀} be-fùmbí \\
        no 1 PROG.PST buy ba2-child only be8-orange  \\
    \trans `No, she bought only oranges for the children.'
\end {xlist}
\end{exe}

\noindent What this example also shows is that the modifiers {\itshape kɔ́ɔ̀} and {\itshape sâ} can be used interchangeably in this context, namely whenever {\itshape kɔ́ɔ̀} expresses restriction. Also (\ref{koo4}) represents such a case. When Nzambi realizes that his family has been killed, he just cries (and does not do anything else). 

\begin{exe} 
\ex\label{koo4}
  \glll Nzàmbí wà nû {\bfseries kɔ́ɔ̀} kìyà léwê \\
         Nzàmbí wà nû kɔ́ɔ̀ kìya H-le-wê \\
        $\emptyset$1.PN 1:ATT 1.DEM.PROX only give OBJ.LINK-le5-cry  \\
    \trans `This Nzambi only gives a cry.'
\end{exe}

In other contexts, {\itshape kɔ́ɔ̀} seems to be less restrictive in its function, but expresses something like `just' or `simply' in English. This is the case in (\ref{koo5}), which is certainly not restrictive since the Bagyeli state that they also wish for other improvements, for instance tin roofs.

\begin{exe} 
\ex\label{koo5}
  \glll nà bí bɛ́sɛ̀ {\bfseries kɔ́ɔ̀} kùrã̂  bɛ̀ dé tù \\
       nà bí b-ɛ́sɛ̀ kɔ́ɔ̀ kùrã̂  bɛ̀ dé tù \\
        COM 1\textsc{pl}.EMPH 2-all just $\emptyset$7.electricity  be LOC inside \\
    \trans `with all of us just electricity be inside.'
\end{exe}

Another way of translating {\itshape kɔ́ɔ̀} into Cameroonian French is {\itshape toujours} `still', which applies in examples such as (\ref{koo6}) and (\ref{koo7}). In both cases, the function of {\itshape kɔ́ɔ̀} is to take up a previous discourse topic and re-introduce it.\footnote{An English translation with `just' also seems plausible and the exact difference between `just' and `still' in these contexts is hard to grasp. Speakers, however, make a difference whether they use {\itshape seulement} `only' or {\itshape toujours} `still' in their translations.}

\begin{exe} 
\ex\label{koo6} 
  \glll  yá mbàà yá mbàà yíì nâ {\bfseries kɔ́ɔ̀} mpù ɛ́ nzìwù lɔ́ táálɛ̀ làwɔ̀ nâ bon \\
         yá mbàà yá mbàà yíì nâ kɔ́ɔ̀ mpù ɛ́ nzìwù lɔ́ táálɛ làwɔ nâ bon \\
         7:ATT second 7:ATT second 7.COP COMP still like.this LOC $\emptyset$1.PN RETRO begin talk COMP good[French]   \\
    \trans `The second, the second is that still as Nze just began to say that, good,'
\end{exe}

\begin{exe} 
\ex\label{koo7} 
  \glll yíì pɔ́nɛ́ {\bfseries kɔ́ɔ̀} lèváá lɛ̀vúdũ̂ nâ bí bá ntɛ́gɛ́lɛ́ bágyɛ̀lì \\
       yíì pɔ́nɛ́ kɔ́ɔ̀ le-váá lɛ̀-vúdũ̂ nâ b-í ba-H ntɛ́gɛlɛ-H H-ba-gyɛ̀lì \\
         7.COP $\emptyset$7.truth still le5-thing 5-one COMP ba2-non.Bagyeli 2-PRES bother-R OBJ.LINK-ba2-Gyeli \\
    \trans `It is true, still the same thing that the non-Bagyeli bother the Bagyeli.'
\end{exe}

Finally, {\itshape kɔ́ɔ̀} seems to express some kind of irrealis modality, as in (\ref{koo8})

\begin{exe} 
\ex\label{koo8}
  \glll  {\bfseries kɔ́ɔ̀} nyɛ́gà á làwɔ́ ndáà \\
         kɔ́ɔ̀ nyɛ́-gà a-H làwɔ-H ndáà \\
         only 1-CONTR 1-PRES speak-R also \\
    \trans `If only him, he would also speak.'
\end{exe}


\noindent For a better understanding of the use and semantic range, a much larger corpus is needed as well as a more systematic investigation of sentential modifiers.




\paragraph{{\itshape sâ} `only'} 
The primary function of the modifier {\itshape sâ} is restrictive, as already seen in (\ref{koo3}). {\itshape sâ} seems to only have scope over single constituents in a clause rather than over the whole sentence. It immediately precedes the constituent that it modifies. In (\ref{sa1}), for instance, {\itshape sâ} precedes the oblique noun phrase {\itshape nà màlɛ́ndí} `from palm trees'. In terms of its meaning, {\itshape sâ} restricts the interpretation to this noun phrase, i.e.\ Nzambi only lives from palm trees and no other crops.

\begin{exe} 
\ex\label{sa1}
  \glll nyɛ̀gà váà nyɛ̀gá tsíyɛ́ {\bfseries sâ} nà màlɛ́ndí màlɛ́ndí máà mɔ́gà \\
         nyɛ-gà váà nyɛ-gá tsíyɛ́ sâ nà ma-lɛ́ndí ma-lɛ́ndí máà mɔ́-gà \\
          1-CONTR here 1-CONTR live-R only COM 6-palm.tree 6-palm.tree 6:DEM 6-CONTR \\
    \trans `Him here, he lives only from palm trees, these palm trees.'
\end{exe}

In (\ref{sa2}), the {\itshape sâ} restricts the object interpretation and thus precedes the object noun phrase {\itshape mwánɔ̀ wɔ́ɔ̀} `your child'. Nzambi asks his friend's wife for her child in return for food. In this example, he restricts the payment for food to her child, rather than accepting money or other goods in return.

\begin{exe} 
\ex\label{sa2}
  \glll  vɛ̂ mɛ̂ {\bfseries sâ} mwánɔ̀ wɔ́ɔ̀ wà wɛ̀ bùdɛ́ nû \\
         vɛ̂ mɛ̂ sâ m-wánɔ̀ w-ɔ́ɔ̀ wà wɛ bùdɛ-H nû \\
          give.IMP 1\textsc{sg}.NSBJ only N1-child 1-POSS.2\textsc{sg} 1:ATT 2\textsc{sg} have-R 1:DEM.PROX \\
    \trans `Give me only your child that you have here.'
\end{exe}

{\itshape sâ} can also modify adverbs, as in (\ref{sa3}).  The implicit contrast of the restriction is `here' as opposed to some other place. Thus, the speaker emphasizes that he stays only in the same place and does not go elsewhere so that his relatives are encouraged to join him in his village.

\begin{exe} 
\ex\label{sa3}
  \glll  ká wɛ́ nyɛ́ mɛ̂ jíì {\bfseries sâ} vâ nâ bá nzíyɛ̀ bá nzíyɛ̀ jìyɔ̀ \\
       ká wɛ-H nyɛ̂-H mɛ̂ jíì sâ vâ nâ ba-H nzíyɛ̀ ba-H nzíyɛ̀ jìyɔ \\
         if 2\textsc{sg}-PRES see-R 1\textsc{sg}.NSBJ stay only here COMP 2-PRES come.SBJV 2-PRES come.SBJV stay   \\
    \trans `When you see me staying only here, so that they come, they come to stay.'
\end{exe}


While {\itshape sâ} is observed in the vast majority of cases to have a restrictive function, there are, however, non-restrictive uses which more convey the sense of `just/simply'. In (\ref{sa4}), there is no restriction on the following locative noun phrase, nor on any other constituent of the phrase.

\begin{exe} 
\ex\label{sa4} 
  \glll à tɛ́lɛ́ {\bfseries sâ} dɛ́ndì témɔ́ \\
         a tɛ́lɛ-H sâ d-ɛ́ndì témɔ́ \\
        1.PST1 stand-R just le5-courtyard middle \\
    \trans `He just stood in the middle of the courtyard.'
\end{exe}










\paragraph{{\itshape lìí} `not yet'}

The least frequently found sentential modifier in the corpus is {\itshape lìí} which is a negative polarity item only occuring with past negation words. This is confirmed by elicitations, given the scarcity of data in the corpus. As such, it is not just simply an adverb modifying a verb, but also depends on the polarity category. Therefore, I classify it as a sentential rather than a verbal modifier.

{\itshape lìí} directly follows the negation word. As such, it is the only sentential modifier whose occurrence is restricted to one position only. In (\ref{lii1}), the modifier occurs between the negation and the main verb.

\begin{exe} 
\ex\label{lii1}
  \glll  mɛ̀ pálɛ́ {\bfseries lìí} bâ \\
          mɛ pálɛ́ lìí bâ \\
           1\textsc{sg}.PST1 NEG.PST yet marry \\
    \trans `I am not yet married.'
\end{exe}

\noindent The same is true for (\ref{lii2}) which also includes an object, but this does not affect the position of the modifier.

\begin{exe} 
\ex\label{lii2}
  \glll  mɛ̀ pálɛ́ {\bfseries lìí} dè mántúà \\
          mɛ pálɛ́ lìí dè H-ma-ntúà \\
           1\textsc{sg}.PST1 NEG.PST yet eat OBJ.LINK-ma6-mango \\
    \trans `I have not yet eaten the mangoes.'
\end{exe}

{\itshape lìí} has only been observed to occur with the negation word {\itshape pálɛ́}. It is not clear whether it can occur also with the variant {\itshape sàlɛ́}.



\section{Information structure}
\label{sec:IS}

Following \citet[156]{guldemann2015}, information structure 
\begin{quote}
``is about how speakers structurally encode propositional content with respect to their assessment of knowledge that is (not) shared by the interlocutors in a particular communicative situation.''
\end{quote}


\paragraph{Topic}
I follow \citet[312]{dik97} in his definition of topic and topicality who states that
\begin{quote}
``Topicality concerns the status of those entities ``about'' which information is to be provided or requested in the discourse. The topicality dimension concerns the participants in the event structure of the discourse''
\end{quote}
Gyeli uses a variety of strategies to express ``aboutness''. In order to follow a current topic in the discourse, not only single clauses in isolation have to been examined, but their context in the discourse so that given information can be distinguished from new or newly requested information. Therefore, I provide the discourse context of each example either by description or by a sentence in the example line.


\paragraph{Focus}
According to \citet[326]{dik97},
\begin{quote}
``The focal information in a linguistic expression is that information which is relatively the most important or salient in the given communicative setting.''
\end{quote}

\noindent \citet[236]{fiedler2010} note that this relative importance or salience is expressed either by
 ``introducing new information into the discourse (information focus), or by standing in explicit or implicit contrast to a set of comparable alternatives (contrastive focus).'' 

Gyeli has at least three ways of expressing focus, namely a dedicated focus position that is immediately after the verb, fronting of an object pronoun to achieve predicate focus (PCF), and cleft constructions in order to express subject focus. 



Gyeli uses a range of strategies to package information in clauses and discourse. The most important information structure strategies are listed in Table \ref{Tab:IS}. Both topic and focus can be encoded in-situ, optionally through an expanded noun phrase. Left dislocation of object and adjunct noun phrases topicalizes these constituents. Object fronting puts the predicate into focus. And finally, cleft constructions are a focus means for subjects and obliques. Since they constitute a subordinate construction, they are discussed in {sec:cleft}.



\begin{table}[!h]
\centering
\begin{tabular}{p{.8cm}p{.8cm}p{.8cm}p{.8cm}p{.5cm}|l}
 \midrule
\multicolumn{5}{c|}{Word order} & \multicolumn{1}{c}{Information structure effect} \\
 \midrule
 & S\textsubscript{{\textsc{top}}} & V & O\textsubscript{{\textsc{foc}}} & X & basic word order \\
 O\textsubscript{i, {\textsc{top}}} & S & V & O\textsubscript{{\textsc{pro}}, i} & X & object left dislocation $\rightarrow$ object topic \\
X\textsubscript{{\textsc{top}}} & S\textsubscript{{\textsc{top}}} &  V & O &  & adjunct left dislocation $\rightarrow$ adjunct topic \\ 
 & S\textsubscript{{\textsc{top}}} & O\textsubscript{{\textsc{pro}}} & V\textsubscript{{\textsc{foc}}} & X & object pronoun fronting $\rightarrow$ predicate focus   \\

It is & S\textsubscript{{\textsc{foc}}} & [...]\textsubscript{{\textsc{rel}}} & & & cleft construction $\rightarrow$ subject focus \\
It is & X\textsubscript{{\textsc{foc}}} & [...]\textsubscript{{\textsc{rel}}} & & & cleft construction $\rightarrow$ adjunct focus \\
 \midrule
\end{tabular}
\caption{Basics of Gyeli information structure}
\label{Tab:IS}
\end{table}


\noindent This list is not exhaustive. For instance, prosodic means seem to be relevant as well, but this requires further research.  Data on information structure stem both from the questionnaire on information structure (mainly the topic and focus translation tasks) by \citet{skopeteas2006} and the Gyeli corpus.\footnote{Information structure questionnaires turned out to be less successful to elicit relevant data since speakers strongly preferred to give one-word answers or provide pragmatically neutral answers. The corpus, however, in combination with the questionnaires, allow some reliable generalizations on information structure phenomena in Gyeli.} 



 


\subsection{In-situ positions}

Information structure roles can be encoded in-situ through basic word order. According to \citet[159]{guldemann2015}, subjects are often default topics, which conflate  ``topicality with the semantic role of intransitive subject/transitive agent, leaving the scope of assertion over the following material.'' This results, in many languages, in a basic linear IS order template of [[TOP] [FOC]] (ibid.), a generalization that also applies in Gyeli. The default focus position is immediately after the verb. According to \citet[793]{downing2014}, this is typical for Bantu languages where, ``(most) focused constituents, including WH-elements, occur in the immediate after verb (IAV) position, while non-focal information commonly occurs in peripheral positions.''


\subsubsection{In-situ topic}
\label{sec:insituTop}


In-situ subjects are either not marked at all, but zero expressed, as illustrated in \sectref{sec:BasicCT}, or they are specially marked through an extended pronominal noun phrase. The latter is the case in (\ref{Emp1}).
In this example, a new topic is introduced. In the previous sentence, the speaker was talking about the team of linguists who come to his village. Now he changes the topic to the Bagyeli themselves and how they react to their visitors.
\begin{exe} 
\ex\label{Emp1} You come to find us here.\\
 \glll  donc {\bfseries bí} yá táálɛ́ bê yàlànɛ̀ àà\\
       donc bí ya-H táálɛ-H bê yàlanɛ àà\\
       so[French] 1\textsc{pl}.SBJ  1\textsc{pl}-PRES begin-R 2\textsc{pl} respond[Bulu] EXCL\\
    \trans `So we, we start to respond to you, mhm.'
\end{exe}

Also, a subject pronoun can be used with the sentential modifier {\itshape ndáà} `also', as in (\ref{Emp3}). The chief of Ngolo addresses the Ngumba and Mabi speakers among the visitors. He points out that they as well, in addition to the European people in the group, also speak French (while he does not). 

\begin{exe} 
\ex\label{Emp3} 
  \glll  ɛ̀sɛ́ {\bfseries béé} {\bfseries ndáà} bèyá làwɔ́ fàlà\\
       ɛ̀sɛ́ béé ndáà bèya-H làwɔ-H fàlà\\
        is.it[French] 2\textsc{pl}.SBJ also 2\textsc{pl}[Kwasio]-PRES speak-R $\emptyset$1.French\\
    \trans `Isn't it, you, you also speak French.'
\end{exe}

Often, the subject pronoun is combined with the contrastive marker -{\itshape ga}, indicating a contrastive topic, as in (\ref{Emp2}). The speaker talks about NGOs and white people who receive money in Europe to help Africans. Assuming that other people in Africa profit from this money, he now states that the people in Ngolo also want to receive help for obtaining electricity, where the marker -{\itshape ga} contrasts the Bagyeli to other African communities.

\begin{exe} 
\ex\label{Emp2} White people working for NGOs receive money in Europe.\\
  \glll  {\bfseries bí} {\bfseries bɔ́gà} yá wúmbɛ́ ndáà pã́ã̀ nyɛ̂ sâ bá gyíbɔ́ ngyùlɛ̀ wá kùrã̂\\
         bí bɔ́-gà ya-H wúmbɛ-H ndáà pã́ã̀ nyɛ̂ sâ ba-H gyíbɔ-H ngyùlɛ̀ wá kùrã̂\\
          1\textsc{pl}.SBJ 2-CONTR 1\textsc{pl}-PRES want-R also start see $\emptyset$7.thing 2-PRES call-R $\emptyset$3.light 3:ATT $\emptyset$7.electricity\\
    \trans `We others, we also want to first see the thing they call the light of electricity.'
\end{exe}

The marker -{\itshape gà} is used in order to contrast a new subject topic from an old one. For instance, in (\ref{ga1}), the speaker talks about the problems the Bagyeli encounter with the Bulu. He states that if a Gyeli person goes hunting on terms of equal sharing with a Bulu person, the Bulu person in turn will deceive him.

\begin{exe} 
\ex\label{ga1} 
  \glll wɛ́ kɛ́ nà nyɛ̂ nkɔ̃̀wáká {\bfseries nyɛ̀gà} à nzíí wɛ̂ vã́ã̀kɛ́ sâ mpù \\
         wɛ-H kɛ̀-H nà nyɛ̂ nkɔ̃̀wáká nyɛ̀-gà a nzíí wɛ̂ vã́ã̀kɛ́ sâ mpù \\
         2\textsc{sg}-PRES go COM 1 equal.sharing 1.SBJ-CONTR 1 PROG.PRES 2\textsc{sg}.NSBJ go[Bulu] do like.this  \\
    \trans `You go with him [= the Bulu] equally sharing. As for him, he is going to do you like this [= tries to trick you].'
\end{exe}

This contrast of subject topics is also illustrated in (\ref{ga2}). Here, Nzambi offers his friend's wife bread fruit in return for her child, specifying the terms of the deal. She will get the bread fruit, while he will eat her child.

\begin{exe} 
\ex\label{ga2} You take the bread fruit.\\
  \glll  {\bfseries wɛ̀gà} wɛ́ kɛ́ nà mɔ̂ {\bfseries mɛ̀gà} mɛ́ lígɛ́ dè mwánɔ̀ wɔ́ɔ̀\\
       wɛ̀-gà wɛ-H kɛ̀-H nà m-ɔ̂ mɛ̀-gà mɛ̀-H lígɛ-H dè m-wánɔ̀ w-ɔ́ɔ̀\\
         2\textsc{sg}.SBJ-CONTR 2\textsc{sg}-PRES go-R COM 6-NSBJ 1.SBJ-CONTR 1\textsc{sg}-PRES stay-R eat N1-child 1-POSS.2\textsc{sg}\\
    \trans `As for you, you take them [= the bread fruit] away. As for me, I stay and eat your child.'
\end{exe}


%\begin{exe} 
%\ex\label{ga3}
%  \glll nyɛ̀gà váà nyɛ̀gá tsíyɛ́ sáà nà màlɛ́ndí, màlɛ́ndí máà mɔ́gà. \\
   %      nyɛ-gà váà nyɛ-gá tsíyɛ́ sáà nà ma-lɛ́ndí, ma-lɛ́ndí máà mɔ́-gà. \\
%          1-CONTR here 1-CONTR live-R only COM 6-palm.tree 6-palm.tree 6:DEM 6-CONTR \\
%    \trans `Him here, he lives only from palm trees, these palm trees.'
%\end{exe}

A final example for the marker -{\itshape gà} is provided in (\ref{ga3}). Again, the speaker contrasts a new subject topic to an old one. The previous topic was himself where he says that he asks his friend for help. As for the friend (`you'), he does not react in the expected way, but causes trouble.

\begin{exe} 
\ex\label{ga3} I send you the message and ask you to help me.\\
  \glll ɛ́ tè {\bfseries wɛ̀gà} wɛ́ njí sâ mbvúndá ɛ́ ndzǐ vâ\\
        ɛ́ tè wɛ̀-gà wɛ-H njì-H sâ mbvúndá ɛ́ ndzǐ vâ\\
        LOC there 2\textsc{sg}.SBJ-CONTR 2\textsc{sg}-PRES come-R do $\emptyset$9.trouble LOC $\emptyset$9.path here\\
    \trans `There you, you come to make trouble on the way here.'
\end{exe}





\subsubsection{{\itshape In-situ} focus}
\label{sec:insitu}

Focus in the immediate-after-verb position seems to be the most common focus strategy in Gyeli for objects and obliques. An example for object focus is given in (\ref{InsiOb}), which is a correction of the clause in (\ref{InsiOa}). 

\begin{exe} 
\ex\label{InsiO} 
\begin{xlist}
\ex\label{InsiOa}
   \glll mùdã̂ à dé mántúà\\
         m-ùdã̂ a dè-H H-ma-ntúà  \\
        N1-woman 1.PST1 eat-R OBJ.LINK-ma6-mango \\
    \trans `The woman ate the MANGOES.'
\ex \label{InsiOb}
  \glll tɔ̀sâ à nzí dè {\bfseries ndísì} \\
     tɔ̀sâ a nzí dè ndísì \\
       no 1 PROG.PST eat $\emptyset$3.rice  \\
    \trans `No, she was eating RICE.'
\end {xlist}
\end{exe}

(\ref{InsiX}) represents an example of in-situ adjunct focus. Here, the oblique noun phrase {\itshape lèwùlà lé vɛ́} `when' occurs in-situ. As explained in \sectref{sec:Questions}, such question noun phrases can also appear phrase-initially, but the general focus position is at the end of a phrase in Gyeli.


\begin{exe} 
\ex\label{InsiX}
  \glll     áá bíì màndáwɔ̀ má zì yáà mɔ̂ fúàlà bwɛ̂ {\bfseries lèwùlà} {\bfseries lé} {\bfseries vɛ́}\\
          áá bíì ma-ndáwɔ̀ má zì yáà m-ɔ̂ fúala bwɛ̂ le-wùlà lé vɛ́\\
              EXCL 1\textsc{pl}.NSBJ ma6-house 6:ATT $\emptyset$7.tin[Bulu] 1\textsc{pl}.FUT 6-NSBJ end receive le5-hour 5:ATT which \\
    \trans `Ah, us, as for the tin houses, WHEN will we receive them?'
\end{exe}

%Finally, also predicate focus can be achieved in-situ, as shown in (\ref{InsiP}). In the answer to the question `What did the woman do with the mangoes?', the predicate {\itshape dè} `eat' appears in-situ, i.e.\ as expected between S and O.

%\begin{exe} 
%\ex\label{InsiP} 
%\begin{xlist}
%\ex\label{InsiPa}
 %  \glll gyí mùdã̂ à sá nà màntúà \\
 %       gyí  m-ùdã̂ a sâ-H nà ma-ntúà  \\
 %       what N1-woman 1.PST1 do-R COM ma6-mango \\
 %   \trans `What did the woman do with the mangoes?'
%\ex \label{InsiPb}
%  \glll à {\bfseries dé} mɔ̂. \\
 %     a dè-H mɔ̂ \\
  %    1.PST1 eat-R 6.OBJ  \\
  %  \trans `She ATE them.'
%\end {xlist}
%\end{exe}








\subsection{Left dislocation}
\label{sec:leftdis}

Left dislocation concerns both object and adjunct noun phrases which can be moved to the left edge of the sentence either in form of a nominal or pronominal noun phrase.


\subsubsection{Left dislocation of nominal noun phrases}
\label{sec:leftdisN}

One means to express topicality is left dislocation. This phenomenon applies mainly to objects.
In nominal object left dislocation, an object noun phrase is left dislocated in front of the subject and later taken up again in-situ by an object pronoun. This is illustrated in (\ref{left1}). Previously to this phrase, the chief of Ngolo talks about how he got injured cutting raffia for his roof. He then changes the topic from `raffia' to `tin-roofed houses' which will prevent future injuries related to cutting raffia. Note that the left dislocated object noun phrase usually occurs with a prosodic break which is indicated by the comma.

\begin{exe} 
\ex\label{left1}
  \glll     áá bíì {\bfseries màndáwɔ̀} {\bfseries má} {\bfseries zì} yáà {\bfseries mɔ́} fúàlà bwɛ̂ lèwùlà lé vɛ́\\
          áá bíì ma-ndáwɔ̀ má zì yáà m-ɔ́ fúala bwɛ̂ le-wùlà lé vɛ́\\
              EXCL 1\textsc{pl}.NSBJ ma6-house 6:ATT $\emptyset$7.tin[Bulu] 1\textsc{pl}.FUT 6-NSBJ end receive le5-hour 5:ATT which \\
    \trans `Ah, us, as for the tin houses, when will we receive them?'
\end{exe}

The same pattern applies in (\ref{left2}) where the speaker talks about the Bulu people. He then changes the topic from the Bulu person to the Gyeli child about whom he says that the Bulu will beat him.

\begin{exe} 
\ex\label{left2}  The Bulu person says that he will quarrel with you [= the Gyeli child].\\
  \glll  pílì {\bfseries mwánɔ̀} {\bfseries bàgyɛ̀lì} àà {\bfseries nyɛ̂} kɛ̀ bíyɔ̀\\
       pílì m-wánɔ̀ ba-gyɛ̀lì àà nyɛ̂ kɛ̀ bíyɔ\\
          when N1-child ba2-Gyeli 1.FUT 1.NSBJ go hit\\
    \trans `At times the Gyeli child, he will go hit it,'
\end{exe}

\noindent While in most cases the left dislocated object is expressed in-situ pronominally, it can also surface lexically, as shown in (\ref{left3}). The discourse context is the same as for (\ref{left1}) where the chief of Ngolo talks about his injury and a scar he got on his forehead. To clarify the source of his scar, he changes the topic to the raffia which he cuts up in the trees. In (\ref{left3}), {\itshape ngùndyá} `raffia' is left dislocated before the subject and the occurs again in its lexical form in-situ.

\begin{exe} 
\ex\label{left3} I think, the machete missed me here [= pointing to his forehead].\\
  \glll    {\bfseries ngùndyá} mɛ́ kɛ́ sɔ́lɛ̀gà {\bfseries ngùndyá} dyúwɔ̀\\
          ngùndyá mɛ-H kɛ̀-H sɔ́lɛga ngùndyá dyúwɔ̀\\
              $\emptyset$9.raffia 1\textsc{sg}-PRES go-R chop $\emptyset$9.raffia on.top\\
    \trans `The raffia, I go to chop the raffia on top.'
\end{exe}

Left dislocation is also used in conjunction with the sentential modifier {\itshape ndáà} `also', as in (\ref{left4}). 

\begin{exe} 
\ex\label{left4} The woman ate the oranges.\\
   \glll  nà {\bfseries màntúà} {\bfseries ndáà} à nzí dè {\bfseries mɔ̂}\\
       nà mà-ntúà ndáà a nzí dè m-ɔ̂\\
       COM ma6-mango also 1 PROG.PST eat 6-NSBJ\\
    \trans `And she also ate mangoes.'
\end{exe}

%While left dislocation of objects is obvious and quite salient---objects usually appear after the verb---left dislocation of subjects is less obvious since they appear at the left edge of the clause anyway. One could argue, however, that left dislocation of subjects takes place with other topic phenomena such as emphatic pronouns and the contrastive -{\itshape ga} which appear before the STAMP marker. It seems that in these instances, there is also a short prosodic break, in contrast to non-dislocated lexical subject noun phrases. Thus, subject topicality achieved by emphatic pronouns and the contrastive marker -{\itshape ga} also involves left dislocation.




Left dislocation can also be achieved through pronouns that combine with an object noun phrase, 
as in (\ref{Emp4}). Nzambi's wife explains to her husbands friend that their fields are not producing enough food. She then changes the topic from the problems in food production to the food itself which she asks the friend for.

\begin{exe} 
\ex\label{Emp4} The field is running out of food.\\
  \glll bèdéwò béndɛ̀ {\bfseries byɔ̀} mɛ́ lɔ́ njì lɛ́bɛ̀lɛ̀ bédéwò bà wɛ̂\\
        be-déwò bé-ndɛ̀ b-yɔ̀ mɛ-H lɔ́ njì lɛ́bɛlɛ H-be-déwò bà wɛ̂\\
           be8-food 8-ANA 8-NSBJ 1-PRES RETRO come  follow be8-food AP 2\textsc{sg}.NSBJ\\
    \trans `This food, I have come to look for the food at your place.'
\end{exe}






\subsubsection{Left dislocation of pronominal noun phrases}
\label{sec:topical}

Left dislocation of pronominal objects which, in contrast to nominal left dislocated objects, are not referenced in-situ again, is often referred to as topicalization.
Thus, in (\ref{Top1}), the object pronoun is left dislocated, but does not occur in-situ after the verb. In this example, the chief of Ngolo talks about his wishes to obtain houses with tin roofs.  He finishes his statements by the summary `This I want.', referring to all the points he brought up about new houses in the village and tin roofs.

\begin{exe} 
\ex\label{Top1} I will build houses in Ngolo, each with a tin roof.\\
  \glll  {\bfseries yɔ́ɔ̀} mɛ́ wúmbɛ́ wû\\
        y-ɔ́ɔ̀ mɛ-H wúmbɛ-H wû\\
         7-NSBJ 1\textsc{sg}-PRES want-R there\\
    \trans `This I want there.'
\end{exe}

In (\ref{Top2}), he similarly talks about a topic, namely a tree that people are going to take down without even asking for permission. He concludes by summarizing the general topic of the tree: `This I have planted.'

\begin{exe} 
\ex\label{Top2} 
  \glll  {\bfseries yɔ́ɔ̀} {\bfseries yɔ́ɔ̀} mɛ̀ jìlɛ́ mà \\
       y-ɔ́ɔ̀ y-ɔ́ɔ̀ mɛ jìlɛ-H mà \\
      7-NSBJ 7-NSBJ 1\textsc{sg}.PST1 place-R COMPL[Kwasio]    \\
    \trans `This, this I have placed [there].'
\end{exe}

While most instances of topicalization seem to involve a pronominal object, as in (\ref{Top1}) and (\ref{Top2}), there are also examples where a lexical object noun phrase is left dislocated, but not cross-referenced in-situ. This is the case in (\ref{Top3}).

\begin{exe} 
\ex\label{Top3} The woman cooked rice for her child.\\
   \glll nà {\bfseries nákúndɛ̀kúndɛ̀} {\bfseries ndáà}, à bíyɛ́lɛ́\\
           nà nákúndɛ̀kúndɛ̀ ndáà a bíyɛlɛ-H\\
       CONJ $\emptyset$1.bean also 1.PST1 cook-PST\\
    \trans `And she also cooked beans.'
\end{exe}











\subsection{Object pronoun fronting}
\label{sec:OBJfront}

The phenomenon of preverbal objects in Benue-Congo languages is extensively discussed by \citet{guldemann2007}. Following him, I propose that the marked preverbal object position moves the object into an extrafocal position, resulting instead in the predicate being in focus. This hypothesis is supported by the fact that only pronominal objects can be fronted before the verb, but not lexical objects. Pronouns usually refer to already given information and are thus less salient in terms of new or contrastive information.

Pronominal objects can be fronted in a way that they occur before a simple predicate, as in (\ref{front1}). While in a pragmatically more neutral clause the object pronoun {\itshape yɔ̂} `it' would occur after the verb, it is here fronted and the predicate appears phrase-finally, making it more salient in terms of information structure. The Nzambi explains to his friend's wife that her child would be very tender when one steams it, wrapped in leaves. He then emphasizes that he will EAT the child, which can be interpreted as an instance of truth value focus, highlighting the truth of his future deeds.


\begin{exe} 
\ex\label{front1} This tender child is good when you wrap it in a  leaf package.\\
  \glll mɛ̀ɛ̀ {\bfseries yɔ̂} {\bfseries dè}\\
         mɛ̀ɛ̀ y-ɔ̂ dè\\
        1\textsc{sg}.FUT 7-NSBJ eat\\
    \trans `I will EAT it [= the child].'
\end{exe}

If a clause contains a complex predicate with an auxiliary, the pronominal object under fronting appears between the auxiliary and the main verb, as shown in (\ref{front2}). The context is the same as in (\ref{front1}). Again, the protagonist of the story stresses what he is going to do with the child, namely eat it. The verb {\itshape dè} `eat' appears in focus position since the pronoun {\itshape nyɛ̂} `him' is defocussed.

\begin{exe} 
\ex\label{front2}
  \glll  mɛ́ lígɛ́ {\bfseries nyɛ̂} {\bfseries dè} \\
        mɛ-H lígɛ-H nyɛ̂ dè \\
         1\textsc{sg}-PRES stay-R 1.NSBJ eat   \\
    \trans `I stay to EAT him [= the child].'
\end{exe}

A similar example is presented in (\ref{front3}). Again, the predicate is complex with an aspectual auxiliary verb that is followed by a pronominal object so that the main verb occurs phrase-finally. Here, the speaker explains the troubles the Bagyeli encounter with their Bulu neighbors.

\begin{exe} 
\ex\label{front3}
  \glll  nyɛ̀ náà à múà {\bfseries wɛ̂} {\bfseries bíyɔ̀} \\
        nyɛ nâ a múà wɛ̂ bíyɔ \\
           1 COMP 1 PROSP 2\textsc{sg}.NSBJ hit \\
    \trans `He [the Bulu person says] that he is about to BEAT you [= the Gyeli person].'
\end{exe}

\noindent He reports that the Bulu often threaten to beat the Bagyeli. With the object pronoun {\itshape wɛ̀} `you' in preverbal position, the verb {\itshape bíyɔ} `hit' is in focus position.








\section{Special clause types}
\label{sec:specialC}

Having investigated the basic word order in simple clauses as well as special constructions relating to information structure, I discuss some special clause types in this section. These include questions, possessor raising, and comparison constructions. 

\subsection{Questions}
\label{sec:Questions}

I distinguish three basic types of questions: i) polar questions, ii) leading questions, and iii) constituent questions (what is also known as wh- questions for English). Generally, polar and leading questions occur in basic word order, but add a question marker either at the beginning or the end of the phrase. Constituent questions, in contrast, are more flexible with respect to the occurrence of the interrogative. I will discuss each of these types in turn,
basing my analysis both on the question types questionnaire developped by \citet{patin2011} as well as questions occurring in the Gyeli corpus.

\paragraph{Polar questions with {\itshape nà(nâ)}}

Polar questions are those which typically entail a yes or no answer.
They are usually marked by the question marker {\itshape nà} or {\itshape nànâ} which grammatically marks a sentence as a question. The first version is the shorter default form {\itshape nà}, as shown in (\ref{PQ1}), which also has a longer emphatic form {\itshape nànâ}, as in (\ref{PQ2}). Both only occur at the beginning of a phrase. 

\begin{exe}
\ex\label{PQ1}
 \glll     {\bfseries nà} wɛ̀ nyɛ́ nyɛ̂  \\
            nà wɛ nyɛ̂-H nyɛ̂ \\
             Q 2\textsc{sg}.PST1 see-R 1.NSBJ  \\
    \trans `Did you see him?'
\end{exe}

The emphatic question marker {\itshape nànâ} in polar questions pragmatically expresses insistence or even disbelief. Thus, in (\ref{PQ2}), the speaker who asks the question rather expects the addressee to not have seen the person in question and insists on getting a true answer. 

\begin{exe}
\ex\label{PQ2}
  \glll    {\bfseries nànâ} wɛ̀ nyɛ́ nyɛ̂ \\
            nànâ wɛ nyɛ̂-H nyɛ̂ \\
             Q 2\textsc{sg}.PST1 see-R 1.NSBJ \\
    \trans `Did you really see him?'
\end {exe}

Prosody does not seem to play a role in terms of indicating a question. Therefore, question markers are the only means to mark questions clearly as such, especially in polar questions which do not employ any other question indicating devices, in contrast to constituent questions which use interrogatives. Nevertheless, the use of question markers is not obligatory, not even in polar questions, as shown in (\ref{PQ3}). In this example, it has to be clear from the context, however, that the sentence is a question. Otherwise, {\itshape nà} as in (\ref{PQ1}) has to be used.

\begin{exe}
\ex\label{PQ3}
  \glll     wɛ̀ nyɛ́ nyɛ̂ \\
             wɛ nyɛ̂-H nyɛ̂ \\
               2\textsc{sg}.PST1 see-R 1.NSBJ \\
    \trans `Did you see him?'
\end {exe}

In addition to their syntactic function of marking a phrase as a question, question markers also have a pragmatic function. In contexts where it is clear that a phrase is meant as a question and {\itshape nà} is still used, the question marker serves as marking emphasis. For instance, (\ref{PQ1}) could also be translated as `Did you really see him?', just as in (\ref{PQ2}). Using the longer form {\itshape nànâ}, as in (\ref{PQ2}), is even more emphatic and indicates the speakers disbelief: speakers would also translate the question in (\ref{PQ2}) as `Are you sure that you saw him?'

 {\itshape nà} can also co-occur with interrogatives, as shown in (\ref{PQ4}). {\itshape nà} is not required to indicate that the sentence is a question since this is already achieved through the interrogative construction {\itshape púù yá gyí} `why'. It seems, however, that {\itshape nà} here has an emphasizing function.%\footnote{Note that this question was elicited with the question type questionnaire rather than steming from natural discourse. Therefore, I exclude the possibilty that the question marker {\itshape nà} here could be the homophonic comitative marker {\itshape nà} `and/with'.}

\begin{exe}
\ex\label{PQ4}
  \glll     {\bfseries nà} {\bfseries púù} {\bfseries yá} {\bfseries gyí} wɛ̀ pálɛ́ gyàgà mányâ \\
         nà púù yá gyí wɛ̀ pálɛ́ gyàga H-ma-nyâ \\
               Q $\emptyset$7.reason 7:ATT what 2\textsc{sg}.PST1 NEG.PST buy OBJ.LINK-ma6-milk \\
    \trans `Why didn't you buy milk?'
\end {exe}




\paragraph{Leading questions with {\itshape ngáà}}

\noindent The question marker {\itshape ngáà} is used for leading questions, i.e.\ polar questions which lead the addressee to give a specific yes or no answer, as expected by the speaker. {\itshape ngáà} roughly corresponds to {\itshape n'est-ce pas} in French and {\itshape right?} or {\itshape isn't it?} in English, which are sometimes also referred to as tag questions. I therefore gloss {\itshape ngáà} as `Q(tag)'. Just like the question marker {\itshape nànâ}, {\itshape ngáà} has both a syntactic and pragmatic function. Syntactically, it encodes question marking. Pragmatically, it leads the addressee to give an expected answer. In contrast to {\itshape nà(nâ)}, {\itshape ngáà} can occur both at the beginning and the end of a question, as shown in (\ref{LQ1}). The expected answer to the questions in (\ref{LQ1}) would be `yes'.

\begin{exe}
\ex\label{LQ1}
\begin{xlist}
\ex \label{LQ1a}
  \gll     wɛ̀ nyɛ́ nyɛ̂ {\bfseries ngáà} \\
               2\textsc{sg}.PST1 see 1.NSBJ  Q(tag)\\
    \trans `You saw him, didn't you/right?'
\ex\label{LQ1b}
 \gll     {\bfseries ngáà} wɛ̀ nyɛ́ nyɛ̂  \\
             Q(tag) 2\textsc{sg}.PST1 see 1.NSBJ   \\
    \trans `Right, you saw him?'
\end {xlist}
\end {exe}

\noindent {\itshape ngáà} is used in the same form for negated questions, as shown in (\ref{LQ2}). Here, the expected answer would be `no'.

\begin{exe}
\ex\label{LQ2}
\begin{xlist}
\ex \label{LQ2a}
  \glll     wɛ̀ nyɛ́lɛ́ nyɛ̂, {\bfseries ngáà} \\
              wɛ nyɛ̂-lɛ nyɛ̂ ngáà \\
               2\textsc{sg}.PST1 see-NEG 1.NSBJ Q(tag) \\
    \trans `You didn't him, did you?'
\ex\label{LQ2b}
 \glll     {\bfseries ngáà}, wɛ̀ nyɛ́lɛ́ nyɛ̂  \\
          ngáà wɛ nyɛ̂-lɛ nyɛ̂ \\
             Q(tag) 2\textsc{sg}.PST1 see-NEG 1.NSBJ  \\
    \trans `Right, you didn't see him?'
\end {xlist}
\end {exe}


\noindent  In contrast to constituent questions, {\itshape ngáà} does not co-occur with {\itshape nà} in the same question. 







\paragraph{Constituent questions}
Constituent questions are expressed by interrogatives. Subject and object questions employ the interrogative pronouns {\itshape nzá} `who' for human/animate and {\itshape gyí} `what' for inanimate entities. Adjunct questions use a range of interrogatives such as {\itshape ɛ́ vɛ́} `where' and oblique noun phrases, such as {\itshape dúbɔ̀ lé vɛ́} `when [= which day]', {\itshape wùlà yá vɛ́} `when [= what time]' and {\itshape púù yá gyí} `why [= what reason]'. I will discuss the various constituent question types sorted by constituent, starting out with subject questions. 


Subject interrogative pronouns always occur in-situ, i.e.\ phrase-initially. An example of a subject question using the human/animate interrogative pronoun {\itshape nzá}  `who' is given in (\ref{SCQ1}).


\begin{exe}
\ex\label{SCQ1}
  \glll     {\bfseries nzá} nzí nyɛ̂ Màmbì \hfill S V O\\
             nzá nzí nyɛ̂ Màmbì \\
               who PROG.PST see PN\\
    \trans `Who saw Mambi?'
\end {exe}

\noindent (\ref{SCQ2}) provides an example for a question asking for an inanimate subject, thus using {\itshape gyí} `what'.

\begin{exe}
\ex\label{SCQ2}
 \glll    {\bfseries gyí} nzí bvúɔ̀ kàsà \hfill S V O\\
           gyí nzî́ bvúɔ̀ kàsà\\
             what PROG.PST break $\emptyset$7.bridge  \\
    \trans `What broke the bridge?'
\end{exe}

As a side note, there seems to be a  preference to use the \textsc{progressive} marker {\itshape nzí} in past questions, even though the meaning is not necessarily progressive. Questions can also be formed without the \textsc{progressive} marker, as in (\ref{SCQ3}), but speakers would spontaneously form questions with this aspect marker while stating that questions without it are also grammatical and apparently mean the same. {\itshape nzí} therefore most likely also serves another function than \textsc{progressive}, but this needs further investigation.


\begin{exe}
\ex\label{SCQ3}
  \glll  gyí bvúɔ́ kàsà  \hfill S V O\\
            gyí bvúɔ̀-H kàsà \\
              what break-R $\emptyset$7.bridge\\
    \trans `What broke the bridge?'
\end{exe}

Other constituents besides objects have two positional options. Either, interrogatives for objects and adjuncts appear in-situ or are left dislocated to a phrase initial position. I will first demonstrate this with object questions.


For object questions, the same interrogative pronouns are used as for subject questions. In (\ref{OCQ1}), the object interrogative pronoun {\itshape nzá} `who' is left dislocated to the beginning of the phrase. As (\ref{OCQ1b}) shows, this also holds for negated questions. Both questions occur in O S V (X) word order.

\begin{exe}
\ex\label{OCQ1}
\begin{xlist}
\ex\label{OCQ1a}
  \glll    {\bfseries nzá} wɛ̀ nzí nyɛ̂ mɛ́nɔ́ yî mákítì \\
      nzá wɛ̀ nzí nyɛ̂ mɛ́nɔ́ yî mákítì \\
             who 2\textsc{sg} PROG.PST see $\emptyset$7.morning 7.DEM ma6.market   \\
    \trans `Who did you see this morning at the market?'
\ex \label{OCQ1b}
  \glll     {\bfseries nzá} wɛ̀ɛ́ kwálɛ̀lɛ̀ \\ 
  nzá wɛ̀ɛ́ kwàlɛ-lɛ \\
               who 2\textsc{sg}.PRES.NEG like-NEG \\ 
    \trans `Who don't you like?'
\end{xlist}
\end{exe}

\noindent Likewise, the inanimate interrogative pronoun {\itshape gyí} `what' can be left dislocated in object questions, as shown in (\ref{OCQ2}). Again, this also holds for negated questions, as in (\ref{OCQ2b}). 

\begin{exe}
\ex\label{OCQ2}
\begin{xlist}
\ex\label{OCQ2a}
  \glll    {\bfseries gyí} bwáà nzí nyɛ̂ tísɔ̀nì \hfill O S V X \\
          gyí bwáà nzí nyɛ̂ tísɔ̀nì \\
             what 2\textsc{pl} PROG see $\emptyset$7.town   \\
    \trans `What did you (Pl.) see in town?'
\ex \label{OCQ2b}
  \glll  {\bfseries gyí} wɛ̀ɛ́ kwálɛ́lɛ́ tísɔ̀nì dé tù \hfill O S V X \\
          gyí wɛ̀ɛ́ kwàlɛ-lɛ tísɔ̀nì dé \\ 
              what 2\textsc{sg}.PRES.NEG like-NEG $\emptyset$7.town LOC inside\\
    \trans `What don't you like in town?'
\ex \label{OCQ2c}
 \glll    {\bfseries gyí} Àdà lã́ã́ pá'á wà sã̂  \hfill O S V X \\
           gyí Àdà lã́ã̀-H pá'á wà sã̂ \\
             what $\emptyset$1.PN read-R $\emptyset$1.side 1:ATT $\emptyset$1.father \\ 
    \trans `What does Ada read for father?'
\end{xlist}
\end{exe}

The object interrogative pronoun can also occur in-situ, as shown in (\ref{OCQ3}) for both {\itshape nzá} `who' and {\itshape gyí} `what'.  In terms of its pragmatics, the in-situ position differs from left dislocatation in terms of information structure. The object position in-situ is the focus position, and thus the object interrogative appears in focus in (\ref{OCQ3}).

\begin{exe}
\ex\label{OCQ3}
\begin{xlist}
\ex \label{OCQ3a}
  \glll      wɛ̀ɛ́ kwálɛ́lɛ́ {\bfseries nzá} \hfill S V O \\
              wɛ̀ɛ́ kwálɛ́-lɛ́ nzá \\
                2\textsc{sg}.PRES.NEG like-NEG who \\
    \trans `WHO don't you like?'
\ex\label{OCQ3b}
 \glll     Àdà lã́ã́ {\bfseries gyí} pá'á wà sã̂  \hfill S V O X \\
            Àdà lã́ã̀-H gyí pá'á wà sã̂ \\
              $\emptyset$1.PN read-R what $\emptyset$1.side 1:ATT $\emptyset$1.father \\
    \trans `WHAT does Ada read for father?'
\end {xlist}
\end {exe}



In questions with double objects, the object interrogative can occur in three positions. In (\ref{QIO}), the question asks for the recipient object (which is often referred to as the direct object, but, as explained in \sectref{sec:HLinker}, direct and indirect objects cannot be distinguished on formal grounds in Gyeli). The object interrogative can appear either in i) left dislocation at the beginning of the phrase, as in (\ref{QIO1}), ii) in the first object slot, as in (\ref{QIO2}), and iii) in the second object slot, as in (\ref{QIO3}).

\begin{exe}
\ex\label{QIO} 
\begin{xlist}
\ex \label{QIO1}
  \glll  {\bfseries nzá} á vɛ́ béfùmbí \hfill O\textsubscript{1} S V O\textsubscript{2}  \\
         nzá a-H vɛ̂-H H-be-fùmbí \\
         who 1-PRES give-R OBJ.LINK-be8-orange  \\
    \trans `Whom does s/he give the oranges?'
\ex\label{QIO2}
  \glll á vɛ́ {\bfseries nzá} bèfùmbí  \hfill S V O\textsubscript{1} O\textsubscript{2} \\
       a-H vɛ̂-H nzá be-fùmbí \\
           1-PRES give-R who be8-orange  \\
    \trans `Whom does s/he give the oranges?'
\ex\label{QIO3}
  \glll  á vɛ́ béfùmbí {\bfseries nzá} \hfill S V O\textsubscript{1} O\textsubscript{2} \\
          a-H vɛ̂-H H-be-fùmbí nzá \\
           1-PRES give-R be8-orange who  \\
    \trans `WHOM does s/he give the oranges?'
\end{xlist}
\end{exe}

\noindent The same holds for {\itshape gyí} when asking for the patient object, as illustrated for all three possible positions in (\ref{QDO}).

\begin{exe}
\ex\label{QDO} 
\begin{xlist}
\ex \label{QDO1}
  \glll  {\bfseries gyí} wɛ́ gyíkɛ́sɛ́ bwánɔ̀ \hfill O\textsubscript{1} S V O\textsubscript{2}  \\
         gyí wɛ-H gyíkɛsɛ-H b-wánɔ̀ \\
         what 2\textsc{sg}-PRES teach-R ba2-child  \\
    \trans `What do you teach the children?'
\ex\label{QDO2}
  \glll  wɛ́ gyíkɛ́sɛ́ {\bfseries gyí} bwánɔ̀\hfill S V O\textsubscript{1} O\textsubscript{2} \\
         wɛ-H gyíkɛsɛ-H gyí b-wánɔ̀ \\
            2\textsc{sg}-PRES teach-R what ba2-child  \\
    \trans `What do you teach the children?'
\ex\label{QDO3}
  \glll  wɛ́ gyíkɛ́sɛ́ bwánɔ̀ {\bfseries gyí} \hfill S V O\textsubscript{1} O\textsubscript{2} \\
           wɛ-H gyíkɛsɛ-H  b-wánɔ̀ gyí \\
           2\textsc{sg}-PRES teach-R ba2-child what \\
    \trans `WHAT do you teach the children?'
\end{xlist}
\end{exe}



Just like object questions, also adjunct questions can occur both phrase-initially or in-situ. I demonstrate this for various adjunct questions.
In (\ref{AQ1}), for instance, the constituent that is asked for is a comitative oblique encoding accompaniment. This is expressed by a comitative marker plus an interrogative pronoun in the question. The oblique question can occur both phrase-initially and in-situ.

\begin{exe}
\ex\label{AQ1}
\begin{xlist}
\ex\label{AQ1a}
  \glll   {\bfseries nà} {\bfseries nzá} wɛ̀ɛ̀ kɛ̀ pɛ̂ \hfill X S V \\
          nà nzá wɛ̀ɛ̀ kɛ̀ pɛ̂ \\
             COM who 2\textsc{sg}.FUT go over.there  \\
    \trans `With whom will you go there?'
\ex\label{AQ1b}
  \glll wɛ̀ɛ̀ kɛ̀ pɛ̂ {\bfseries nà} {\bfseries nzá} \hfill S V X \\
           wɛ̀ɛ̀ kɛ̀ pɛ̂ nà nzá \\
           2\textsc{sg}.FUT go over.there COM who \\
       \trans `WITH WHOM will you go there?'
%\ex \label{nanza3}
 % \gll  nà bà nzá báà kɛ̀ pɛ̂? \\
  %        COM 2 who 2.FUT go there \\
  %  \trans `Who (Pl.) will go there?'
\end{xlist}
\end{exe}


\noindent The same pattern holds for oblique questions comprised of an associative plural construction, as in (\ref{AQ2}).

\begin{exe}
\ex\label{AQ2}
\begin{xlist}
\ex \label{AQ2a}
  \glll  {\bfseries bà} {\bfseries nà} {\bfseries nzá} báà kɛ̀ pɛ̂ \hfill X S V \\
         bà nà nzá báà kɛ̀ pɛ̂? \\
           AP COM who 2.FUT go over.there \\
    \trans `They and who will go there?'
\ex \label{AQ2b}
  \glll  báà kɛ̀ pɛ̂ {\bfseries bà} {\bfseries nà} {\bfseries nzá} \hfill S V X \\
         báà kɛ̀ pɛ̂ bà nà nzá̂ \\
           2.FUT go over.there AP COM who \\
    \trans `They and who will go there?'
\end{xlist}
\end{exe}



Some verbs with reciprocal meaning require the comitative marker {\itshape nà}. They behave peculiarly in question formation in that they both require an interrogative pronoun in left dislocation and a comitative oblique noun phrase at the end of the question. The object is taken up again in the oblique phrase by a pronominal resumptive. This is shown in (\ref{QCOM}).

\begin{exe}
\ex\label{QCOM}
\begin{xlist}
\ex \label{QCOM1}
  \glll      {\bfseries nzá} yáà lã́ {\bfseries nà} {\bfseries nyɛ̂}  \\
              nzá yáà lã̂-H nà nyɛ̂ \\
                who 1\textsc{pl}.PST2 talk-R COM 1.NSBJ \\
    \trans `Who did we talk to?'
\ex\label{QCOM2}
 \glll    {\bfseries nzá} wɛ̀ nzí làdtɔ̀ {\bfseries nà} {\bfseries nyɛ̂} tísɔ̀nì \\
            nzá wɛ nzî-H làdtɔ̀ nà nyɛ̂ tísɔ̀nì \\
              who 2\textsc{sg} PROG-PST meet COM 1.NSBJ  $\emptyset$7.town \\
    \trans `Who did you meet in town?'
\end{xlist}
\end{exe}



%\noindent - other contexts where the comitative is required are instrumental/manner questions including 'with' \\
%- here, the comitative usually occurs phrase initially \\
%- two different constructions: 1) [PRO COM who] 'they and who', 2) [COM 2-who] 'with who (Pl)'

%\begin{exe}
%\ex\label{with}
%\begin{xlist}
%\ex \label{with1}
%  \glll      {\bfseries bá} {\bfseries nà} {\bfseries nzá} bâ kɛ̀ pɛ̂?  \\
%              bá nà nzá bâ kɛ-H pɛ̂ \\
%               2-PRES COM who 2.FUT go there \\
%    \trans `Who will they go there with?'
%\ex\label{with2}
 %\glll    {\bfseries nà} {\bfseries bà-nzá} bwánɔ̀ bá nzí dúnà ɛ́ wû?   \\
    %        nà ba-nzá b-wánɔ̀ ba-H nzî-H dúnà ɛ́ wû  \\
       %       COM 2-who ba2-child 2-PRES PROG-R quarrel LOC there \\
    %\trans `Who do the children argue with there?'
%\end {xlist}
%\end {exe}

Other examples of adjunct questions concern locative questions. Again, as shown in (\ref{QLOC}), the locative oblique phrase can occur phrase-initially or in-situ, even though the left dislocated variant seems to be much more frequent, given its relatively unmarked status.

\begin{exe}
\ex\label{QLOC} 
\begin{xlist} 
\ex\label{QLOCa} 
  \glll      {\bfseries ɛ́} {\bfseries vɛ́} wɛ́ɛ̀ lúmɛ̀lɛ̀ bwánɔ̀ sùkúlì  \hfill X1 S V O X2 \\
              ɛ́ vɛ́ wɛ́ɛ̀ lúmɛlɛ b-wánɔ̀ sùkúlì \\
               LOC where 2\textsc{sg}.FUT send ba2-child $\emptyset$7.school \\ 
    \trans `Where will you send the children to school?'
\ex\label{QLOCb} 
  \glll      wɛ́ɛ̀ lúmɛ̀lɛ̀ bwánɔ̀ sùkúlì {\bfseries ɛ́} {\bfseries vɛ́}  \hfill X1 S V O X2 \\
               wɛ́ɛ̀ lúmɛlɛ b-wánɔ̀ sùkúlì ɛ́ vɛ́\\
                2\textsc{sg}.FUT send ba2-child $\emptyset$7.school LOC where \\
    \trans `WHERE will you send the children to school?'
\end{xlist}
\end {exe}

Temporal questions are also formed with oblique noun phrases. Depending on the expected time specificity, speakers usually use {\itshape dúbɔ̀ lé vɛ́} `what day', as in (\ref{with1}), or {\itshape wùlà yá vɛ́} `what time', as in (\ref{with2}). Again, both examples can occur phrase-initially and in-situ with the in-situ position being the more marked one.

\begin{exe}
\ex\label{with}
\begin{xlist}
\ex \label{with1}
  \glll      {\bfseries dúbɔ̀} {\bfseries lé} {\bfseries vɛ́} à nzí pámò \hfill X S V\\
              d-úbɔ̀ lé vɛ́ a nzî-H pámò \\
               le5-day 5:ATT which 1 PROG-R arrive \\
    \trans `When did she arrive [= what day]?'
\ex\label{with2}
 \glll    à nzí pámò {\bfseries wùlà} {\bfseries yá} {\bfseries vɛ́}  \hfill S V X \\
             a nzî-H pámò  wùlà yá vɛ́ \\
             1 PROG-R arrive  $\emptyset$7.hour 7:ATT which  \\
    \trans `WHEN did she arrive [= what time]?'
\end {xlist}
\end {exe}

\noindent Finally, also purpose obliques including {\itshape púù yá gyí} `what reason' are expressed following the same structure, as (\ref{QPur}) shows.

\begin{exe}
\ex\label{QPur}
\begin{xlist}
\ex \label{QPura}
  \glll     {\bfseries púù} {\bfseries yá} {\bfseries gyí} bá gyíbɔ́ nyɛ̂ \hfill  X S V\\
              púù yá gyí ba-H gyíbɔ-H nyɛ̂ \\
               $\emptyset$7.reason 7:ATT what 2-PRES call-R 1.NSBJ \\
    \trans `Why do they call him?'
\ex\label{QPurb}
  \glll       bá gyíbɔ́ nyɛ̂ {\bfseries púù} {\bfseries yá} {\bfseries gyí} \hfill S V X\\
               ba-H gyíbɔ-H nyɛ̂ púù yá gyí\\
                2-PRES call-R 1.NSBJ $\emptyset$7.reason 7:ATT what\\
    \trans `WHY do they call him?'
\end {xlist}
\end {exe}





\subsection{Possessor raising}
\label{sec:PossR}

Possessor raising is a pervasive phenomenon in Gyeli. While I use the term possessor raising in line with the literature on this topic, I do not imply an analysis of raising in the syntactic tree, but rather a marked possession construction.  Thus, the possessor can be expressed as the subject or object of a clause, avoiding adnominal possession marking and benefactive obliques.  In (\ref{PR1}), the possessor is expressed in the subject.


\begin{exe} 
\ex\label{PR1}
  \glll  {\bfseries mɛ́} dvúɔ́ nkû \\
          mɛ-H dvúɔ̀-H nkû \\
           1\textsc{sg}-PRES hurt-R $\emptyset$3.foot  \\
    \trans `My foot hurts.'
\end{exe}

In most cases, however, the possessor has object status. In (\ref{PR2}), for instance, the possessor {\itshape mɛ̂} takes the object position while {\itshape mbɔ̀} `arm' occurs as a bare locative oblique noun phrase.

\begin{exe} 
\ex\label{PR2}
  \glll    ká yí nyí {\bfseries mɛ̂} mbɔ̀ mpángì yí kùgá nâ nyíì {\bfseries wɛ̀} mbɔ̀ \\
           ká yi-H nyî-H mɛ̂ m-bɔ̀ mpángì yi-H kùga-H nâ nyíì wɛ m-bɔ̀ \\
             when 7-PRES enter-R 1\textsc{sg}.NSBJ N3-arm $\emptyset$7.bamboo 7-PRES can-R COMP enter.SBJV 2\textsc{sg} N3-arm  \\
    \trans `When it goes into my arm... the bamboo can sting your arm.'
\end{exe}

A possessor can also occur in copula constructions, as shown in (\ref{PR3}). Here, the possessor appears in the copula complement.

\begin{exe} 
\ex\label{PR3}
  \glll  nzà nyíì {\bfseries mɛ̀} mɔ̂ \\
          nzà nyíì mɛ̀ mɔ̂ \\
           $\emptyset$9.hunger 9.COP 1\textsc{sg}.NSBJ $\emptyset$3.stomach  \\
    \trans `I am hungry (lit.: hunger is me in the stomach).'
\end{exe}

While the previous examples could also have been expressed by possessive pronouns as modifiers to the noun, other possessor raising constructions are rather equivalent to benefactives. In (\ref{PR4}), for example, the structure could be modified to `build houses for me' with a purpose or benefactive oblique phrase introduced by {\itshape púù yá} (see \sectref{sec:OBL}). 

\begin{exe} 
\ex\label{PR4} 
  \glll mɛ̀ bùdɛ́ nâ á lwɔ́ngɔ́ {\bfseries mɛ̂} màndáwɔ̀\\
        mɛ bùdɛ-H nâ a-H lwɔ́ngɔ-H mɛ̂ ma-ndáwɔ̀ \\
       1\textsc{sg} have-R COMP 1-PRES build[Kwasio]-R 1\textsc{sg}.NSBJ ma6-house   \\
    \trans `I say that she [Nadine] builds me houses.'
\end{exe}

\noindent The same benefactive reading holds for copula constructions, as in (\ref{PR5}).

\begin{exe} 
\ex\label{PR5}
  \glll     nlã̂ wá zì ndáwɔ̀ nyà zì nyíì {\bfseries mɛ̂} {\bfseries vé} \\
          nlã̂ wá zì ndáwɔ̀ nyà zì nyíì mɛ̂ vé \\
              $\emptyset$3.story 3:ATT $\emptyset$7.tin $\emptyset$9.house 9:ATT tin 9.COP 1\textsc{sg}.NSBJ where \\
    \trans `The problem with the tin, where is the tin (roofed) house for me?'
\end{exe}

As a counterpart to benefactive readings, possessor raising can also express adverse functions, as in (\ref{PR6}) where the speaker experiences a bad event. The construction is further special in terms of information structure since the possessor object pronoun is fronted before the verb so that the verb appears in focus position (see \sectref{sec:OBJfront}). This shows that possessor objects indeed behave identical to other objects.

\begin{exe} 
\ex\label{PR6}
  \glll bùdì bà sílɛ̃́ɛ̃̀ {\bfseries mɛ̂} wɛ̀ ndáwɔ̀ tù vâ \\
        b-ùdì ba sílɛ̃́ɛ̃̀ mɛ̂ wɛ̀ ndáwɔ̀ tù vâ \\
       ba2-person 2.PST1 finish.COMPL 1\textsc{sg}.NSBJ die $\emptyset$9.house inside here  \\
    \trans `The people have all died here inside the house.'
\end{exe}







\subsection{Comparison constructions}
\label{sec:CC}


Comparison and superlative constructions in Gyeli, just as in many other Bantu and generally African languages, as observed, for instance, by \citet[157]{stassen84} are expressed verbally with the verb {\itshape bálɛ} `surpass'. This holds for the comparison of the quality of two entities, as in (\ref{CC1}). In this example, the compared quality is {\itshape mpà} `good', an adjective, followed by the infinitival form of {\itshape bálɛ} `surpass'. The slot of the adjective can also be filled with nouns denoting quality, size, or color, for instance with {\itshape nkpámá} `new (cl. 3/4)' or {\itshape mpùlɛ́} `yellow (cl. 3/4)'. Morphosyntactically, there is no difference in the use of such a noun or an adjective as a comparison parameter.

\begin{exe} 
\ex\label{CC1}
  \glll kàbà yíì mpà bálɛ̀ sɔ́tì \\
        kàbà yíì mpà bálɛ sɔ́tì \\
       $\emptyset$7.dress 7.COP good surpass $\emptyset$1.trousers  \\
    \trans `The dress is better than the trousers.'
\end{exe}

The pattern is the same for adverbial comparison. In (\ref{CC2}), {\itshape mpà} serves as an adverb to {\itshape kɛ̀} `go, run'. Just as in the previous example, it is followed by the comparison verb.

\begin{exe} 
\ex\label{CC2}
  \glll Màmbì á kɛ́ mpà bálɛ̀ Àdà \\
        Màmbì a-H kɛ̀-H mpà bálɛ̀ Àdà \\
       $\emptyset$1.PN 1-PRES go-R good surpass $\emptyset$1.PN  \\
    \trans `Mambi runs better than Ada.'
\end{exe}

\noindent {\itshape bálɛ} is further used in comparison of quantities. Here, {\itshape bálɛ} follows the object noun phrase that the quantity refers to and directly precedes the entity that is subject to comparison, namely the person Màmbì.

\begin{exe} 
\ex\label{CC3}
  \glll Adà à tsìlɔ́ békáládɛ̀ bálɛ̀ Màmbì \\
        Adà a tsìlɔ-H H-be-káládɛ̀ bálɛ̀ Màmbì \\
       $\emptyset$1.PN 1.PST1 write-R OBJ.LINK-be8-letter surpass $\emptyset$1.PN  \\
    \trans `Ada wrote more letters than Mambi.'
\end{exe}

In (\ref{CC2}) and (\ref{CC3}), the comparison is between two subjects. {\itshape bálɛ} is also used to compare two objects while the subject is identical, as in (\ref{CC3a}).

\begin{exe} 
\ex\label{CC3a}
  \glll Àdà à dé mántúà {\bfseries bálɛ̀} mànjù \\
        Àdà a dè-H H-ma-ntúà bálɛ ma-njù \\
       $\emptyset$1.PN 1.PST1 eat-R OBJ.LINK-ma6-mango surpass ma6-banana \\
    \trans `Ada ate more mangoes than bananas.'
\end{exe}

{\itshape bálɛ} can also function as the only verb in a clause that is tonally inflected for tense and mood, as in (\ref{CC4}). Here, the comparison is between the second constituents of a noun + noun genitive construction while the first constituent of the second construction is elided.

\begin{exe} 
\ex\label{CC4}
  \glll lèdyṹũ̀ lé dẽ̂ bálɛ́ nàkùgúù \\
        le-dyṹũ̀ lé dẽ̂ bálɛ-H nàkùgúù  \\
       le5-heat 5:ATT today surpass-R yesterday  \\
    \trans `Today it's warmer than yesterday.'
\end{exe}


In (\ref{CC6}), a comparison construction is used to express semantically a superlative by comparing one person's driving style to that of everyone else.

\begin{exe} 
\ex\label{CC6}
  \glll Adà á dvùdɔ́ màtúà bálɛ̀ bɔ́gà \\
        Adà a-H dvùdɔ-H màtúà bálɛ bɔ́-gà \\
       $\emptyset$1.PN 1-PRES drive-R $\emptyset$1.car surpass 2-other  \\
    \trans `Ada drives the car faster than all [= the fastest].'
\end{exe}


\noindent In contrast, in (\ref{CC5}), a superlative is expressed without comparing  two entities. Instead, {\itshape bálɛ} follows an object noun phrase which is subject to the superlative interpretation while {\itshape kɛ̀ mpfúndɔ́} encodes in which way Ada's car is the best, namely in going fast.

\begin{exe} 
\ex\label{CC5}
  \glll Adà á dvùdɔ́ màtúà bálɛ̀ kɛ̀ mpfúndɔ́ \\
        Adà a-H dvùdɔ-H màtúà bálɛ kɛ̀ mpfúndɔ́ \\
       $\emptyset$1.PN 1-PRES drive-R $\emptyset$1.car surpass go $\emptyset$3.speed  \\
    \trans `Ada drives the fastest car.'
\end{exe}


Finally, some comparison construction types take additionally to {\itshape bálɛ} the adverb {\itshape mpù} `like'. This is the case in equatives, as shown in (\ref{CC7}).

\begin{exe} 
\ex\label{CC7}
  \glll mɛ̀ɛ́ {\bfseries bálɛ́lɛ́} bɛ̀ nà mɔ̀nɛ́ ɛ́ {\bfseries mpù} nàkùgúù \\
        mɛ̀ɛ́ bálɛ-lɛ bɛ̀ nà mɔ̀nɛ́ ɛ́ mpù nàkùgúù \\
       1\textsc{sg}.PRES.NEG surpass-NEG be COM $\emptyset$1.money LOC like yesterday \\
    \trans `I don't have as much money as yesterday.'
\end{exe}

\noindent Further, {\itshape mpù} is used in comparisons of non-identical objects, as in (\ref{CC8}).

\begin{exe} 
\ex\label{CC8}
  \glll Àdà à dé mántúà {\bfseries bálɛ̀} {\bfseries mpù} Màmbì à dé mánjù \\
        Àdà a dè-H H-ma-ntúà bálɛ mpù Màmbì a dè-H H-ma-njù \\
       $\emptyset$1.PN 1.PST1 eat-R OBJ.LINK-ma6-mango surpass like $\emptyset$1.PN 1.PST1 eat-R OBJ.LINK-ma6-banana \\
    \trans `Ada ate more mangoes than Mambi bananas.'
\end{exe}

Constructions involving the comparison of identical objects is done without {\itshape mpù}, but only with {\itshape bálɛ̀} `surpass', as in (\ref{CC9}).

\begin{exe} 
\ex\label{CC9}
  \glll Àdà à dé mántúà {\bfseries bálɛ̀} mànjù \\
        Àdà a dè-H H-ma-ntúà bálɛ ma-njù \\
       $\emptyset$1.PN 1.PST1 eat-R OBJ.LINK-ma6-mango surpass ma6-banana \\
    \trans `Ada ate more mangoes than bananas.'
\end{exe}

\noindent Having described major types and phenomena of simple clauses, I now turn to complex clauses in the next chapter. 







