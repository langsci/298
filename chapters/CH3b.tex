\chapter{Morphology}
\label{sec:WordForm}

This chapter covers two broad aspects of Gyeli morphology. In the first part, I outline the  forms and types of bound morphemes. These serve as ingredients to form words either through inflection,  derivation, or composition. I follow \posscitet{haspelmath2010} textbook definitions of these terms. Inflection is the morphological process of producing word forms of a lexeme. Inflectional morphemes in Gyeli express grammatical categories such as agreement, tense, mood, negation, and objecthood. As such, the lexeme remains in the same part of speech. Many of the inflectional morphemes are  syntactically required and thus appear obligatorily. Additionally, their attachment is fully productive and predictable. In contrast, derivational affixes create new lexemes that belong to the same word family by adding grammatical morphemes. A derived lexeme can belong either to the same or a different word class than its source lexeme. Derivational morphemes are syntactically optional. Also, it is lexically specified which lexeme can take which derivation affix. As such, attachment of derivational affixes is less predictable. Finally, composition is a type of word formation that combines lexemes from different word families. In Gyeli, compounds typically include two lexical morphemes.    Inflection is discussed along with the morpheme types in part one of this chapter. Derivation and composition processes are discussed in the second part.

\section{Morpheme types}
\label{sec:MorphType}

In this section, I give an overview of the types of affixation morphemes found in the Gyeli language. I limit the discussion to overt non-root morphemes. That is, all morphemes discussed in this section are overt,\footnote{I do not consider null-forms here that are found in some nouns and agreement targets. To be consistent with noun class and agreement marking, however, I do represent them in glosses.}  bound,\footnote{The anaphoric marker {\itshape ndɛ̀} is an exception. It occurs as a bound morpheme with pronouns, but can also follow nouns as a free morpheme.} and grammatical.  Thus, lexical roots are not discussed here, but in \chapref{sec:POS}. Similarly for non-overt morphemes, such as portmanteau morphemes like, for instance, the subject-tense-aspect-mood-polarity (STAMP) marker or certain copulas: These portmanteau morphemes are free and occur as words in their own right, as also presented in \chapref{sec:POS}. 

I organize the presentation through the opposition between morphemes that precede the lexical root, prefixes, and those that follow the root, suffixes. Table \ref{Tab:AffixType} lists all prefix and suffix forms in the language.\footnote{Noun class and agreement prefixes often have alternate forms that are phonologically conditioned. In the table, I count a form and its alternate as only one form in order to not artificially increase the number of forms.} The table also presents the basic functional distinction between derivation and inflection morphemes. Note that the suffix -{\itshape lɛ} is both a derivational morpheme (when used as a verb expansion suffix) and an inflectional morpheme (when used as negation marker).  The table further specifies the more concrete functions associated with an affix, for instance, as a noun class prefix or a verb extension suffix,  and the part(s) of speech with which it occurs.\footnote{`various' in the parts of speech column under prefixes always refers to the set of limiting modifiers, possessor pronouns, and quantifiers, which I had to abbreviate for space reasons.}
 % \todo{discuss this table with me.  I think you should move the footnote about `various' to make it visible from looking at the table and should label le as ``deriv./infl.'', and maybe have the deriv and infl labels be vertical.}




\begin{table}[!h]
\centering
\small
\begin{tabular}{l|lll|lll}
 \midrule 
 					&   \multicolumn{3}{c|}{Prefixes} &  \multicolumn{3}{c}{Suffixes} \\
\cline{2-7}  
					& Forms & Function & POS         & Forms & Function & POS       \\ 
 \midrule
\multirow{3}{*}{Deriv.} & {\itshape na}- & SIM   	&  N, QUAL,	  &  -{\itshape ɛ̀dɛ̀} & NOM	   & N	\\
					& 		& 	&  ADV  		&  -{\itshape a} & NOM, EXT & N, V 	 \\
					& 		& 	& 			& -{\itshape ala} & EXT	& V 		\\
					& 		& 	& 			& -{\itshape ɛlɛ} & EXT	& V 		\\
					& 		& 	& 			& -{\itshape ɛga}/-{\itshape aga} & EXT & V \\
					& 		& 	& 			& -{\itshape ɛsɛ} & EXT	& V 		\\
					& 		& 	& 			& -{\itshape ɔwɔ} & EXT	& V 		\\
					& 		& 	& 			& -{\itshape bɔ}/-{\itshape wɔ} & EXP	& V 		\\
					& 		& 	& 			& -{\itshape kɛ}/{\itshape gɛ} & EXP	& V 		\\ 
\hdashline[0.6pt/4pt]
					& 		& 	& 			& -{\itshape lɛ} & EXP, NEG	& V 		\\
\hdashline[0.6pt/4pt]
\multirow{7}{*}{Infl.}          & {\itshape m}- & NC, AGR	& N, various 	&  (-){\itshape ndɛ̀} &  ANA	& DEM, free	\\
					& {\itshape n}- &  NC, AGR	& N, MOD 	&  -{\itshape gà} & CONTR	&  PRO 	\\
					& {\itshape ba}-/{\itshape b}- & NC & N 	        & -{\itshape o} &  VOC & 	N, ADV	\\
					& {\itshape mi}-		& NC & N 		&   -H & TM	&  STAMP, V 	\\
					& {\itshape le}-/{\itshape d}- & NC & N 	&  & 	&  \\
					& {\itshape ma}-/{\itshape m}-		& NC & N 		&  & 		&			\\
					& {\itshape be}-		& NC & N 			& &  		& 			\\ 
					& {\itshape w}-		& AGR & various	& &  		& 	 		\\ 
					& {\itshape bà}-/{\itshape b}-	& AGR & various 	& &  		& 	\\ 
					& {\itshape bá}-		& AGR & NUM 			& &  		& 			\\ 
					& {\itshape mì}-/{\itshape my}-		& AGR & various	& &  	&	\\ 
					& {\itshape mí}-		& AGR & NUM			& &  		& 		\\ 
					& {\itshape lè}-/{\itshape l}-		& AGR & various	& &  	& 	\\ 
					& {\itshape mà}-/{\itshape m}-		& AGR & various & &  	& 	\\ 
					& {\itshape má}-		& AGR & NUM 		& &  		& 	 		\\ 
					& {\itshape y}-		& AGR & various	& &  		& 		\\ 
					& {\itshape bì}-/{\itshape by}- & AGR & various & &  		& 	\\ 
					& {\itshape bé}-		& AGR & NUM 			& &  		& 			\\ 
					& {\itshape ny}-		& AGR & various	& &  		& 			\\ 
					& H-		& OBJ.LINK & N 			& &  		& 			\\ 
 \midrule 
Total					& 20 & 4			& 	7		& 14 &  8			& 	7	\\
 \midrule
\end{tabular}
\caption{Frequency of affix types by form and function}
\label{Tab:AffixType}
\end{table} 

The bottom line of Table \ref{Tab:AffixType} provides frequencies of forms, functions, and the parts of speech that affixes attach to. With 20 prefix and 14 suffix forms, Gyeli has a total of 34 affixes. Prefix forms are higher in number, constituting 59\% of the affixes.  Many prefix forms are segmentally identical, for example {\itshape mi}-, {\itshape mì}-, and {\itshape mí}-. They differ, however, in their tonal pattern and thus must  be formally distinguished. The affixes show a basic functional distribution:  Most prefixes are inflectional (19 out of 20), while the majority of suffixes are derivational (9 out of 14 suffixes are solely derivational and the suffix -{\itshape lɛ} is both derivational and inflectional).

Though more forms are prefixes than are suffixes, prefix forms map onto fewer functions, namely only 4, while suffixes have 8 different functions. Most prefixes are agreement and/or noun class prefixes.\footnote{The relation between noun class, agreement class, and grammatical number is discussed in \sectref{sec:Gender}.} In addition to these most frequent functions, there is also an object linking H tone and the similative marker {\itshape na}-. Suffixes display the inverse distribution: Relatively fewer forms map onto more functions. Most suffix forms are functionally derivational extension or expansion morphemes.\footnote{See \sectref{sec:EXtp} for the difference between extension and expansion suffixes.} Other derivational suffixes serve as nominalization morphemes. Inflectional suffixes cover the functions of negation, anaphoric, contrastive, and vocative marker, and a suffix H tone that marks various tense and mood categories. 

Cross-linguistically, it is not typical that   anaphoric, contrastive, or vocative suffixes appear as inflection morphemes. In Gyeli, they differ from the other inflectional affixes since they are not obligatory. In order to make this distinction, I  call them `markers'. I still consider them as inflection morphemes, however, for two reasons. First, unlike the derivation affixes, they do not form new lexemes, i.e., they do not have an entry in the lexicon. Second, their attachment is completely predictable, unlike for derivation affixes. For example, every non-subject pronoun can take the contrastive marker -{\itshape gà} (but not every verb can take a causative derivational suffix). 

Finally, Table \ref{Tab:AffixType} lists the parts of speech that affixes attach to. Prefixes and suffixes attach to a roughly equal number parts of speech,
but differ in the specific word classes to which they attach. Prefixes are restricted to the domain of the noun phrase,  attaching to nouns (noun class prefixes and the similative marker) and to agreement targets  such as limiting modifiers (MOD), possessor pronouns, quantifiers, and numerals\footnote{See \sectref{sec:NAdjuncts} for the criteria by which quantifiers  and numerals are formally distinguished.}. In contrast, suffixes feature a broader variety of word classes, encompassing both noun and verb phrases. Nominalization and vocative  suffixes attach to nouns. Extension, expansion, negation, and tense-mood suffixes attach to verbs. Also demonstratives, pronouns, adverbs, and the subject-tense-aspect-mood-polarity (STAMP) marker are word classes that suffixes attach to. In the following, I will briefly outline the various prefixes and suffixes grouped by function.

 

%\todo{in the following two sections it was not always clear to me exactly what information was to be discussed here versus later.  Some discussions are very detailed (nominalization), while others are postponed (verbal expansion), and I didn't see the underlying logic.  If this could be clarified, it would make it a lot stronger}



\subsection{Prefixes}
\label{sec:Prefix}

A noun stem can maximally take three prefixes, as illustrated in (\ref{Prefix-extent}).

\begin{exe}
\ex \label{Prefix-extent} \textsc{object linking H tone -- noun class -- similative -- stem}
\end{exe}

\noindent The prefix that is closest to the stem is the similative marker {\itshape ná}-. This can be preceded by a plural noun class prefix and an object linking H tone. 

Gyeli has four different functional types of prefixes: the derivational similative prefix {\itshape ná}-, and the inflectional noun class, agreement, and object linking H tone prefixes. I now discuss each briefly.



\subsubsection{The similative prefix {\itshape na}-} 
\label{sec:SIM}

The derivational similative marker {\itshape na}- forms a functional category on its own expressing the meaning of `like'. The prefix is related to the free morpheme {\itshape ná} which serves as a similative marker in noun + noun naming contructions, as discussed in \sectref{sec:SIMword}. Words with the prefix {\itshape na}- are derived from  either a (diachronic) verb or noun or are synchronically opaque.  The application of {\itshape na}- results in common nouns, proper nouns,  some adjectives, and temporal adverbs, as shown for each type in (\ref{SIMNouns}). In the derivation of common nouns, proper nouns, and adjectives, {\itshape na}- takes a H tone whereas, in the derivation of adverbs, it takes a L tone.


\begin{exe}
\ex\label{SIMNouns}
\begin{xlist}
\ex {\bfseries ná-gyàlɛ́} `breastfeeding woman [lit.\ like nursing period]' (common noun)
\ex {\bfseries Ná-nzɛ̂} (Nanzé) `female name [lit.\ like panther]' (proper noun)
\ex {\bfseries ná-vyû} `black [lit.\ like blackened]' (adjective)
\ex {\bfseries nà-mɛ́nɔ́} `tomorrow [lit.\ like morning]' (adverb)
\end{xlist}
\end{exe}

The {\itshape na}- similative marker is the most lexicalized prefix in the language since its use is not productive. Instead,  it is lexically specified which nouns, adjectives, and adverbs occur with this marker.  Especially in the case of nouns with the {\itshape na}- prefix, one could even argue that the prefix is synchronically frozen to the lexical stem since in many instances the meaning of the lexical stem is opaque.  There are several reasons, however, why I consider {\itshape na}- as a prefix and not as part of the lexical stem.  First, nouns with the {\itshape na}- prefix are structurally different from other common nouns. If one counted {\itshape na}- as part of the nominal stem, some of these stems would have a syllable length of four syllables. As discussed in \sectref{sec:SyllN}, however, the maximum syllable length in stems is three syllables (and even this is dispreferred, accounting for only 10\% of the nouns in the database).  Second,  the {\itshape na}- prefix occurs quite frequently and regularly, especially in the derivation of female names from male names and in adjectives. This suggests that there is a formal pattern (rather than just a random CV syllable shape).   Third,  there is a clear function attributed to {\itshape na}-, namely that of expressing similarity, as shown in the examples in (\ref{SIMNouns}).  Derivation with the prefix {\itshape na}- is discussed in greater detail in \sectref{sec:NOMSIM}.



\subsubsection{Noun class prefixes}
\label{sec:NCPre}

Noun class prefixes are inflectional bound morphemes that attach only to the part of speech of common nouns (but not proper nouns). There are 11 different overt forms which can be grouped into 6 underlying categories, based on phonological conditioning. The forms and their alternates are listed in Table \ref{Tab:NCmorph}.\footnote{The prefix {\itshape be}- does not have a listed alternate form because there is no known instance of a noun using this prefix and having a stem-initial vowel.} 

\begin{table}[!h]
\centering
\begin{tabular}{ll|l}
 \midrule 
NC form & Alternate form  & Phonological condition \\
 \midrule
m- & n- &  assimilation of place of articulation \\
ba- & b- & before stem-initial vowel \\ 
mi- & m- & before stem-initial vowel \\
le- & d- &  before stem-initial vowel \\
ma- & m- & before stem-initial vowel \\
be & & \\
 \midrule
\end{tabular}
\caption{Noun class prefix forms}
\label{Tab:NCmorph}
\end{table} 

\noindent  Noun class prefixes fill the second of three possible prefix slots in nouns, potentially preceded by the object linking H tone (see \sectref{sec:OBJTone}) and followed by a similative marker (see \sectref{sec:SIM}). 

It is an inherent property of each noun which noun class prefix(es) the noun can take. Some noun forms do not take any overt prefix at all. Since noun class prefixes are part of the gender and agreement system which operates on a morpho-syntactic rather than solely morphological level,  these prefixes  are discussed in greater detail in \sectref{sec:NC} where their forms are organized according to noun and agreement classes.\footnote{In the context of gender and agreement, I also view a null-form as a category, but since it is not overt, I do not list it as a morpheme in this section.}




\subsubsection{Agreement prefixes}
\label{sec:AGRPre}

Like noun class prefixes, agreement prefixes are inflectional bound morphemes. They attach to  different parts of speech, however, including a range of agreement targets. In contrast to nouns, agreement targets have only one prefix slot. Gyeli has 13 agreement prefix forms plus an additional 5 alternate forms due to phonological conditioning. All forms are listed in Table \ref{Tab:AGRmorph}, specifying which form attaches to which part(s) of speech. 

\begin{table}[!h]
\centering
\small
\begin{tabular}{ll|ll}
 \midrule 
AGR form & POS & Alternate form  & Phonological condition \\
 \midrule
m- & MOD, POSS, QUANT &   &   \\
n- &  MOD &     &     \\
w- & MOD, POSS, QUANT &     &     \\
bà- & MOD, POSS, QUANT &  b- & before stem-initial vowel \\ 
bá- & NUM &     &  					\\
mì- & MOD, POSS, QUANT & my- & before stem-initial vowel \\
mí  & NUM &        & 					\\
lè- & MOD, POSS, QUANT &  l- &  before stem-initial vowel \\
mà- & MOD, POSS, QUANT &  m- & before stem-initial vowel \\
má- &  NUM &     & 					\\
bì-  & MOD, POSS, QUANT &  by- & 	before stem-initial vowel	\\
bé- & NUM &  & 						\\
ny-  & MOD, POSS, QUANT &   & 					\\
 \midrule
\end{tabular}
\caption{Agreement prefix forms}
\label{Tab:AGRmorph}
\end{table} 

Most consonantal and all L tone CV- prefix forms attach to the same set of agreement targets, namely limiting modifiers, possessor pronouns, and quantifiers. The prefix {\itshape n}- is an exceptional form in that it only attaches to certain limiting modifiers. All H tone agreement prefixes attach to numerals.\footnote{Limiting modifiers, possessor pronouns, quantifiers, and numerals are not the only agreement targets in Gyeli, but they are the parts of speech that mark agreement by means of a prefix. Other agreement targets have free forms which are described as parts of speech in \chapref{sec:POS}; all agreement targets are listed in \sectref{sec:AGRtargets}.} Like noun class prefixes, agreement prefixes have a morpho-syntactic dimension within the gender marking system. \sectref{sec:AGR} provides information on how agreement prefix forms pattern into agreement classes.


\subsubsection{Object linking H tone}
\label{sec:OBJTone}


Some morphemes in Gyeli are not segmental, but solely tonal. This is the case for the H tone that attaches to the left of common nouns in certain contexts.\footnote{Proper nouns do not take an object linking H tone which is tied to the fact that proper nouns do generally not take noun class prefixes. Therefore, the object linking H tone does not have a toneless TBU to attach to.} In terms of its function, this H tone prefix marks a noun as the object that is closest to the verb. As such, it is an inflectional morpheme that has to appear obligatorily in the specific environment. I call this prefix `object linking H tone' and gloss it as `OBJ.LINK', as shown in (\ref{OBJ40}).

\begin{exe} 
\ex\label{OBJ40}
  \glll nkɛ̀ nyì nzí sílɛ̃́ɛ̃̀ {\bfseries bé}déwò. \\
          nkɛ̀ nyi nzí sílɛ̃́ɛ̃̀ {\bfseries H}-be-déwò \\
          $\emptyset$9.field 9 PROG.PST finish.COMPL {\bfseries OBJ.LINK}-be8-food   \\
    \trans `The field was already running out of food.'
\end{exe}

The object linking H tone only appears on otherwise toneless CV- shape noun class prefixes, but is not realized on null-form or consonantal noun class prefixes.\footnote{Object nouns with null-form and consonantal noun class prefixes are completely unchanged; no downstep phenomena could be observed.}
More examples of the object linking H tone and information on its function as marking grammatical relations is provided in \sectref{sec:HLinker}.







\subsection{Suffixes}
\label{sec:Suffix}

Gyeli  suffixes can be categorized into eight different functions: nominalization, extension and expansion, negation, anaphoric marker, contrastive marker, vocative marker, and a tense-mood marking H tone suffix. I will outline each of these types in the following, discussing extension and expansion suffixes in the same section since their function is the same (but the level of productivity differs).
Derivation suffixes are only briefly mentioned in this section, followed by more details in \sectref{sec:FormProcess} within the discussion of derivation processes, while inflectional suffixes are outlined in greater length here.


\subsubsection{Nominalization suffixes}
\label{sec:NOMSuff}

Gyeli has two nominalization suffixes, namely -{\itshape ɛ̀dɛ̀} and -{\itshape a} which occur with common nouns. The latter also serves as the passive extension suffix in verbal derivation. Nominalization suffixes do not occur in all derived nouns. In fact, a minority of nominalized nouns uses a nominalization suffix, and it is lexically specified which nouns take such a suffix. The distribution of -{\itshape ɛ̀dɛ̀} and -{\itshape a} seems to be complementary. -{\itshape ɛ̀dɛ̀} appears in agentive deverbal nouns of gender 1/2, but is not found in other genders and non-agentive nouns. In contrast, -{\itshape a} is used in non-agentive deverbal nouns in all other genders that deverbal nouns occur in. \sectref{sec:NOM} gives a more detailed account of nominalization processes.

The suffix -{\itshape a} is used more frequently and more generically than -{\itshape ɛ̀dɛ̀}. It occurs in two types of nominalizations. First, it is found in some deverbal nouns where the source verb either ends in /ɛ/ or /ɔ/, as shown in (\ref{NOMa}).\footnote{Nouns derived from verbs with other vowel endings such as /i/, /o/, and /a/ never undergo vowel change.} The resulting deverbal nouns are all clearly assigned to a gender, as discussed in \sectref{sec:NOM}, and behave morpho-syntactically just like other common nouns, as discussed in \sectref{sec:commonN}.

\begin{exe} \ex \label{NOMa}
\begin{xlist}
\ex tálɛ `begin' $\rightarrow$ ma-tál-{\bfseries á} `beginning'
\ex dígɛ `look' $\rightarrow$ ma-díg-{\bfseries á} `vision'
\ex dìlɛ `bury' $\rightarrow$ ma-dìl-{\bfseries á} `funeral'
\ex líbɛlɛ `show' $\rightarrow$ ma-líbɛ́l-{\bfseries á} `appearance, showing'
\ex tfúdɔ `pinch' $\rightarrow$ tfúd-{\bfseries á} `pinch (n.)'
\ex tsìlɔ `write' $\rightarrow$ n-tsìl-{\bfseries á} `hand writing'
\end{xlist}
\end{exe}

\noindent All derived nouns in (\ref{NOMa}) take a H tone on the nominalization suffix. Tonal changes in nominalization are discussed further in \sectref{sec:NOM}.

There are also deverbal nouns that keep the verb's final vowels /ɛ/ and /ɔ/ and do not take a final H tone, as shown in (\ref{NOMb}). Therefore, affixation of -{\itshape a} and a H tone assignment in nominalization do not seem to be phonological rules.

\begin{exe} \ex \label{NOMb}
\begin{xlist}
\ex bwàlɛ `be born' $\rightarrow$ ma-bwàlɛ̀ `birth'
\ex gyɛ̀'ɛlɛ `pray' $\rightarrow$ ma-gyɛ̀'ɛ̀lɛ̀ `prayer'
\ex dɔ̀ `negotiate' $\rightarrow$ ma-dɔ̀ `negotiation'
\ex tɛ̀mbɔwɔ `set (sun)' $\rightarrow$ ma-tɛ̀mbɔ́wɔ́ `sunset'
\end{xlist}
\end{exe}

The second usage of the nominalization suffix -{\itshape a} is a nominalized participle form, examples of which are given in (\ref{nomp}).  325 verbs in the verb database of 377 verbs allow the nominalized participle. As such, it is the most productive verbal derivation. In contrast to other deverbal nouns with -{\itshape a}, however, the nominalized participle is syntactically restricted to the predicate position in copula constructions, as discussed in \sectref{sec:COP}. Other differences from common nouns are described in \sectref{sec:NounPart}. The derivation process always includes a nasal prefix as well and a specific tonal pattern which are explained in \sectref{sec:NOMPart}.

\begin{exe} \ex \label{nomp}
\begin{xlist}
\ex tsíbɔ `grind' $\rightarrow$ n-tsíb-{\bfseries â} `ground (thing)' 
\ex tálɛ `begin' $\rightarrow$ n-tál-{\bfseries â} `begun (thing)'
\ex gyàga `buy' $\rightarrow$ n-gyàg-{\bfseries á} `bought (thing)'
\ex jì `open' $\rightarrow$ n-jìy-{\bfseries á} `opened (thing)'
\end{xlist}
\end{exe}

There are only two known instances of the -{\itshape ɛ̀dɛ̀} suffix. Both are used in deverbal nouns of gender 1/2 as agentive nouns. They are listed in (\ref{NOMede}).

\begin{exe} \ex \label{NOMede}
\begin{xlist}
\ex gyámbɔ `cook' $\rightarrow$ n-gyámb-{\bfseries ɛ̀dɛ̀} `cook (n.)'
\ex gyímbɔ `dance' $\rightarrow$ n-gyímb-{\bfseries ɛ̀dɛ̀} `dancer'
\end{xlist}
\end{exe}

\noindent The -{\itshape ɛ̀dɛ̀} suffix might be a more marked form for agentive nouns in order to disambiguate between other nominalized forms. As explained in \sectref{sec:NOM}, agentive nouns are generally formed by combining the verb stem with a noun class marker. This also works, for instance, with the verb {\itshape gyímbɔ} `dance'  and the derived noun {\itshape n-gyímbɔ̀} `sorcerer'. In this instance, -{\itshape ɛ̀dɛ̀} might be used to distinguish   {\itshape n-gyímbɔ̀} `sorcerer' from {\itshape n-gyímb-ɛ̀dɛ̀} `dancer'.



\subsubsection{Extension and expansion suffixes}
\label{sec:EXtp}

Extension and expansion suffixes are derivational suffixes which derive verbs from other verbs, changing their valency. The difference between extension and expansion pertains to the suffix's level of productivity. Extension morphemes are synchronically productive, while expansion morphemes are not. Gyeli has six extension and three expansion morphemes, as listed in Table \ref{Tab:AffixType}. Each of them is discussed in detail in \sectref{sec:VDeriv}. 

\subsubsection{Negation suffix -{\itshape lɛ}}
\label{sec:NEGSuff}

The suffix -{\itshape lɛ} has two functions and for both attaches to verbs. First, it serves as a derivational expansion suffix, as discussed in \sectref{sec:DiaEx}. Second, -{\itshape lɛ} also occurs as an inflectional suffix marking negation. As a negation suffix, -{\itshape lɛ} productively attaches to all verb stems in the present tense, as exemplified in (\ref{negle}). Tonal changes depend on the verb's stem tones and are discussed in detail in \sectref{sec:NEGPRES}.

\begin{exe} \ex \label{negle}
\begin{xlist}
\ex gyámbɔ `cook' $\rightarrow$ gyámbɔ́-{\bfseries lɛ́} `not cook'
\ex kòla `add' $\rightarrow$ kólà-{\bfseries lɛ̀} `not add'
\end{xlist}
\end{exe}

\noindent In other tenses, auxiliary negation verbs are used that contain the suffix -{\itshape lɛ}, but whose lexical meaning is synchronically unknown. 




\subsubsection{Anaphoric marker -{\itshape ndɛ̀}}
\label{sec:ANASuff}

The anaphoric marker {\itshape ndɛ̀} signals reference to an entity that has been mentioned before in the discourse. It occurs both as a  bound and free morpheme in different contexts, in which speakers consistently judge the construction containing   {\itshape ndɛ̀}  as consisting of one or two words, respectively.   When following a noun or an identificational marker, {\itshape ndɛ̀} is a free morpheme. This is further discussed in \chapref{sec:POS} and \sectref{sec:ANAfree}. If {\itshape ndɛ̀} follows a demonstrative, it is analyzed as a bound suffix to the demonstrative, as illustrated in (\ref{41t}).

\begin{exe} 
\ex\label{41t}
  \glll bèdéwò {\bfseries bíndɛ̀} byɔ̀ mɛ́ lɔ́ njì lɛ́bɛ̀lɛ̀ bédéwò bà wɛ̀.\\
        be-déwò bí-ndɛ̀ byɔ̀ mɛ-H lɔ́ njì lɛ́bɛlɛ H-be-déwò bà wɛ̀ \\
           be8-food 8.DEM-ANA 8.EMPH 1-PRES RETRO come  follow be8-food AP 2\textsc{sg}.NSBJ  \\
    \trans `That (aforementioned) food, I have come to look for the food at your place.'
\end{exe} 

The demonstrative form that {\itshape ndɛ̀} attaches to differs formally from the proximal and distal demonstrative paradigms described in \sectref{sec:DEM}. It takes the segmental CV shape of the proximal paradigm with a plain vowel, opposed to long vowels of the distal paradigm. The tonal pattern differs, however, since the form that {\itshape ndɛ̀} attaches to has a H tone rather than a falling tone as in the proximal paradigm. Given that many agreement encoding morphemes have a similar shape within their agreement class, one might wonder if the element preceding {\itshape ndɛ̀} is really a demonstrative. Possible other candidates could be an agreement prefix or attributive marker. I will rule out both of these possibilities  and then explain why I consider the morpheme a demonstrative, despite formal differences.

Agreement prefixes with a plain vowel and a H tone are found with numerals and genitive markers, as shown in \sectref{sec:AGR}. Given the formal identity between these agreement prefixes and the element that precedes {\itshape ndɛ̀}, one might be tempted to analyze {\itshape ndɛ̀} as a stem rather than a suffix and the CV element as its agreement prefix.  This explanation has to be ruled out, however, since the occurrence of agreement prefixes is obligatory, but there are instances where {\itshape ndɛ̀} occurs without the CV element, namely when analzyed as a free form.

Another possibility would be to analyze the CV element as an attributive marker. As shown in \sectref{sec:ATT}, many of the attributive markers across different agreement classes have  a CV shape with a plain vowel and a H tone. Most attributive markers link a noun to a second constituent that could be another noun or another part of speech, such as adjectives or interrogative pronoun, as discussed in \sectref{sec:CONC}. Thus, this analyzsis would also make sense syntactically. Arguments against this explanation, however, concern the form of some attributive markers and their distribution. First, the attributive marker forms of agreement classes 1, 3, 7, and 9 differ from the CV shape element found with {\itshape ndɛ̀}. For instance, in agreement class 1, the attributive marker is {\itshape wà}, while {\itshape ndɛ̀} would be preceded by {\itshape nú}-; in agreement class 7, the attributive marker is {\itshape yá}, but {\itshape ndɛ̀} is preceded by {\itshape yí}-. Second, there are examples where {\itshape ndɛ̀} plus its preceding CV morpheme follow a true attributive, as shown in (\ref{ANAnoATT}). This makes it very clear that the morpheme cannot be an attributive marker.

\begin{exe} 
\ex\label{ANAnoATT} 
  \glll mùdì {\bfseries wà} {\bfseries nú}ndɛ́ dígɛ́ mísì. \\
       m-ùdì wà nú-ndɛ̀ dígɛ-H m-ísì \\
        N1-person 1:ATT 1.DEM-ANA look-R ma6-eye \\
    \trans `That (aforementioned) person looks with his eyes [= thinks very hard].'
\end{exe}

With both these other two options ruled out, I decide to analyze the preceding CV shape element as a demonstrative. Despite its formal mismatch to the independent proximal and distal demonstrative forms in terms of tone pattern or vowel length, there are reasons that support this categorization. First, all CV elements preceding {\itshape ndɛ̀} are segmentally identical to the proximal demonstrative paradigm. Second,   demonstratives and the anaphoric marker are functionally and semantically related. They both serve to pick out referents from a set of entities. The anaphoric marker can be understood as a specification of general demonstratives in that it points the addressee to a referent that is not spatially distant, but that has come up in the discourse before. This specification seems, however, optional since both demonstratives (in anaphoric contexts) and anaphoric markers can appear independently of each other.



\subsubsection{Contrastive marker -{\itshape gà}}
\label{sec:CONTRS}

The morpheme -{\itshape gà} is an inflectional suffix that attaches to non-subject pronouns, as shown in (\ref{122t}).

\begin{exe} 
\ex\label{122t} 
  \glll wɛ́ kɛ́ nà nyɛ̂ nkɔ̃̀wáká, {\bfseries nyɛ̀gà} à nzíí wɛ̂ vã́ã̀kɛ́ sâ mpù. \\
         wɛ-H kɛ̀-H nà nyɛ̂ nkɔ̃̀wáká nyɛ̀-gà a nzíí wɛ̂ vã́ã̀kɛ́ sâ mpù \\
         2\textsc{sg}-PRES go COM 1 equal.sharing 1-CONTR 1 PROG.PRES 2\textsc{sg}.NSBJ go[Bulu] do like.this  \\
    \trans `You go with him equally sharing, he is going to do you like this [= tries to trick you].'
\end{exe}

\noindent -{\itshape gà} serves to track referents and, in terms of information structure, indicates a switch of topics, as explained in \sectref{sec:insituTop}.
The suffix appears to be derived from the limiting modifier -{\itshape ɔ́(nɛ́)gá} `other', as discussed in \sectref{sec:other}. 

\subsubsection{Vocative marker -{\itshape o}}
\label{sec:VOCSuff}

All proper nouns can take the vocative suffix -{\itshape o}, for instance as in {\itshape Mìnsêm-o} or {\itshape Màmá-o}. The suffix attaches to the noun without undergoing assimilation; thus a final vowel of the noun stem does not delete. The tone of the suffix depends on speaker proximity. If the addressee is close to the speaker, the suffix has a L tone, if the addressee is further away, it has a H tone. 
The vocative suffix is not exclusively restricted to proper names, but can also be used with common nouns. These occurrences are, however,  limited to common nouns expressing a relation that can be used as address, such as {\itshape nyá-ò} `mother' and {\itshape tá-ò} `father'.  The vocative can also attach to the locative adverb {\itshape wɛ̂} `there', as shown in (\ref{VOC}), where it also combines with the distal H tone.

\begin{exe} 
\ex\label{VOC} 
  \glll mùdì kí tàtɔ̀ wú{\bfseries ó}! \\
       m-ùdì kí tàtɔ wû-o-H \\
        N1-person NEG scream there-VOC-DIST \\
    \trans `Nobody scream over there!'
\end{exe}

\subsubsection{Tense-mood H tone suffix}
\label{sec:TMHSuff}


A H tone suffix attaches to the subject-tense-aspect-mood-polarity (STAMP) marker and/or verbs in certain tense-mood categories. The STAMP marker takes the H tone suffix to mark \textsc{present} and \textsc{subjunctive}, while verbs take the the H tone suffix to encode \textsc{recent past} and \textsc{remote past}.   
These processes are described in detail in \sectref{sec:GramTM}.






\section{Derivation and compounding}
\label{sec:FormProcess}


Having discussed the different morpheme types and their distribution, I now turn to describing the language's word formation processes. This includes nominalization, verbal derivation, and compounding.



\subsection{Nominalization}
\label{sec:NOM}


Nominalization is a word formation process in which nouns are formed from lexemes of other word classes. In Gyeli, the source word class for nominalization is generally restricted to verbs, at least for the derivation processes that are synchronically transparent.\footnote{In nominalizations with the similative marker {\itshape ná}-, the derivation process is rather opaque so that the derivational source is synchronically not recognizable, as discussed in \sectref{sec:NOMSIM}.}

Formally, there are several means to mark nominalization on a derived noun, namely:
\begin{enumerate}[label=(\roman*)]
\itshapeem prefixation of a noun class prefix or similative marker {\itshape ná}-
\itshapeem suffixation of -{\itshape a} or -{\itshape ɛ̀dɛ̀}
\itshapeem final tone change
\end{enumerate}

\noindent Based on how these means are systematically used in combination, three different types of nominalized forms can be achieved.  First, there are those which are full nouns and assigned to a gender. Their prefixation pattern is based on assigned gender, and they have occasional suffixation of -{\itshape a} or -{\itshape ɛ̀dɛ̀} as well as  occasional tone change to final H.  Second, there are defective nouns, which are nominalized participles.  These always manifest prefixation of N- and  suffixation of -{\itshape a}, and  always have a final tone change to H or HL.  Third, there are those  derived with {\itshape ná}, producing nouns and adjectives. These always manifest prefixation of {\itshape ná}-, but never suffixation  nor tonal changes.


\noindent Full deverbal nouns, found in genders 1/2, 3/4, 5/6, 6, and 7/8, and the nominalized participle use maximally all three formal means in the nominalization process, while derivation with {\itshape na}- is never subject to word final change, i.e.\ suffixation and tone change. Full nouns differ from the nominalized participle in the systematicity with which suffixation applies and the kind of final tone change that takes place, as outlined below. 

What all three nominalization types have in common is that they take some sort of prefix. Full deverbal nouns are assigned to different genders. Depending on the gender they are assigned to, affixation of a noun class prefix is predictable. For instance, derived nouns in gender 1/2 will always take a nasal noun class prefix in the singular and the noun class prefix {\itshape ba}- in the plural. Nominalized participles always take a homorganic nasal prefix while nouns derived with the similative always take the {\itshape na}- prefix.

In contrast to prefixation, suffixation and tone change occuring in nominalization seem less predictable. Nouns derived with {\itshape na}- never take a suffix, nominalized participles always take the suffix -{\itshape a}, and full deverbal nouns sometimes take a suffix: As explained in \sectref{sec:NOMSuff}, the suffixes -{\itshape ɛ̀dɛ̀} and -{\itshape a} occur in deverbal nouns of different genders. Their attachment seems lexically specified.
The same is true for tonal changes in full deverbal nouns. Tonal changes can occur on full deverbal nouns in all genders except for gender 1/2.  The change affects the underlyingly toneless TBUs of a verb, namely all syllables after the first one (see \sectref{sec:toneless}). In deverbal nominalization, all the tones become lexicalized, i.e. there are no toneless TBUs in noun stems. The verbal toneless units lexicalize either as a L, as in (\ref{NomTBU1}) or a H, as in (\ref{NomTBU2}).

\begin{exe}
\ex\label{NomTBU}
\begin{xlist}
\ex\label{NomTBU1} ma-bwàl{\bfseries ɛ̀ }`birth' < bwàlɛ `give birth'
\ex\label{NomTBU2} ma-sɔ̀s{\bfseries í }`happiness' < sɔ̀si `be happy'
\end{xlist}
\end{exe}

\noindent Whether the final vowel(s) are lexicalized as a H or a L tone seems not predictable from their forms or meaning. Final H and L tones are found with any vowel quality and within the same gender. In contrast, nominalized participles undergo obligatory tone change that is predictable, as discussed in \sectref{sec:NOMPart}. 

Given the variability in suffixation and H tone change, a more systematic approach to present the data on nominalized forms in more detail is by their derivation outcome rather than the grammatical means used in the derivation. I first present full deverbal nouns that are assigned to gender 1/2, 3/4, 5/6, 6, or 7/8.    
(For more information on genders, see \sectref{sec:genders}.) Gender assignment seems largely meaning driven. For instance, deverbal agentive nouns are assigned to gender 1/2 while event nouns are found in the transnumeral gender 6. Generally, deverbal nouns are found in all major genders except for gender 9/6. I then discuss nominalized participles as a type of defective noun. Nominalized forms derived with the similative {\itshape na}- are discussed in a distinct \sectref{sec:NOMSIM} following nominalization because {\itshape na}-  not only derives nouns, but also adjectives. I also treat this type of nominalization separately because (i) nouns with {\itshape na}- only use limited nominalization means, excluding suffixation and tone change,  and (ii) their derivational source is significantly more opaque than that of other derived nouns.



\subsubsection{Deverbal agentive nouns in gender 1/2}
\label{sec:NOM12}

Deverbal nouns in gender 1/2 semantically designate a human or at least animate entity as an agent.
These agentive nouns typically describe the `doer' of an action. As animate entities, they are countable in Gyeli and thus always come with a plural form of the {\itshape ba} noun class, as described in \sectref{sec:NC}.\footnote{Nouns for humans are also found in other genders in Gyeli, but gender 1/2 is the human class in Proto-Bantu and many other Bantu languages synchronically. Also, in Gyeli most humans are assigned to gender 1/2.}

\paragraph{Prefixation} All deverbal nouns in gender 1/2 take a nasal prefix in the singular and the prefix {\itshape ba}- in the plural. The systematic attachment of a nasal prefix in the singular is remarkable since most nouns of agreement class 1 do not take any prefix at all (see \sectref{sec:NC}).
The type of nasal prefix in class 1 depends on the phonological properties of the noun's stem-initial consonant, as explained in \sectref{sec:NPlaceAss}. If the stem starts with a bilabial consonant, the nasal will be a labial nasal /m/ as in (\ref{prem}).  


\begin{exe}
\ex\label{prem} m- prefix
\begin{xlist}
\ex m-bɛ́dɔ̀ `climber' < bédɔ `climb'
\ex m-bwàlɛ̀  `parent' <  bwàlɛ `be born'
\end{xlist}
\end{exe}

\noindent On the other hand, if the consonant is an alveolar consonant, it will be an alveolar nasal /n/ as in (\ref{pren}).

\begin{exe}
\ex\label{pren} n- prefix
\begin{xlist}
\ex n-sálɛ̀ `maker' < sálɛ `make (tr.)'
\ex n-dìlɛ̀ `undertaker' <  dìlɛ `bury'
\ex  n-jì `opener' < jì `open'
%\ex add other bisyllabic               
\end{xlist}
\end{exe}

\noindent Finally, if the consonant is a velar, as in (\ref{preN}), the nasal will be a velar nasal /ŋ/.\footnote{In general orthography, however, I do not distinguish alveolar and velar nasals, as explained in \chapref{sec:Phon}.}

\begin{exe}
\ex\label{preN} ŋ- prefix
\begin{xlist}
\ex ŋ-gyàgà `buyer' <  gyàga `buy'
\ex  ŋ-kòlɛ̀ `helper' <  kòlɛ `help'
\ex ŋ-kwã́ã̀lɛ̀ `spy (n.)' <  kwã́ã̀lɛ `spy (v.)'
\end{xlist}
\end{exe}



\paragraph{Suffixation} Most deverbal nouns in gender 1/2 do not take any nominalization suffix, but retain the original verb stem, as shown in (\ref{BiN}) with the examples displaying different final vowels of /a/, /ɛ/, and /ɔ/. 


\begin{exe}
\ex\label{BiN} 
\begin{xlist}
\ex n-gyàgà `buyer' < gyàga `buy' 
\ex n-kòlɛ̀ `helper'   < kòlɛ `help'
\ex n-tsìlɔ̀ `writer' < tsìlɔ `write' % check if n-tsílɔ̀   
\ex n-jíbɔ̀ `sb. who closes' < jìbɔ `close'
\ex n-gyìmbɔ̀ `sorcerer'   < gyìmbɔ `dance'      
\end{xlist}
\end{exe}

All known deverbal nouns in gender 1/2 that do not take a nominalization suffix are bisyllabic. In the examples in (\ref{BiN}), this is obvious since the verb stem is bisyllabic as well. There are, however, also cases where a bisyllabic version of a monosyllabic verb is, at least synchronically, not used in the language, as in (\ref{Napp}). The derived noun is still bisyllabic, receiving the non-productive extension -{\itshape lɛ} which is discussed in \sectref{sec:DiaEx}. Trisyllabic derived nouns without an extension suffix are not known.

\begin{exe}
\ex\label{Napp}
\begin{xlist}
\ex  n-dèlɛ̀ `eater' < ?dèlɛ `eat (?)' < dè `eat'
\ex n-kɛ̀lɛ̀ `walker' < ?kɛ̀lɛ̀ `walk (?)' <    kɛ̀ `walk'
\end{xlist}
\end{exe}

\noindent Another opaque exception to the general retention of the verb stem is (\ref{Napp2}). Not only is the derivation process not clear, also the final vowel of the noun changes to /i/. There are no other nouns that follow this pattern.

\begin{exe}
\ex\label{Napp2}
 n-jíbí `thief'  < ? <  jíwɔ `steal'  
\end{exe}


If suffixation of deverbal nouns in gender 1/2 occurs, it is always with the suffix -{\itshape ɛ̀dɛ̀} (but never with the nominalization suffix -{\itshape a}). Examples of this are given in (\ref{ede}).

\begin{exe}
\ex\label{ede} 
\begin{xlist}
\ex n-gyámbɛ̀dɛ̀ `cook (n.)' < gyámbɔ `cook'
\ex n-gyìmbɛ̀dɛ̀ `dancer'     < gyìmbɔ `dance'       
\end{xlist}
\end{exe}

\paragraph{Tone change}  Deverbal nouns in gender 1/2 do not undergo final tone change to a H tone, unlike deverbal nouns other genders. 



\subsubsection{Deverbal nouns in gender 3/4}
\label{sec:NOM34}

Deverbal nouns in gender 3/4 are less frequent than those in gender 1/2 or 6. They are, however, formally very similar to nominalized participles, discussed in \sectref{sec:NOMPart}. All of them take a nasal prefix (in class 3), they all take the nominalization suffix -{\itshape a}, and bisyllabic nouns take a H tone on the final vowel, as shown in (\ref{34dNa}).

\begin{exe}
\ex\label{34dNa} 
\begin{xlist}
\ex n-tsìl-á `hand writing' < tsìlɔ `write'
\ex n-sàl-á `crevice' < sàlɔ `cut lengthwise'
\ex  n-lvúm-á `fork' <  lvúmɔ `sting'
\end{xlist}
\end{exe}

\noindent In contrast to nominalized participles, deverbal nouns in gender 3/4 are full nouns including a singular and a plural form with the noun class prefix {\itshape mi}-. They occur in all nominal environments, as described in \sectref{sec:N}, while nominalized participles do not.

Unlike deverbal agentive nouns of gender 1/2, deverbal nouns in gender 3/4 are not restricted to a bisyllabic pattern. As (\ref{34dNb}) shows, there are also instances of mono- and trisyllabic derived nouns. In these cases, the change to a H tone on the final vowel does not apply. 
 
\begin{exe}
\ex\label{34dNb} 
\begin{xlist}
\ex n-la ̂`story' < lâ `tell'
\ex n-sá'àwà `repeated movement (e.g.\ leaves)' < sá'àwa `move repeatedly, fidget'
\end{xlist}
\end{exe}

\subsubsection{Deverbal nouns in gender 5/6}
\label{sec:NOM56}

Deverbal nouns in gender 5/6 seem to be rare, just like those in gender 3/4. They all take the gender's noun class prefixes, {\itshape le}- in the singular class 5 and {\itshape ma}- in the plural class 6. There are no known instances of nominalization suffix attachment and nouns generally retain the final vowel of the verb, as shown in (\ref{56dNa}). 

\begin{exe}
\ex\label{56dNa} 
\begin{xlist}
\ex le-jìlɔ̀ `weight' <  jìlɔ `be heavy'
\ex le-dã̀ `pond, source, well'  <  dã̀ `draw water'
\end{xlist}
\end{exe}

\noindent An exception to the final vowel is presented in (\ref{56exc}) where the derivation path is opaque. The final vowel of the synchronically existing verb and the derived noun do not match.

\begin{exe}
\ex\label{56exc} le-sù'ù `waterfall' < ?sù'ù `pour (?)'  < sùbɛ `pour out'
\end{exe}

Deverbal nouns in gender 5/6 are either bi- or trisyllabic with the noun class prefix and a mono- or bisyllabic verb stem. There are instances where the verb stem is trysyllabic, as in (\ref{56dNb}), but in the derived noun, the first and second verb syllables are merged.

\begin{exe}
\ex\label{56dNb} le-fwálá `end, border, summit' <  fúala `end (RECIP)'
\end{exe}

\noindent The example in (\ref{56dNb}) presents the only known instance where the final vowel of the derived noun takes a H tone; all other examples retain the original surface tone of the verb stem.


\subsubsection{Deverbal event nouns in gender 6}
\label{sec:NOM6}

A vast number of deverbal nouns are assigned to the transnumeral gender 6. Semantically, deverbal nouns in this gender represent an event, as  examples in (\ref{eventN1}) through (\ref{eventN4}) show (with the exception of  {\itshape ma-nyâ} `(breast)milk' which is in this gender for its status as a liquid mass noun). 
Formally, all deverbal nouns in this gender take the noun class prefix {\itshape ma}- and are uncountable, lacking a singular counterpart in class 5.  They differ, however, with respect to suffixation of the nominalization suffix -{\itshape a} and tone behavior on the final vowel. Since these two features are independent of each other, there are four different classes of deverbal event nouns in class 6. (\ref{eventN1}) shows those that do not take a nominalization suffix nor final H tone.

\begin{exe}
\ex\label{eventN1}
\begin{xlist}
\ex ma-sâ `game, playing'  < sâ `make, do'
\ex ma-bwã̂sà `thoughts'< bwã̂sa `think'
\ex ma-nyànò `pain' < nyàno `hurt'
\ex ma-nyâ `(breast)milk' < nyâ `suckle, lick'
\ex ma-gyɛ̀'ɛ̀lɛ̀ `prayer' < gyɛ̀'ɛlɛ `pray'
\ex ma-dɔ̀ `negotiation, discussion' < dɔ̀ `negotiate, discuss'
\ex ma-bwàlɛ̀ `birth' < bwàlɛ `be born' < bwà `give birth'
\end{xlist}
\end{exe}

\noindent (\ref{eventN2}) provides an example of suffixation with the nominalization affix -{\itshape a}, but no final H tone change.

\begin{exe}
\ex\label{eventN2} ma-díg-à `vision' < dígɛ `look, watch'
\end{exe}

\noindent The third class comprises those nouns that take both the nominalization suffix -{\itshape a} and a final H tone, as in (\ref{eventN3}). (\ref{eventN3c}) further illustrates that derivation from trisyllabic verbs resulting in four-syllable nouns (including the noun class marker) is possible as well.

\begin{exe}
\ex\label{eventN3} 
\begin{xlist}
\ex ma-tál-á `beginning' < tálɛ `begin'
\ex ma-dìl-á `funeral' < dìlɛ `bury' 
\ex\label{eventN3c} ma-líbɛ́l-á `showing, apparition (of moon') < líbɛlɛ `show'
\end{xlist}
\end{exe}

\noindent Finally, there are also nouns that do not take a nominalization suffix, but a final H tone, as in (\ref{eventN4}).

\begin{exe}
\ex\label{eventN4} 
\begin{xlist}
\ex ma-pámó `appearance, rise' < pámo `appear'
\ex ma-tɛ̀mbɔ́wɔ́ `sun set' < tɛ̀mbɔwɔ `set (sun)'
\ex ma-sɔ̀sí `joy'  < sɔ̀si `be happy'
\end{xlist}
\end{exe}


\subsubsection{Deverbal nouns in gender 7/8}
\label{NOM78}

%Deverbal Result Nouns?

%Deverbal State Nouns?

Gender 7/8 also hosts deverbal nouns. They take the noun classes of their classes, namely $\emptyset$ for class 7 and {\itshape be}- in class 8. All examples presented here have a plural form, even the abstract nouns, such as {\itshape tfúgà, be-tfúgà} `suffering, sufferings' or {\itshape kwàlɛ́, be-kwàlɛ́} `love (for different things/people)'.

Within deverbal nouns of gender 7/8, there are several formal subclasses, determined by the presence or absence of a nominalization suffix and/or final H tone. Examples in (\ref{78N1}) neither take the suffix -{\itshape a} nor a final H tone, but are form identical to the verb they are derived from.

\begin{exe}
\ex\label{78N1} 
\begin{xlist}
\ex sálɛ̀ `work (n.)' < sálɛ `make, do (tr.\ v.)'
\ex tfúgà `suffering' < tfúga `suffer'
\end{xlist}
\end{exe}

\noindent In contrast, (\ref{78N2}) exhibits cases where the final tone is changed to H, but no suffixation of -{\itshape a} occurs.

\begin{exe}
\ex\label{78N2} 
\begin{xlist}
\ex sá `thing' < sâ `make, do'
\ex kwàlɛ́ `love (n.)' < kwàlɛ̀ `love (v.)'
\end{xlist}
\end{exe}

\noindent In (\ref{78N3}), both the final vowel changes to H tone and the nominalization suffix -{\itshape a} is attached.

\begin{exe}
\ex\label{78N3} tfúd-á `pinch (n.)' < tfúdɔ `pinch (v.)'
\end{exe}

\noindent And again, there are cases where the derivation process is synchronically not clear. In (\ref{78N4a}), the source of {\itshape ndɛ̀} that is attached to {\itshape kɛ̀} `walk' is unknown. In (\ref{78N4b}), it seems that there might have been another verb form from which the noun has been derived, but which does not exist synchronically anymore.

\begin{exe}
\ex\label{78N4} 
\begin{xlist}
\ex\label{78N4a} kɛ̀ndɛ̀ `walk (n.)' <  kɛ̀ `walk (v.)'
\ex\label{78N4b} lɔ̀gɔ̀ `curse (n.)' < ? <  lùà `curse (v.)' 
\end{xlist}
\end{exe}



%In most cases, the deverbal noun in class 7 retains the exact segments of the verb it is derived from. Tonal changes, however, may apply then which distinguish the noun from the verb. BE MORE EXPLICIT [???]
%Further, a few nouns extend the original verb stem as in {\itshape kɛ̀-ndɛ̀} `walk (n.)' and {\itshape sá-lɛ̀} `work (n.)'. These extensions seem to be rather verbal than nominal extensions. 

%Some deverbal nouns that are assigned to gender 7/8 lack a singular form which is based on their status as a mass noun as in (\ref{eventmass}).  

%\begin{exe}
%\ex\label{eventmass} 
%\begin{xlist}
%\ex be-déwɔ̀ `food' < dè `eat'
% \ex be-jíì `anger' < jíga `be angry'
% be-kílì `slyness, guile' < kílɔwɔ `be vigilant'
%\end{xlist}
%\end{exe}









\subsubsection{Nominalized past participles}
\label{sec:NOMPart}

The nominalized past participle is the most productive type of derivation, more productive than  full deverbal nouns or derived verbs which are discussed in \sectref{sec:VDeriv}. In the database of 377 verbs, 325 (86\%) allow for a nominalized participle.\footnote{Frequencies of derived verbs such as reciprocal, passive, or causative are provided in Table \ref{Tab:SumVext} in \sectref{sec:VDeriv}.} It seems that the only restriction for a verb not to have a nominalized past participle form is semantic in nature and includes verbs of saying or intransitive verbs such as {\itshape dyúà} `swim' or {\itshape sìsɔ} `be happy'. Grammatical properties of nominalized past participles as well as their status as nouns in terms of parts of speech are discussed in \sectref{sec:NounPart}. Semantically, they encode resultativeness, as shown in (\ref{NomPart1}).

\begin{exe}
\ex\label{NomPart1} 
\begin{xlist}
\ex n-kòl-á `(be) helped' < kòla `help (v.)'
\ex n-dvùb-á `(be) soaked' < dvùba `soak'
\ex n-gyámb-â `(be) cooked' < gyámbɔ `cook (v.)'
\ex  n-tfúmb-â `(be) wrinkled' < tfúmba `wrinkle (v.)'
\end{xlist}
\end{exe}



The derivation of nominalized participles is formally identical to that of deverbal full nouns. It involves prefixation of a nasal, suffixation of -{\itshape a}, and a tonal change on the final vowel. In contrast to deverbal full nouns (even within the same gender), these three features apply regularly on all derived forms. The tonal pattern on the suffix -{\itshape a} is determined by the tone on the first TBU, as shown in (\ref{NomPart2}) for bisyllabic verbs. If the first tone is L, the  suffix -{\itshape a} will take a H tone. If the first tone is H, the suffix will take a HL contour tone.

\begin{exe}
\ex\label{NomPart2} 
\begin{xlist}
\ex n-dvùb-{\bfseries á} `(be) soaked' < dvùbɔ `soak'
\ex m-bɔ̀g-{\bfseries á} `(be) enlargened' < bɔ̀gɛ `enlarge'
\ex n-jímb-{\bfseries â} `(be) lost' < jímbɛ `lose'
\ex n-sɛ́l-{\bfseries â} `(be) peeled' < sɛ́lɔ `peel'
\end{xlist}
\end{exe}

In fact, two syllables is the minimum requirement of length for nominalized past participles. In this, it differs from full deverbal nouns such as {\itshape n-jì} `eater' which is derived from {\itshape dè} `eat'. The nominalized participle form, however, is {\itshape n-jìy-á} `(be) open', as shown in (\ref{NomPart5}).  Monosyllabic verb stems keep their final vowel in the first syllable and attach the suffix -{\itshape a}  as the second syllable, inserting an epenthetic consonant between the two vowels. The potential epenthetic vowels mainly include {\itshape y}, {\itshape w}, and {\itshape ng} which each occur in about a third of the monosyllabic verbs; there are a few exceptional cases which take {\itshape l}, {\itshape s}, or {\itshape n}. Only the insertion of {\itshape ng} as epenthetic consonant is mostly predictable.\footnote{There are a few exceptions, e.g.\ {\itshape  má'à} `accuse' is not derived with {\itshape ng}, but with {\itshape g} in {\itshape mágâ} `(be) accused', despite the nasal. The glottal stop seems to have more weight than the nasal, but other exceptions exist as well that do not appear to have an obvious explanation for their exceptionality, for instance {\itshape nyàg-á} `(be) defecated' as derived from {\itshape nyàà} `defecate'.} It occurs in verbs that start with a nasal consonant and/or that have a nasalized vowel, as shown in (\ref{NomPart3}).

\begin{exe}
\ex\label{NomPart3} 
\begin{xlist}
\ex ndà{\bfseries ng}-á `(be) crossed' < ndà `cross'
\ex n-là{\bfseries ng}-á `(be) passed' < lã̀ `pass'
\ex n-lá{\bfseries ng}-â `(be) read' < lã̂ `read'
\ex nyí{\bfseries ng}-â `(be) entered' < nyi̊ `enter'
\end{xlist}
\end{exe}

\noindent The insertion of {\itshape g} is predictable if the monosyllabic verb contains a glottal stop. There are, however, many instances of g insertion which are not predictable, for instance in {\itshape n-tsìg-á} `(be) alive', derived from {\itshape tsìɛ̀} `live', as opposed to {\itshape n-tsíy-â} `(be) cut' which is derived from {\itshape tsíɛ̀} `cut'.

\begin{exe}
\ex\label{NomPart4} 
\begin{xlist}
\ex n-kwà{\bfseries g}-á `(be) ground' < kwà `grind'
\ex n-dvù{\bfseries g}-á `(be) hurt' < dvùɔ̀ `hurt'
\ex n-ká{\bfseries g}-â `(be) rolled up' < ká'à `roll up'
\ex m-pá{\bfseries g}-â `(be) dug out' < pá'à `dig out'
\end{xlist}
\end{exe}

\noindent Further examples of {\itshape y} insertion are given in (\ref{NomPart5}).

\begin{exe}
\ex\label{NomPart5} 
\begin{xlist}
\ex m-wɛ̀{\bfseries y}-á `(be) dead' < wɛ̀ `die'
\ex n-jì{\bfseries y}-á `(be) open' < jì `open'
\ex n-kwé{\bfseries y}-â `(be) fallen' < kwê `fall'
\ex m-vɛ́{\bfseries y}-â `(be) given' < vɛ̂ `give'
\end{xlist}
\end{exe}

As (\ref{NomPart3}) through (\ref{NomPart5}) show, the tonal patterns on the suffix is predictable as well. If the monosyllabic verb has a L tone, the derived form will have a L tone on the first and a H tone on the second syllable. If the monosyllabic verb has a HL contour tone, the first TBU of the derived form will surface H and the second HL. 

Finally, nominalized past participles can also have three syllables. In this case,  the tonal pattern of the suffix does not change according to the first TBU, but is the same for all derived forms: the second TBU surfaces H and the suffix -{\itshape a} HL, as shown in (\ref{NomPart6}).

\begin{exe}
\ex\label{NomPart6} 
\begin{xlist}
\ex m-bèl{\bfseries án-â} `(be) used' < bèlanɛ `use'
\ex n-lèb{\bfseries ál-â} `(be) followed' < lèbɛlɛ `follow'
\ex n-súm{\bfseries ál-â} `(be) greeted' < súmɛlɛ `greet'
\ex m-víy{\bfseries ál-â} `(be) touched' < víyala `touch'
\end{xlist}
\end{exe}




\subsection{Derivation with similative {\itshape na}-}
\label{sec:NOMSIM}

The similative prefix {\itshape na}- derives common and proper nouns as well as adjectives and temporal adverbs. In this, it differs from other nominalization markers discussed in \sectref{sec:NOM} which only result in the word class of common nouns. Formally, derivation with {\itshape na}- functions the same way for adjectives, common, and proper nouns,\footnote{While in most cases the derivational source is synchronically opaque, it still does not look as if there is any final vowel change to -{\itshape a} or tone change of the final vowel, as often found in deverbal nominalization.} but is tonally different in the derivation of adverbs. In all cases, the only derivation marker is the prefix {\itshape na}-. With nouns and adjectives, {\itshape na}- takes a H tone whereas it takes a L tone with derived adverbs.

Derivation with the similative marker {\itshape na}- is more diverse in its derivational source than nominalization processes discussed in \sectref{sec:NOM}. In most cases, the derivational source is, in fact, synchronically opaque.\footnote{See \sectref{sec:SIM} for why {\itshape na}- should still be viewed as a derivational morpheme.} There are some clear cases, however, where the derivational source is a noun, as for instance in the proper name {\itshape Ná-nzɛ̌} which is derived from  {\itshape nzɛ̌} `leopard'.
There are also instances where the derivational source is likely to be a diachronic stative verb that is, however, not used synchronically anymore, as with the adjectives in (\ref{naSIM1}). Especially the cross-linguistically uncommon color categories of a lightened or a darkened color suggest a change of state and make a verbal source likely.

\begin{exe}
\ex\label{naSIM1} {\itshape ná}- with adjectives
\begin{xlist}
\ex ná-vyû(vyû) `black [lit.\ like blackened]' 
\ex ná-bè(bè) `red [lit.\ like reddened]' 
\ex ná-mbàmbàlà `white [lit.\ like whitened]' 
\ex ná-yɛ̂(yɛ̂) `lightened color [lit.\ like bleached out]'
\ex ná-pfû(pfû) `darkened color [lit.\ like darkened]'
\end{xlist}
\end{exe}

\noindent Further evidence for a verbal derivation source comes from \citet[382]{cheucle2014} who analyzes the Proto-A80 particle °{\itshape na}- as a deverbal morpheme.\footnote{According to her data, °{\itshape na}- is synchronically a lot more productive in Bekwel (A85). Also colors in Bekwel are preceded by this morpheme. Cheucle views Bekwel color terms as nouns (p. 138) while the potential verbal source seems unclear.}

Nouns derived with {\itshape na}- include both common and proper nouns. As for {\itshape na}- derived common nouns, they are all in gender 1/2 and their similative prefix can be preceded by the plural noun class prefix {\itshape ba}-, as shown in (\ref{naSIM2}). As a CV- shape noun class prefix, {\itshape ba}- also then allows for the attachment of another prefix, namely the object linking H tone, as discussed in \sectref{sec:OBJTone}.  In contrast, singular noun forms with the similative marker never take a noun class prefix or  object linking H tone.  Semantically, common nouns derived with {\itshape na}- comprise mostly animals, especially insects.

\begin{exe}
\ex\label{naSIM2} {\itshape ná}- with common nouns
\begin{xlist}
%\ex ná-bàŋkúdí, ba-nábàŋkúdí `female Agama lizard'
%\ex ná-ŋkàálɛ́, ba-ná-ŋkyàálɛ́ `termite mound'
\ex ná-búnjã̂, ba-ná-búnjã̂ `bed bug'
%\ex ná-yûyû, ba-ná-yûyû `vertigo'
%\ex ná-kúlúú, ba-ná-kúlúú `forest tortoise ({\itshape Kinixys homeana})'
\ex ná-mìnsógɛ̀, ba-ná-mìnsógɛ̀ `palm rat'
\ex ná-mángɔ̀(mángɔ̀), ba-ná-mángɔ̀(mángɔ̀) `male Agama lizard'
\ex ná-yûyû, ba-ná-yûyû `vertigo'
\end{xlist}
\end{exe}

\noindent With proper nouns, {\itshape na}- only occurs in female names, deriving them from male names, as illustrated in (\ref{naSIM3}). 

\begin{exe}
\ex\label{naSIM3}  {\itshape ná}- with proper nouns
\begin{xlist}
\ex Ná-ngyɛ́mbá (female name) > Ngyɛ́mbá (male name)  
\ex Ná-ntùngù (female name)  > Ntùngù (male name)
\ex Ná-yímá (female name) > Yímá (male name)
\ex Ná-bàmù (female name) > Bàmù (male name)
\end{xlist}
\end{exe}

The prefix {\itshape na}- also derives adverbs, as shown in (\ref{naSIM4}). Unlike all other derivations with {\itshape na}-, the prefix takes a L tone with adverbs and the derivational source is always a noun.

\begin{exe}
\ex\label{naSIM4}  {\itshape nà}- with adverbs
\begin{xlist}
\ex nà-mɛ́nɔ́ `tomorrow' > mɛ́nɔ́ `morning'
\ex nà-kùgúù `yesterday' > kùgúù `evening'
\end{xlist}
\end{exe}


In terms of frequency, the prefix {\itshape na}- is found with eight common nouns in the 875 entries noun database which is less than 1\%. The similative marker is relatively more widespread amongst proper names with 16 occurrences which is a third of a sample of about 50 female proper names. The similative marker occurs with half of the 12 adjectives. This concerns all 5 color terms as well as {\itshape ná-tĩ̂} `straight'. Only two examples of derived adverbs have been found, but the class of adverbs is small in the first place.


 













\subsection {Verbal derivation}
\label{sec:VDeriv}

%[???] auxiliaries cannot be derived or exceptions?

Bantu languages are known for their multitude of productive verb extensions, also known under the term of `verbal derivation'. These suffixes  bring about a valence change from intransitive to transitive verbs and may generally include such categories as applicatives, causatives, reversives, or reciprocals.

Table \ref{Tab:SumVext} summarizes verbal derivation morphemes in Gyeli, including both extensions and expansions, while Table \ref{Tab:SumVextex} gives examples for each one. \citet{nurse08} defines extensions as verbal ``productive derivational suffixes'' that ``change the valency and meaning of [verb] roots'' (p.\ 311). In Gyeli, they comprise the forms -{\itshape ala}, -{\itshape a}, -{\itshape ɛsɛ}, -{\itshape ɛlɛ}, -{\itshape ɛga}, and -{\itshape ɔwɔ}. In contrast, the Gyeli expansions -{\itshape kɛ}, -{\itshape lɛ},\footnote{It is not clear whether this suffix is related to the applicative. As shown in \sectref{sec:DiaEx}, there are instances of valency increase, as expected for the applicative, but also cases where the opposite happens. Also, there does not seem to be a phonological rule according to which the expansion suffix could have been reduced from the applicative form. Given the inconclusive data on a potential relation between -{\itshape lɛ} and the applicative suffix -{\itshape ɛlɛ}, I consider -{\itshape lɛ} as a form in its own.} and -{\itshape bɔ} are not productive synchronically. They are low in number and, even more importantly, it is difficult to match their form onto a functional category.  


\begin{table} 
\centering
\begin{tabular}{llll}
Status & Form & Category label & \# verbs \\  \midrule
\multirow{6}{*}{\bfseries extensions} & -ala & {\scshape reciprocal} & 270  \\
& -a & {\scshape passive} & 105 \\
& -ɛsɛ & {\scshape causative} & 89 \\
& -ɛlɛ & {\scshape applicative} & 34 \\
& -ɛga & {\scshape autocausative middle voice} & 28  \\
& -ɔwɔ & {\scshape positional middle voice} & 5 \\  \midrule
\multirow{2}{*}{\bfseries expansions} & -kɛ & {\scshape ???} &  10 \\
 &  -lɛ & {\scshape ???} & 6 \\ 
 & -bɔ & {\scshape reversive} & 1  \\
\end{tabular}
\caption{Summary of verbal derivation morphemes}
\label{Tab:SumVext}
\end{table} 

% SEPARATE lɛ AND ɛlɛ OR ONE CATEGORY?

While historically the derivational system was most likely more productive, it is synchronically determined in the lexicon whether a verb takes verb extensions and, if so, which.  There is no verb that takes all possible extensions. Also, there seems to be a general tendency to reduce verb extensions. For instance, the applicative and causative are currently merging into one transitivizing category, blurring semantic distinctions. 

Gyeli verb roots generally only take one derivation morpheme which appears to correlate with the verb stem restriction to three syllables, as discussed in \sectref{sec:SyllV}. There are a few exceptions, however. Within the limits of a maximum of three syllables, a verb may combine two extensions/expansions. This is, for instance, the case with passives formed from other extensions such as the causative, applicative, or positional middle voice (see \sectref{sec:PASS}). Another exception to the trend of allowing only one derivation morpheme concerns the causative that may show (remnants of) a combination with the applicative, (\ref{CAUSAPP}), or the expansion morpheme -{\itshape lɛ}, (\ref{CAUSle}), again respecting the three syllable maximum of the verb stem. Examples such as in (\ref{CAUSAPP}) are rare. One could likewise assume that  -{\itshape s}- in (\ref{CAUSAPP}) is an epenthetic consonant, as disussed in \sectref{sec:StructVerb}. Since /s/ as an epenthetic consonant is rare as well, however, it is possible that all of these instances stem from an original causative morpheme. Synchronically, this cannot be determined with certainty. Combinations of causative and applicative morphemes in Gyeli respect the originally fixed causative-applicative suffix ordering, as discussed by \citet{good2005}.


\begin{exe}
\ex\label{CAUSAPP}
 \gll  kà-s-ɛlɛ  \\
         catch-CAUS-APPL \\
    \trans `light sth. (make sth. catch fire)'
\end {exe}

\noindent In combinations of the causative and the expansion -{\itshape lɛ}, in contrast, the expansion morpheme precedes the causative suffix, as shown in (\ref{CAUSle}).  Synchronically, it is not clear what this expansion does or what its semantic function is, as I discuss in more detail in \sectref{sec:DiaEx}. In (\ref{CAUSle}), -{\itshape lɛ} may indicate a perfective reading:\footnote{While there is definitely a difference in valency involved as well, {\itshape bwà-lɛ} `be born' does not match the passive forms discussed in \sectref{sec:PASS}.} {\itshape bwà} `give birth' → {\itshape bwà-lɛ} `be born' → {\itshape bwà-l-ɛsɛ} `make give birth'.

\begin{exe}
\ex\label{CAUSle}
 \gll  bwà-l-ɛsɛ  \\
         catch-lɛ-CAUS \\
    \trans `make give birth (e.g.\ midwife)'
\end {exe}

\noindent Some verbs lacking the bisyllabic expansion form with -{\itshape lɛ}, still use /l/  as an epenthetic consonant in the causative form, for instance in {\itshape bâ} `marry' → {\itshape bál-ɛsɛ} `make marry' (but having no form {\itshape bálɛ}). In verb forms that take two different epenthetic consonants with different derivation morphemes, one of the consonants is often /l/, which may have its origin in the expansion morpheme -{\itshape lɛ}. Extensions derived from the -{\itshape lɛ} form include passive and applicative, for example in {\itshape bû} `destroy' → {\itshape búl-a} `destroyed', while the reciprocal is formed with /y/ {\itshape búy-ala} `destroy each other'. As stated above, however, this observation does not translate into any synchronic rule and is currently lexically specified.

 As Table \ref{Tab:SumVext} shows, extension forms vary hugely in the number of verbs they combine with, which may have different causes. While categories such as causative or applicative seem to have become reduced, other extensions such as -{\itshape ɔwɔ} and -{\itshape ɛga} are restricted semantically. -{\itshape ɔwɔ} as a positional category, for instance, only combines with semantically compatible verb roots. It should also be mentioned that the numbers given in the table should not be taken as absolute. For one, despite my attempt to elicit the entire paradigm of possible extended verb forms, there is the possibility that the speaker could not think of any appropriate context and rejected a possible extended verb form on these grounds, while another speaker would have accepted a potential form. So there may actually be more forms. 


\begin{table} 
\centering
\scalebox{0.88}{
\begin{tabular}{l|lllll}
 Category  & \multicolumn{5}{l}{Example} \\  \midrule
 {\scshape reciprocal}  & lúnd-ala & `fill one another'& → & lúndɔ & `fill ()' \\
 {\scshape passive}  & lúnd-a & `be filled' & → & lúndɔ & `fill oneself' \\
 {\scshape causative} & lúnd-ɛsɛ & `make sth. full'  & → & lúndɔ & `fill oneself' \\
 {\scshape applicative} &  lúnd-ɛlɛ & `fill sth.' & → & lúndɔ & `fill oneself'  \\
 {\scshape autocausative}  & vìd-ɛga & `turn (by itself)' & → & vìdɛ & `turn sth.' \\
 {\scshape positional}  & kɛ̀l-ɔwɔ & `assume hanging position' & → & kɛ̀lɛ & `hang sth.' \\  \midrule
 {\scshape -kɛ}  & jí-kɛ & `burn sth.' & → & jíyɛ & `burn (intr.)' \\
 {\scshape -lɛ} & bwà-lɛ & `be born' & → & bwà & `give birth' \\ 
 {\scshape -bɔ/wɔ}  &   jì-bɔ & `close' & → & jì & `open' \\
\end{tabular}}
\caption{Examples of verbal derivation morphemes}
\label{Tab:SumVextex}
\end{table} 


\noindent Another issue concerns verb forms that have an extension or expansion, but no synchronic underived form. I treat them as underived forms here, i.e.\ I do not count them as extensions in the table in order to be consistent across categories. While it is easy to recognize, for instance, a causative or applicative form, it is much harder for possible expansions such as -{\itshape kɛ}.  As indicated in Table \ref{Tab:SumVext}, there are 10 instances of this morpheme serving as an expansion to an underived form. There are, however, 5 instances in my database where a -{\itshape kɛ} ending appears as an apparent underived form itself, taking yet its own extension morphemes. Synchronically, it is not possible to determine whether this -{\itshape kɛ} carries any morphological function or whether it is simply a random lexical form. 
Table \ref{Tab:SumVextex}, as a summary, provides examples of each extension and expansion category, including the underived verb form.

In the following, I will describe each derivation morpheme and its semantic functions in a decreasing order of frequency. As discussed in \sectref{sec:Tonology}, all derivation morphemes are underlyingly toneless. Therefore, they are represented without tonal marking here.


\subsubsection{Reciprocal \textit{-ala}}
\label{sec:REC}

The verb extension -{\itshape ala} is by far the most frequent in Gyeli. Out of 377 verbs in the database, 270 (71.6\%) allow for this extension which I label as reciprocal. Further, there are eight occurrences of verb stems ending in -{\itshape ala} that do not have an underived form.

In terms of the extension's semantic function, it has mostly a reciprocal meaning, as the examples in (\ref{RECIP}) show, which express `mutuality'.


\begin{exe} \ex \label{RECIP}
\begin{tabular}{lllll}
dvùɔ̀ & `hurt (intr.)' & → & dvùg-ala & `hurt one another' \\
dyúwɔ & `hear' & → & dyúw-ala & `understand each other' \\
gyíwɔ & `call' & → & gyíw-ala & `call each other' \\
kwàlɛ & `love' & → & kwàl-ala & `love each other' \\
tsíndɔ & `push' & → & tsínd-ala & `push each other' \\
bâ & `marry' & → & bán-ala & `marry each other' \\
kɛ̃̀ & `shave' & → &  kèng-ala & `shave each other' \\
\end{tabular}
\end{exe} 


Beyond this reciprocal meaning, there are many instances of verbs whose semantics do not allow for a reciprocal use. In these cases, the extension -{\itshape ala} has a `togetherness' reading, as shown in (\ref{togetherness}).

\begin{exe} \ex \label{togetherness}
\begin{tabular}{lllll}
nyùlɛ & `drink' & → & nyùl-àlà & `drink together' \\
kɔ́sɛ & `cough' & → & kɔ́s-ala & `cough together' \\
pámɔ & `show up' & → & pám-ala & `show up together' \\
tɛ́bɔ & `get up' & → & tɛ́b-ala & `get up together' \\
bwà & `become big' & → & bɔ̀g-ala & `become big together' \\
kwê & `fall' & → & kwéy-ala & `fall together' \\
nyî & `enter' &  → & nyíng-ala & `enter together' \\
\end{tabular}
\end{exe} 

It is possible that verbs which do allow a reciprocal meaning may get a `togetherness' reading, depending on the context. This, however, needs further investigation.





\subsubsection{Passive \textit{-a}}
\label{sec:PASS}

I will discuss the contrast between active and passive constructions following \posscitet{siewierska2013} defining criteria for passive constructions which I illustrate in (\ref{passdef}).

\begin{exe}
\ex\label{passdef}
\begin{xlist}
\ex \label{passdef1}
  \glll  bùdì bá tsìlɔ́ békálàdɛ̀.\\
	b-ùdì ba-H tsìlɔ-H H-be-kálàdɛ̀ \\
             ba2-person 2-PRES write-R OBJ.LINK-be8-book  \\
    \trans `People write books.'
\ex\label{passdef2}
 \glll  bèkálàdɛ̀ bé tsìl{\bfseries á} (nà bùdì). \\
	be-kálàdɛ̀ be-H tsìl-a-H nà b-ùdì \\
         be8-book 8-PRES write-{\bfseries PASS}-R COM ba2-person  \\
    \trans `Books are written (by people).'
\end{xlist}
\end{exe}


\noindent  (\ref{passdef1}) is the active, while (\ref{passdef2}) is the contrasting passive construction. According to \citet{siewierska2013}, ``the subject of the active corresponds to a non-obligatory oblique phrase of the passive or is not overtly expressed,'' which is the case for the subject {\itshape bùdì} in (\ref{passdef1}). Another characteristic of passive constructions is that their subjects correspond to the direct object in the active counterpart, as with {\itshape bèkálàdɛ̀} `books'. Siewierska also points out that passive constructions are pragmatically more restricted than active constructions, which is true in Gyeli as well. Finally, she notes that passive constructions receive a special morphological marking of the verb. In the case of Gyeli, this is a final vowel -{\itshape a}, in most cases, as will be discussed below.

Generally, passive forms are far less frequent than reciprocals, with only 105 attested instances (27.9\% of the verbs in the database). Speakers appear to prefer the active form with the impersonal third person plural of class 2 and are forced to use this for the majority of verbs which do not have a passive form.  Morphological marking of the passive on the verb in Gyeli differs phonologically, depending on the syllable number of the verb form the passive is derived from. Passives from mono- and bisyllabic roots differ from trisyllabic ones. I will discuss both in turn.

\paragraph{Passive formation from mono- and bisyllabic roots} The passive in Gyeli is formed by the extension -{\itshape a}, resulting in a bisyllabic verb stem if it is derived from a mono- or bisyllabic verb root, as shown in (\ref{passive1}).


\begin{exe} \ex \label{passive1}
\begin{tabular}{lllll}
kwàlɛ & `love' & → & kwàl-a & `be loved' \\
bvúɔ̀ & `break sth.' & → & bvúg-a & `be broken' \\
jì & `open' & → & jìy-a & `be open' \\
dyû & `kill' & → & dyúw-a & `be killed' \\
jíwɔ & `steal' & → & jíy-a & `be stolen' \\
vìdɛ & `turn sth.' & → & vìd-a & `be turned' \\
bàwɛ & `carry sth.' & → &  bàw-a & `be carried' \\
\end{tabular}
\end{exe} 


All these instances have an underived form. There are, however, 36 bisyllabic verbs ending in -{\itshape a} which are underived, non-passive forms. Examples are given in (\ref{passive2}). In fact, these verbs cannot be passivized nor do they have a passive meaning. Expressing passive meaning as in (\ref{passive1}) is not possible for them since their ending is identical with the passive suffix.


\begin{exe} \ex \label{passive2}
\begin{tabular}{ll}
gyàga & `buy'  \\
kòla & `add'  \\
kìya & `give'  \\
bwàndya & `despise' \\
\end{tabular}
\end{exe} 

For other bisyllabic verb stems ending in -{\itshape a} which do not have an underived form, agentivity is less specified. The examples in  (\ref{passive3}) can be thought of as having a non-specified agent while the subject takes the semantic role of an experiencer.


\begin{exe} \ex \label{passive3}
\begin{tabular}{ll}
vòwa & `wake up'  \\
wùsa & `forget'  \\
káka & `shiver'  \\
kánda & `crack (intr.; e.g.\ bottle or glass)' \\
sìya & `wash, bathe sb./oneself'  \\
\end{tabular}
\end{exe}

Finally, a few bisyllabic passive forms take a final -{\itshape ɛ} rather than the usual passive -{\itshape a} extension, as shown in (\ref{passive1a}) which lists all known instances. 

\begin{exe} \ex \label{passive1a}
\begin{tabular}{lllll}
bwè & `catch' & → & bùl-ɛ & `be caught' \\
sàlɔ & `cut lengthwise' & → & sàl-ɛ & `be cut lengthwise' \\
tìnɔ & `harvest tubers' & → & tìl-ɛ & `be harvested (tubers)' \\
\end{tabular}
\end{exe} 

\noindent These exceptions are specified in the lexicon rather than stemming from a predictable morpho-phonological rule. Their origin and/or motivation is not clear at this point.


\paragraph{Passive formation from trisyllabic stems} In a few rare cases, the passive can also be formed from trisyllabic stems, i.e.\ from verbs which already have an extension such as the causative, applicative, or positional middle voice. In these cases, not only the final vowel changes to -{\itshape a}, but also that of the second syllable, as shown in (\ref{passive1b}). The passive forms that are derived from applicatives -{\itshape ɛlɛ} are identical with the reciprocal forms. I do not mark morpheme breaks with a hyphen for these passive forms since morpheme boundaries are not clear-cut. Rather, an extension morpheme such as -{\itshape awa} has to be considered a portmanteau morpheme, encoding both the passive via the vowels /a/ and the positional via the consonant /w/. 

\begin{exe} \ex \label{passive1b}
\begin{tabular}{lllll}
bál-ɔwɔ & `bend down' & → & bálawa & `be bent down' \\
bén-ɛlɛ& `raise, lift sth.' & → & bénala & `be lifted (lift each other)' \\
bùm-ɛlɛ & `hit (nail)' & → & bùmala & `be hit (hit each other)' \\
dyɔ̀l-ɛsɛ & `make laugh' & → & dyɔ̀lasa & `be made to laugh' \\
pín-ɛsɛ & `squeeze' & → & pínasa & `be squeezed' \\
\end{tabular}
\end{exe} 

Historically, the passive extension is likely to have developed from the middle voice suffix -{\itshape aga} which is still used in Mabi as passive. In Gyeli, the velar stops got lost and the vowel contracted. In careful speech, the final -{\itshape a} is sometimes still lengthened, for instance in {\itshape gyàmbaa} `be cooked' which is derived from {\itshape gyámbɔ} `cook',  but in fast speech and most lexemes, it surfaces as a short vowel.

The use of passive verbs is rather restricted, nevertheless. For one, many underived verbs do not allow for passivization, even though this would semantically be possible. Also, in terms of text frequency, even verbs that do have a passive form are rarely used.\footnote{The passive forms discussed in this section stem mainly from elicitations.} In natural speech, the Bagyeli prefer to use an active construction with a class 2 (3\textsuperscript{rd} person plural) subject as an agent which remains semantically unspecified, as in (\ref{Passba}).

\begin{exe}
\ex\label{Passba}
 \glll bá gyàgá má-ntúà  \\
	ba-H gyàga-H H-ma-ntúà \\
         2-PRES buy-R OBJ.LINK-ma6-mango  \\
    \trans `They buy the mangos (= the mangos are bought).'
\end {exe}

\noindent See also \sectref{sec:IS} on information structure for a more detailed discussion.


\paragraph{Relation to other derivation forms} The passive appears to be related to two other derivation forms: the autocausative and the nominalized past participle. The passive could be the shortened form of the autocausative -{\itshape aga}, discussed in \sectref{sec:AutoCaus}. As explained there, -{\itshape aga} is the regular passive suffix in Mabi. In Gyeli, it appears to have split into two categories: the passive and the autocausative. This can be seen in a few instances where the passive suffix is a lengthened vowel, as in (\ref{PASS1}). It seems lexically specified whether a verb can take the lengthened passive form. In any case, the lengthened suffix is in free variation with the default short form.

\begin{exe} \ex \label{PASS1}
\begin{tabular}{lllll}
kfúdɛ & `cover' & → & kfúd-a(a) & `be covered' \\
wàwɛ & `spread' & → & wàw-a(a) & `be spread' \\
gyámbɔ & `cook' & → & dyúg-a(a) & `be cooked' \\
kwɛ̀lɔ & `cut down' & → & kwɛ̀l-a(a) & `be cut down' \\
\end{tabular}
\end{exe} 

\noindent In a likely scenario, the consonant /g/ has been deleted from -{\itshape aga}, developing into a lengthened passive form which still exists in a few lexemes while the synchronic default form is a short vowel.

Semantically, the shift from autocausative middle voice to passive seems natural. In both cases, the agent is not overtly expressed. The main difference seems to concern the attribution of agentivity. In the autocausative, the subject has a certain degree of agentivity, while, in the passive, the subject is clearly the patient.  Given the distinct functions of passive and autocausative, quite a few verbs take both extensions. This is true for all examples in (\ref{PASS1}); others are listed in Appendix \ref{sec:AppendixI}.

The passive form is also related to the nominalized past participle described in \sectref{sec:NOMPart}. The difference between the two  is both structural and semantic. The passive verb form is preceded by a STAMP marker, as in (\ref{PASS}), while the nominalized past participle requires the STAMP copula (as discussed in \sectref{sec:COP}) that agrees with the subject, as shown in (\ref{NomPart}). 

\begin{exe}
\ex\label{PASS}
 \glll yí kɛ̀là  \\
	yi-H kɛ̀l-a \\
         7-PRES hang-PASS  \\
    \trans `It is being hung.'
\end {exe}

\begin{exe}
\ex\label{NomPart}
 \glll yíì nkɛ̀lá  \\
         yíì n-kɛ̀l-a-H \\
	COP NOM-hang-PASS-NOM \\
    \trans `It has been hung [lit. It is a `hung-up one'].'
\end {exe}

\noindent The meaning difference between the two constructions is in fact aspectual. The passive construction views an event as ongoing while the nominalized form is more resultative.%\footnote{The English translation does not do these constructions justice in terms of their tense marking which is both present. The German translation gets closer to the tense translation, opposing `Es wird aufgehängt.' for the passive form and `Es ist aufgehängt.' for the nominalized form.}


\subsubsection{Causative \textit{-ɛsɛ}}
\label{sec:CAU} 

The causative extension morpheme -{\itshape ɛsɛ} changes the argument structure of the verb in that it increases the verb's valency, turning intransitive verbs into transitive and transitive verbs into ditransitive ones. \citet{song2013} defines causative constructions as denoting complex situations
\begin{quote}
``consisting of two component events [...]: (i) the {\bfseries causing event}, in which the {\bfseries causer} does or initiates something; and (ii) the {\bfseries caused event}, in which the {\bfseries causee} carries out an action, or undergoes a change of condition or state as a result of the causer’s action.''
\end{quote}
This definition becomes clearer when looking at (\ref{Causdef}) where the causer, {\itshape Màmbì}, does something, namely teaching which is the causing event. As a consequence, the causee, {\itshape Àdà}, does something, namely learning English which is the caused event.

\begin{exe}
\ex\label{Causdef}
 \glll Màmbì á gyíkɛ́sɛ́ Àdà ngɛ̀lɛ́nɛ̀  \\
	Màmbì a-H gyík-ɛsɛ-H Àdà ngɛ̀lɛ́nɛ̀ \\
         $\emptyset$1.PN 1-PRES learn-CAUS-R $\emptyset$1.PN $\emptyset$1.English  \\
    \trans `Mambi teaches Ada English (lit. makes Ada learn English).'
\end {exe}

This type of morphological causative, as opposed to lexical and syntactic causatives (see \citet[3]{song96}), is marked on the verb by a suffix. The morphological causative is not the only causative construction found in Gyeli. Also syntactic causatives using the verb {\itshape sâ} `make' plus the complementizer {\itshape nâ}, as in (\ref{CAUS1}), are quite common.

\begin{exe} 
\ex\label{CAUS1}
  \glll     mɛ́ nzíí sâ nâ wɛ́ dyɔ̀\\
	mɛ-H nzíì-H sâ nâ wɛ-H dyɔ̀ \\
              1\textsc{sg}-PRES PROG-R make COMP 2\textsc{sg}-PRES laugh  \\
    \trans `I make you laugh.'
\end{exe}

The morphological causative in Gyeli is formed by the suffix -{\itshape ɛsɛ}. 89 verbs in the database (23.6\%) have a causative suffix. There are another 6 verbs with a causative ending which do not  have an underived form. Examples are provided in (\ref{CAUS}).

\begin{exe} \ex \label{CAUS}
\begin{tabular}{lllll}
gìyɔ & `cry' & → & gìl-ɛsɛ & `make cry' \\
gyímbɔ & `dance' & → & gyímb-ɛsɛ & `make dance' \\
dyúwɔ & `hear, perceive' & → & dyúg-ɛsɛ & `make feel' \\
nyâ & `suckle, lick' & → & nyáng-ɛsɛ & `breast-feed' \\
mìno & `swallow' & → & mìn-ɛsɛ & `make swallow' \\
jíyɔ & `burn (intr.)' & → & jíg-ɛsɛ & `make angry' \\
lùnga & `grow (intr.)' &  → & lùng-ɛsɛ & `raise, make grow' \\
gyíkɛ & `learn' & → & gyík-ɛsɛ & `teach' \\
\end{tabular}
\end{exe} 

Some medial consonants of underived verb forms are subject to change in verbal derivation. This is precisely the case with epenthetic consonants such as /w/ (between /u/ and /ɔ/) and /y/ (between /i/ and /ɔ/) which may be replaced by another consonant in the derived forms. In this respect, bisyllabic underived verbs behave parallel to monosyllabic roots, as discussed in \sectref{sec:StructVerb} for stem final vowels.

% talk in phonology about vowel change de dilese

While in the great majority of cases, the suffix -{\itshape ɛsɛ} expresses causativity, there are a few cases where the semantic lines between causative and applicative are blurred, as for instance with the verb {\itshape dvùbɔ} `dip, soak'.  For these, both the underived verb can be used, as in (\ref{causative1}), or the causative, as in (\ref{causative2}).

\begin{exe}
\ex\label{causative}
\begin{xlist}
\ex \label{causative1}
  \glll  mɛ́ dvùbɔ́ pɛ̀mbɔ̀ ɛ́ kɔ̀fí \\
	mɛ-H dvùbɔ-H pɛ̀mbɔ̀ ɛ́ kɔ̀fí \\
              1\textsc{sg}-PRES dip-R $\emptyset$1.bread LOC $\emptyset$7.coffee \\
    \trans `I dip the bread in coffee.'
\ex\label{causative2}
 \glll  mɛ́ dvùbɛ́sɛ́ wɛ̂ màjíwɔ́ \\
	mɛ-H dvùb-ɛsɛ-H wɛ̂ ma-jíwɔ́ \\
         1\textsc{sg}-PRES dip-CAUS-R 2\textsc{sg} ma6-water  \\
    \trans `I dip you in water.'
\end{xlist}
\end{exe}

The distribution and frequency of the underived versus the causative form needs further investigation. The occurrence of comparable cases in the corpus is so rare that no generalizations can be made at this point.


\subsubsection{Applicative \textit{-ɛlɛ}}
\label{secLAPP}


The extension -{\itshape ɛlɛ}  is significantly rarer in Gyeli than the causative -{\itshape ɛsɛ}, with only 34 (9\%) instances in the database. Further, there are no verbs ending in -{\itshape ɛlɛ} that have no underived form. I refer to the -{\itshape ɛlɛ} suffix as `applicative', a category that is commonly found in Bantu languages.

Morphosyntactically, the applicative changes the verb's valency by increasing ``the number of object arguments selected by the predicate [...] by one with respect to the basic construction'' (Polinsky 2013). \citet[278]{peterson97} specifies that, in applicative constructions:
\begin{quote} ``thematically peripheral objects are treated in a more core or direct object manner, and in terms of discourse, they often have higher relative topicality in applicative constructions as compared to when they occur in non-applicative constructions.'' \end{quote}

\noindent Gyeli forms applicatives both from intransitive (\ref{APPintr}) and transitive (\ref{APPtr}) verbs, which seems to be the typical case in Bantu languages, according to \citet{polinsky2013}. 

\begin{exe} \ex \label{APPintr}
\begin{tabular}{lllll}
nyùmbɔ & `smell (intr.)' & → & nyùmb-ɛlɛ & `smell sth.' \\
swásɔ & `dry (intr.)' & → & swás-ɛlɛ & `dry sth.' \\
bédɔ & `go up' & → & béd-ɛlɛ & `mount sth.' \\
lúndɔ & `fill oneself' & → & lúnd-ɛlɛ & `fill sth.' \\
sɔ́'ɔ̀ & `continue' & → & sɔ́s-ɛlɛ & `continue with sth.' \\
jímbɛ & `get lost' & → & jímb-ɛlɛ & `lose sth.' \\
bámɔ & `scold (intr.)' &  → & bám-ɛlɛ & `scold sb.' \\
dyũ̂ & `be hot' & → & dyúng-ɛlɛ & `heat sth.' \\
\end{tabular}
\end{exe} 

\noindent Further, \citet{polinsky2013} distinguishes applicative constructions in terms of the semantic role of the applied object, pointing out that Bantu languages typically licence benefactive and other semantic roles. This is also true for Gyeli. Benefactive contexts usually arise with applicatives formed from transitive verbs, for instance as shown in (\ref{APPtr}) for {\itshape gyámbɔ} `prepare'. In these cases, a second object is added which often takes the role of a benefactive or an instrumental. 

\begin{exe} \ex \label{APPtr}
\begin{tabular}{lllll}
lúmɛ & `send' & → & lúm-ɛlɛ & `send to sb.' \\
gyámbɔ & `prepare' & → & gyámb-ɛlɛ & `prepare for sb.' \\
dyúwɔ & `hear, perceive' & → & dyúw-ɛlɛ & `listen' \\
vísɔ & `cover' & → & vís-ɛlɛ & `cover sth. {\tiny (+ INSTR/BEN)}' \\
kfùbɛ & `provoke' & → & kfùb-ɛlɛ & `provoke sb. {\tiny (+ INSTR/BEN)}' \\
vídɛ & `turn sth.' & → & víd-ɛlɛ & `turn sth. {\tiny (+ INSTR/BEN)}' \\
\end{tabular}
\end{exe} 

Applicatives which are derived from intransitive verbs typically do not have a benefactive reading. In fact, they differ significantly in the distribution of semantic roles across arguments from applicatives that are derived from transitive verbs. The subject of the intransitive verb, which has the role of an undergoer, is expressed as the object in the applicative form, as shown in (\ref{AppCau}). In many of these instances, the applicative forms have a causative meaning.

\begin{exe} \ex \label{AppCau}
\begin{tabular}{lllll}
vásɛ & `rise (dough)' & → & vás-ɛlɛ & `make (dough) rise' \\
vè'è & `try on clothes' & → & vè'-ɛlɛ & `make sb. try clothes on' \\
kɔ́sɛ & `cough' & → & kɔ́s-ɛlɛ & `make cough' \\
\end{tabular}
\end{exe} 

In contemporary speech, the applicative and the causative seem to be merging into one category, with the applicative most likely becoming lost, given its lower frequency in comparison to the causative.  It is rare that a verb has both an applicative and a causative form. In my database, I  found only 5 instances where a verb takes both -{\itshape ɛsɛ} and -{\itshape ɛlɛ}. In the majority of cases, a verb has a causative, but no applicative form.
%{\todo think about applicative merge, not the same function}

It is not surprising that these two categories are merging since, semantically, there is some overlap between them. For instance, the applicative form {\itshape nyíngɛlɛ} `insert', derived from {\itshape nyî} `enter',  may be viewed as adding an applied object to the underived verb form. On the other hand, semantically, it can also be thought of as a causative context in the sense of `making sth.\ enter'. The same is true for {\itshape dyû} `be hot' which has an applicative form {\itshape dyúng-ɛlɛ} `heat sth.' Again, an object is added to an otherwise intransitive verb, resulting in a reading of `applying heat to sth.' At the same time, semantically, it can also be thought of as `make sth. hot'.\footnote{\citet{bostoen2011} report a similar syncretism of applicative and causative for Mbuun (Bantu B87). According to them, however, the syncretism in Mbuun is based on phonological rather than semantic grounds.}

Just like the causative, also the applicative extension has a periphrastic alternative to convey the same, or at least similar, meaning, as shown in (\ref{APPperi}).

\begin{exe} 
\ex\label{APPperi}
  \glll     mɛ́ gyá gyá mpá'à wɔ̂ \\
	mɛ-H gyâ-H gyá mpá'à w-ɔ̂ \\
              1\textsc{sg}-PRES sing-R $\emptyset$7.song $\emptyset$3.side 3-2\textsc{sg}.POSS   \\
    \trans `I sing a song for you.'
\end{exe}



\subsubsection{Autocausative middle voice \textit{-ɛga/-aga}}
\label{sec:AutoCaus}

The extension -{\itshape ɛga}/-{\itshape aga} appears 28 times in the verb database which means that 7.4\% of the verbs allow this extension. Further, there are 4 verbs with this extension which have no synchronic underived form. 

In contrast to other extensions, this derivation has two variant suffixes: -{\itshape ɛga} and -{\itshape aga}. A specific verb will only take one of the two forms, i.e.\ it is not possible for a given verb to use either one or the other. The choice for one of the two suffix forms seems to be lexically specified rather than depending on phonological rules. Even though there is a tendency that -{\itshape aga} is used after the glide /j/ (`y' in orthography) as well as after /m/ or /mb/, there are also a few cases where -{\itshape ɛga} appears after these consonants. Given that their form is very similar while the function is the same, I consider these two suffixes as belonging to the same category.  It is possible that the form -{\itshape aga} has its origin in the neighboring language Mabi where the suffix is used productively for passive formation. This, however, does not explain why -{\itshape ɛga} is used for some and -{\itshape aga} for other verbs and how the existing distribution comes about. In terms of frequency, -{\itshape ɛga} is found more often than its variant -{\itshape aga}, the latter appearing only nine times in contrast to -{\itshape ɛga} with 19 times.

The suffix variants -{\itshape ɛga} and -{\itshape aga} constitute one of two middle voice categories in Gyeli. I distinguish, in terms of terminology, the autocausative middle voice extension -{\itshape ɛga}/-{\itshape aga} from the `positional' middle voice suffix -{\itshape ɔwɔ}, discussed in \sectref{sec:PosVerbs}. Unlike valency-increasing extensions, such as the applicative or causative, the middle voice constitutes a category `intermediate  in  transitivity  between  one-participant  and  two-participant  events', as defined by \citet[3]{kemmer93}.\footnote{Note that \citet{kemmer93} primarily defines the middle voice as a semantic category which, in some languages, receives formal marking. I deviate from this notion in that I consider middle voice categories in Gyeli as formal categories which map onto certain functions.} In Gyeli, the autocausative middle voice typically denotes one-participant events. It requires only one argument (the subject), having a valency decreasing effect. 
The autocausative, as exemplified in (\ref{Autocaus}), is accordingly intransitive, derived from transitive verbs. Semantically, the subject of autocausative verbs incorporates the roles of both agent and undergoer, while syntactically the agent remains under-specified. Often, a certain self-causation is implied in such events which I translate as `by itself'.


\begin{exe} \ex \label{Autocaus}
\begin{tabular}{lllll}
vìdɛ & `turn (tr.)' & → & vìd-ɛga & `turn (by itself)' \\
wàwɛ & `spread sth.' & → & wàw-ɛga & `spread (by itself)' \\
jìna & `dive' & → & jìn-ɛga & `sink (intr), melt (intr)' \\
kfúdɛ & `cover sth.' & → & kfúd-ɛga & `cover (by itself)' \\
lɛ̀ndo & `flow' & → & lɛ̀nd-ɛga & `flow (by itself) \\
lɛ́gɛ & `singe' & → & lɛ́g-ɛga & `singe (by itself) \\
tfúmbɔ & `wrinkle sth.' & → & tfúmb-aga & `get wrinkled (by itself)' \\
líyɔ & `clear land' & → & líy-aga & `clear (by itself)' \\
\end{tabular}
\end{exe} 

Cross-linguistically, there seems to be a strong relation between middle voice and reflexive constructions. \citet{
kemmer93} even assumes  that middle marking evolves from reflexive constructions. Speakers indeed tend to translate autocausative middle voice forms with a French reflexive construction using {\itshape se}, for example {\itshape tfúmb-aga} `get wrinkled (by itself)' would be translated as {\itshape se plier} in French. Nevertheless, I argue that the autocausative in Gyeli constitutes a basic system which is not derived from reflexive constructions. This view is parallel to \posscitet{maldonado2009} observation on South American languages where middle voice also is a basic system independent of reflexives. 

In comparison to the autocausative suffix, Bantu reflexives are canonically expressed by an affix preceding the stem, which \citet[109]{meeussen67} calls `infix' and  reconstructs as *-ᶖ́-  (-jᶖ-? -jᶖᶖ-?)  for Proto-Bantu. Such a prefix is not found in Gyeli. Reflexivity in Gyeli is rather expressed by object pronouns plus {\itshape mɛ́dɛ̀} `self' as in (\ref{reflex}) or, in other cases, carry reflexive meaning lexically as in {\itshape síya} `wash (oneself)'.


\begin{exe} 
\ex\label{reflex}
  \glll     mɛ́ nyɛ́ mɛ̂ mɛ́dɛ̀ \\
	mɛ-H nyɛ̂-H mɛ̂ mɛ́dɛ̀ \\
              1\textsc{sg}-PRES see-R 1\textsc{sg}.NSBJ self   \\
    \trans `I see myself.'
\end{exe}

Given these constructions which differ formally very much from the autocausative, there is no obvious reason to assume that they are related or even that the autocausative has evolved from the reflexive. On the other hand, the autocausative is structurally more similar to the passive in Mabi, which has the extension -{\itshape aga} or may even be related to the passive extension -{\itshape a(a)} in Gyeli itself. This is discussed in more detail in \sectref{sec:PASS}


\subsubsection{Positional middle voice \textit{-ɔwɔ}}
\label{sec:PosVerbs}

The extension -{\itshape ɔwɔ} constitutes the second type of middle voice category in Gyeli. -{\itshape ɔwɔ} is the least frequent verb extension in Gyeli with a total of 15 occurrences, 11 of which are part of the 377 verb database while four have not been considered for this database. Out of the 11 occurrences within the database, only six (1.6\%) are used productively in the sense that they have synchronically an underived verb form.  

I label this category as ‘positional middle voice’ since almost\footnote{The one known exception to posture reference is the verb {\itshape bwèd-ɔwɔ} `be tasty/sweet'. } all verbs with this extension describe the event of assuming a position, as illustrated in (\ref{Positional}). 

\begin{exe} \ex \label{Positional}
\begin{tabular}{llllp{4cm}}
kɛ̀lɛ & `hang sth.' & → & kɛ̀l-ɔwɔ & `assume a hanging position' \\
kfúdɛ & `cover sth.' & → & kfúd-ɔwɔ & `lie down by covering head with arms' \\
kwádɔ & `twist sth.' & → & kwád-ɔwɔ & `assume a crooked position' \\
ngwáwɔ & `bend sth.' & → & ngwáng-ɔwɔ & `bend (intr.)' \\
pwásɔ & `flatten sth.' & → & pwás-ɔwɔ & `assume a flatened position, stretch out' \\
\end{tabular}
\end{exe}

\noindent The same is true for verbs of this ending which do not seem to have a synchronic underived form, as exemplified in (\ref{Positional2}).


\begin{exe} \ex \label{Positional2}
\begin{tabular}{ll}
bál-ɔwɔ & `bend down' \\
kwàng-ɔwɔ & `lie down on side' \\
gyí-ɔwɔ & `lean back' \\
pwàngy-ɔwɔ & `lie down stretched out (French: {\itshape s'allonger})'  \\
sɛ̀ngy-ɔwɔ & `assume inclined position' \\
\end{tabular}
\end{exe}

\citet[75]{schadeberg2003} uses the term `positional' for a stative category that talks about ‘assuming a position’ or ‘being in a position’. He reconstructs °{\itshape -am-} as the positional extension for PB which differs significantly in the segmental material {\itshape -ɔwɔ} in Gyeli. Nevertheless, both forms seem to carry the same meaning.

Schadeberg does not consider the derivation °{\itshape -am-} in PB as middle voice. He mentions, however, that this extension is known to have become a passive suffix in certain Bantu languages of zone C (cf.\ \citet[76]{schadeberg2003}). For languages such as Gyeli and Mabi, it seems that passive forms are more related to the autocausative middle voice category, as described in \sectref{sec:AutoCaus} and \sectref{sec:PASS}.


\paragraph{Passivization of the positional} A few positional forms can further be derived to passive forms by substituting the two final vowels /ɔ/ by the passive vowel /a/, as shown in (\ref{PosPass}).\footnote{Passive forms of the positional middle voice were not given for all positional verb forms. Given that passive forms are generally restricted and less frequent than logically possible, it seems that the same is true for passives of positional forms rather than assuming that these are gaps in the data, which in particular instances might be the case.} 


\begin{exe} \ex \label{PosPass}
\begin{tabular}{llllp{4cm}}
bál-ɔwɔ & `bend down' & → & bál-awa & `be bent down' \\
pwàs-ɔwɔ & `stretch out' & → & pwás-awa & `be stretched out' \\
\end{tabular}
\end{exe}

\paragraph{Middle voice categories in comparison} Comparing both middle voice categories, the autocausative and the positional, they do not only differ in their extension forms, but also in their distribution of admissible subjects, and their semantics. Subjects of the positional middle voice are typically human, at least animate, while the autocausative allows both animate and inanimate subjects. Very often, however, subjects of autocausative verb forms are inanimate, given that they incorporate the role of an undergoer which for many transitive verbs such as {\itshape kfúdɛ} `cover' or {\itshape lɛ́gɛ} `singe' is typically inanimate.

In terms of semantics, the agent in autocausative forms is underspecified, implying a certain self-causation which is possibly more metaphorical than real. For instance, when using the form {\itshape wàw-ɛga} `spread (by itself)' with a subject such as `seeds', this is generally understood as `the seeds spread by themselves'. In reality, they are probably spread by the wind or some other agent such as animals which is not salient enough to deserve mentioning. Thus, the subject is treated as the agent, even though this might not be the case in the world. In contrast, the agent of positional verb forms is always identical with the subject.


A verb can have both middle voice forms. Given the low frequency of forms of both middle voice categories, there are not many examples, but one is the verb {\itshape kwádɔ} `twist' which has both the autocausative {\itshape kwád-ɛga} `get twisted, twist by itself' and the positional {\itshape kwád-ɔwɔ} `assume a twisted, curved position'.  The autocausative typically has an inanimate subject, for instance a rope or a net, while the positional form has a human subject. Further, this verb has a passive form {\itshape kwád-a} `be twisted'. Table \ref{Tab:VerbScale} shows the whole range of possible agent specifications in Gyeli.


\begin{table} 
\begin{tabular}{lp{3cm}ll}
Transitive → &   Positional → &    Autocausative → &    Passive \\  \midrule
two participants &  agent=SBJ & agent=SBJ implied &  agent=non-SBJ \\  \midrule
kwádɔ &  kwádɔwɔ & kwádɛga  & kwáda  \\
`twist sth.' &  `assume twisted position' &  `get twisted' &  `be twisted' \\

\end{tabular}
\caption{Scale of decreasing expression of agentivity}
 \label{Tab:VerbScale}
\end{table}



\subsubsection{Expansions}
\label{sec:DiaEx} 

Expansions, in contrast to extensions, are not productive. They are low in frequency and do not have an obvious core function. Gyeli has three expansion suffixes which I will discuss in turn.

\paragraph{{\bfseries -kɛ/gɛ}}

The expansion suffix -{\itshape kɛ} or its weakened form -{\itshape gɛ} is found ten times in the database as a derivation from an underived verb form. Further, five other verbs in the database show this ending, all of which are transitive and do not have an underived intrasitive form.

This suffix has different effects for different verbs which is lexically specified. In most instances, the suffix -{\itshape kɛ} is valency inceasing, turning an intransitive verb into a transitive one, as shown in (\ref{ketr}).\footnote{Some verbs with a sequence of /wa/ or /ua/ in their underived form change to /ɔ/ in the derived form, as with {\itshape bwà} `become big' changing to {\itshape bɔ̀kɛ} `make big'. Whether this change happens is lexically specified and not a general phonological rule since there are verbs with the same sequences which do not change to /ɔ/, for example {\itshape bwà} `be born' having the derived form {\itshape bwà-lɛ} `be born'.} %[ADD TO PHONOLOGY SECTION]}


\begin{exe} \ex \label{ketr}
\begin{tabular}{llllp{4cm}}
bwà & `become big' & → & bɔ̀-kɛ & `make sth. big' \\
kàgɔ & `promise (intr.)' & → & kà-gɛ & `promise (tr.)' \\
lṹã̀ & `whistle' & → & lɔ̃́n-gɛ & `whistle sth.' \\
tɛ́'ɛ̀ & `be soft' & → & tɛ́-gɛ & `soften sth.' \\
tɔ̀à & `boil (intr.)' & → & tɔ̀-kɛ & `boil sth.' \\
bô & `lie down (intr.)' & → & bú-gɛ & `lie sth. down' \\
\end{tabular}
\end{exe}


In another case, which may be purely an exception, the inverse happens and the expansion -{\itshape kɛ} serves as a valency decreasing suffix, as in (\ref{keintr}).

\begin{exe} \ex \label{keintr}
\begin{tabular}{llllp{4cm}}
bvúɔ̀ & `break sth.' & → & bvú-kɛ & `break (intr.)' \\
\end{tabular}
\end{exe}


For the majority of instances where the suffix -{\itshape kɛ} has a valency increasing effect, this is semantically linked to a causative meaning, for instance in examples such as {\itshape bɔ̀-kɛ} `make big' or {\itshape tɛ́-gɛ} `soften sth.'. The -{\itshape kɛ} expansion is, however, distinct from the standard causative -{\itshape ɛsɛ}, and not an allomorph, as cases of verbs show which have both suffixes.
 For instance, the verb {\itshape jíyɛ} `burn (intr.)', as shown in (\ref{kecaus}), allows -{\itshape kɛ} as a valency increasing expansion {\itshape jígɛ} `burn sth'. Also, the causative form {\itshape jí-g-ɛsɛ} is found with the figurative meaning `make sb. angry'. 


\begin{exe} \ex \label{kecaus}
\begin{tabular}{llllp{4cm}}
jíyɛ & `burn (intr.)' & → & jí-gɛ & `burn sth.' \\
 &  & → & jí-g-ɛsɛ & `make sb.\ angry' \\
dvùɔ̀ & `hurt (intr.)' & → & dvù-gɛ & `hurt sb.' \\
 &  & → & dvù-g-ɛsɛ & `make sb. hurt' \\
\end{tabular}
\end{exe}

An alternative analysis to the suffixes -{\itshape kɛ/gɛ} and -{\itshape lɛ} would be to assume an expansion -{\itshape ɛ} which takes different epenthetic vowels /g/ and /l/, as described in \sectref{sec:StructVerb}. Under this view, /g/ in {\itshape jíg-ɛ} `burn sth.'  would be treated as a root final epenthetic consonant. Given the tendency of a distinct causative function with the the expansion -{\itshape kɛ/gɛ} which is not found with -{\itshape lɛ}, I analyze -{\itshape gɛ/kɛ} and -{\itshape lɛ} as distinct expansion morphemes rather than assuming one expansion -{\itshape ɛ} with different epenthetic consonants.


\paragraph{{\bfseries -lɛ}}

Another non-productive suffix is -{\itshape lɛ} with only 6 derived forms in the database. -{\itshape lɛ} is a frequent ending of bisyllabic verbs, however; 21 underived bisyllabic verbs end in this syllable. It is, however, uncertain whether this is a phonologically wide-spread syllable in verbs or whether diachronically there was a productive extension morpheme -{\itshape lɛ}.

As with the suffix -{\itshape kɛ/gɛ}, it is difficult to pinpoint -{\itshape lɛ}'s function. Often, it seems to be valency increasing, transitivizing an intransitive verb form, as in (\ref{letr}).


\begin{exe} \ex \label{letr}
\begin{tabular}{llllp{4cm}}
vû & `leave' & → & vú-lɛ & `get rid of sth.' \\
jí(yɔ) & `sit, live' & → & jí-lɛ & `seat sb.' \\
tɛ́-bɔ & `rise' & → & tɛ́-lɛ & `place sth. upright' \\
\end{tabular}
\end{exe}

In other cases, however, the -{\itshape lɛ} suffix more seems to have a passivizing function, as in (\ref{lepass}). Usually, passivization is achieved by the passive morpheme -{\itshape a}. In these two cases, however, no such form is available and rather the -{\itshape lɛ} suffix is used.


\begin{exe} \ex \label{lepass}
\begin{tabular}{llllp{4cm}}
bwà & `give birth' & → & bwà-lɛ & `be born' \\
tìnɔ & `harvest tubers' & → & tì-lɛ & `be harvested' \\
\end{tabular}
\end{exe}

Given these different uses of -{\itshape lɛ}, it is not possible to provide a unified category label for this expansion.


\paragraph{{\bfseries -bɔ/wɔ}}

Finally, another frequent suffix is the expansion -{\itshape wɔ/bɔ} used in bisyllabic verbs. With only two derived forms and eight verbs without an underived form the database provides few examples. This, again, makes it difficult to make generalizations about its function. It is tempting to assume a reversive category when considering (\ref{reverse}). 


\begin{exe} \ex \label{reverse}
\begin{tabular}{llllp{4cm}}
jì & `open sth.' & → & jì-bɔ & `close sth.' \\
\end{tabular}
\end{exe}

Other examples, however, do not support this hypothesis, but rather suggest that in some cases at least, -{\itshape bɔ/wɔ} has a detransitivizing effect, as in (\ref{wointr}).\footnote{In the two first cases, it is hard to specifiy which form is the derived and which is the underived form since both verbs have an expansion morpheme, but there is no monosyllabic form without derivation morpheme.}


\begin{exe} \ex \label{wointr}
\begin{tabular}{llllp{4cm}}
sɔ̀-lɛ & `hide sth.' & → & swà-wɔ & `hide (intr.)' \\
tɛ́-lɛ & `place sth.' & → & tɛ́-bɔ & `rise' \\
láà & `tell sth.' & → & là-wɔ & `speak' \\
\end{tabular}
\end{exe}


\subsection{Zero-derivation}
\label{sec:ZeroDer}

Zero-derivation is found in only a few domains.
Almost all postpositions are zero-derived from nouns, as shown in Table (\ref{Tab:ZeroDeriv1}).\footnote{The only unclear case is the postposition {\itshape dé} `in' for which a possible nominal source is synchronically not known.} Postpositions and their source noun do not differ in form, but in their morphosyntactic behavior and distribution, as explained in \sectref{sec:LOCgen}.

\begin{table} 
\centering
\begin{tabular}{lll}
Lexeme & Postposition & Nominal source \\  \midrule
sí & `under, down' & `ground' \\
 dyúwɔ̀ &  `up, on top' & `sky' \\
tɛ́mɔ́ &  `between' & `middle' \\
písɛ̀ &   `behind' & `behind, back (n.)' \\
sɔ̂ & `in front, before' & `front (n.)' \\
\end{tabular}
\caption{Derivation of postpositions}
\label{Tab:ZeroDeriv1}
\end{table}

In the absence of any derivational marking, one might object that it is difficult to pinpoint the grammaticalization path from noun to postposition or vice versa. The phenomenon that locative adpositions are derived from body-part and environmental landmark nouns, however, has been observed by, for instance, \citet[215]{kiessling2008} for African languages and \citet{bowden1992} for Oceanic languages. It is rather noteworthy that, in Gyeli, these expressions are grammaticalized as postpositions instead of prepositions, as would be expected for Bantu languages \citep{dryer2013b}.

Another potential case of zero-derivation includes the quantifier {\itshape bvùbvù} `many' and its nominal counterpart {\itshape bvúbvù} `multitude' (cl. 9). In this case, however, there is a difference in the tonal pattern. Since this is the only example, it is not clear, however, if the tonal difference marks derivation or happened by chance. It is further not clear whether the noun is the source or the derived form. 









\subsection{Compounding}
\label{sec:Compound}

In comparison to derivation, compounding is a less productive word formation strategy.
Gyeli has two types of compound nouns which differ in their derivation source and complexity. Most compounds are formed from a nominalized verb and its nominal complement. A few compounds are derived from two underived nouns. Both types are discussed in the following.

\subsubsection{Deverbal noun-noun compounds}
\label{sec:VNCompound}


The most productive type of compounding is comprised of a nominalized verbal root and a noun, as illustrated in (\ref{CompoundTemp}). Most nominal compounds semantically designate an agent, as shown in (\ref{CompoundVN1a}). Accordingly, the verbal root is nominalized as a deverbal noun of gender 1/2, as described in \sectref{sec:NOM12}.\footnote{A more detailed discussion of compounding in Bantu, especially in Bemba, is provided in \citet{basciano2011}.}

\begin{exe}
\ex\label{CompoundTemp} [N\textsubscript{deverbal} + N]\textsubscript{N}.
\end{exe}

The noun that follows the nominalized verb is the verb's direct argument that cannot be omitted as the nominalized verb of these constructions on its own is ungrammatical. The complement noun, however, is ``not necessarily [an object] in the traditional syntactic sense'' \citep{schadeberg2003}. The tonal pattern of a deverbal compound, as illustrated in (\ref{CompoundVN1a}), differs from the patterns found in a verb phrase between verb and object, as discussed in \sectref{sec:SynH} and \sectref{sec:HLinker}. In a VP, the noun class prefix of the nominal argument takes an object linking H tone and the final vowel of the verb takes a H tone in realis categories.  In compounds, all these TBUs surface with a L tone though. 

\begin{exe}
\ex\label{CompoundVN1a} 
\begin{xlist}
\ex mbòmɛ̀-{\bfseries mà}pɔ̂ `messenger' \\ < bòmɛ `bark, announce' + ma-pɔ̂ `news'
\ex ntsíɛ̀-{\bfseries bè}nyàgà `butcher' \\ < tsíɛ̀ `cut' + be-nyàgà `cows'
\ex nlɔ́lɛ̀-{\bfseries mì}nkɔ̀lɛ́  `weaver, tailor' \\ <  lɔ̂ `sew, weave' + mi-nkɔ̀lɛ́ `threads'
\ex ngyàgɛ̀sɛ̀-{\bfseries bè}sâ `vendor, merchant' \\ < gyàg-ɛsɛ `make buy' + be-sâ `things'
\ex mbwálɛ̀sɛ̀-{\bfseries b}ùdì `midwife' \\ < bwà(l)-ɛsɛ `make give birth' + b-ùdì `people'
\ex nlímbɔ́-{\bfseries mà}mbɔ̀ `connoisseur, educated person' \\ <  límbo `know' + ma-mbɔ̀ `things'
\ex nsálɛ̀-{\bfseries mà}nkɛ̃̂ `farmer' < sá-lɛ `do (tr.)' + ma-nkɛ̃̂ `fields'
\end{xlist}
\end{exe}

The tonal difference between objects in a VP and complement nouns in a compound can be explained by the compounds' lexicalization history. Rather than stemming from a nominalized VP, these compounds have their origin in a N + N attributive construction, as discussed in \sectref{sec:CONC}, whose first constituent is a deverbal agentive noun. This is in line with \citet[87]{schadeberg2003} who points out that compound ``nouns may originate from a genitival (connexive) [attributive] construction'' which then become lexicalized as noun, as shown in (\ref{CompoundVN}). 

\begin{exe}
\ex\label{CompoundVN} 
\gll  °mbòmɛ̀ wà màpɔ̂  $\rightarrow$ mbòmɛ̀ $\emptyset$ màpɔ̂  $\rightarrow$    mbòmɛ̀-màpɔ̂  \\
 	m-bòmɛ̀ wà ma-pɔ̂ \\
	N1-announce 1:ATT ma6-news 
\glt `messenger [lit. announcer of news $\rightarrow$ news-announcer]'
\end{exe}

\noindent Even in many synchronic attributive constructions, the attributive marker can optionally be omitted, as discussed in \sectref{sec:CONOM}. In deverbal compounds, the omission of the attributive marker is no longer optional, but has become lexicalized. This lexicalization path explains why the prefix of the complement has a L tone rather than an object linking H tone. Since the preceding attributive marker {\itshape wà} has a L tone, the following prefix surfaces L as well (in contrast to the plural version shown in (\ref{CompoundVN1b})).   Another piece of evidence for lexicalization from an attributive construction comes from the plural formation of these compounds explained below. 


There are two types of  compounds which differ in the number of the argument nominal which is either plural or singular/transnumeral noun. In (\ref{CompoundVN1a}), all argument nouns are plural, marked by the plural noun class prefixes in bold.
The number of the argument nominal has an impact on the plural formation of the compound noun. If the argument noun has a plural prefix, as in (\ref{CompoundVN1a}), its plural counterpart does not constitute a compound noun, but a N + N attributive construction. (\ref{CompoundVN1b}) shows the plural forms of the examples in (\ref{CompoundVN1a}). They are comprised of the plural nominalized verb, the plural argument noun and an attributive marker agreeing with the first noun that links the two constituents.\footnote{I represent the noun class prefix of the nominalized verb as toneless which will take its surface tone from its syntactic environment. While the CV- noun class prefix of the second constituent is underlyingly toneless as well, it surfaces with a H tone which it acquires through high tone spreading from the preceding attributive marker.}

\begin{exe}
\ex\label{CompoundVN1b} 
\begin{xlist}
\ex ba-bòmɛ̀ bá má-pɔ̂ `messengers' 
\ex ba-tsíɛ̀ bá bé-nyàgà `butchers'
\ex ba-lɔ́lɛ̀ bá mí-nkɔ́lɛ̀ `weavers, tailors'
\ex ba-gyàgɛ̀sɛ̀ bá bé-sâ `vendors, merchants'
\ex ba-bwálɛ̀sɛ̀ bá b-ùdì `midwives'
\ex ba-límbɔ́ bá má-mbɔ̀ `connoisseurs, educated people'
\ex ba-sálɛ̀ bá má-nkɛ̃̂ `farmers'
\end{xlist}
\end{exe}

\noindent The structural difference between singular compound nouns and their non-compound plural counterparts is due to their different stages in lexicalization. As described in \sectref{sec:CONOM}, attributive markers can be omitted from N + N constructions under certain morphophonological and semantic conditions. Two plural noun constituents and a CV- shape noun class prefix on the second constituent, however, inhibit the omission of the attributive marker,  explaining why the singular form is more lexicalized as a noun than its plural counterpart.

The second and less frequent type of deverbal compounds has a singular or transnumeral argument noun, as illustrated in (\ref{CompoundVN2a}).

\begin{exe}
\ex\label{CompoundVN2a} 
\begin{xlist}
\ex nkẽ̀-nlô `gecko'\footnote{It is believed that geckos eat people's hair while they are sleeping.} \\ < kẽ̀ `shave' + nlô `head'
\ex mbúlɔ̀-mã̂ `fisherman' \\ < búlɔ `fish (v.)' + mã̂ `sea'
\end{xlist}
\end{exe}

\noindent In these cases, the plural counterpart remains a compound as well, as shown in (\ref{CompoundVN2b}). Rather than transforming into a N + N attributive construction, the compound only takes a plural noun class prefix for the nominalized verb while the second constituent remains unchanged. It thus appears that compounds with singular second constituents are more lexicalized than those with plural second constituents.

\begin{exe}
\ex\label{CompoundVN2b} 
\begin{xlist}
\ex ba-nkẽ̀-nlô `geckos'
\ex ba-búlɔ̀-mã̂ `fishermen'
\end{xlist}
\end{exe}

As mentioned above, most compounds of the [VN] type constitute agent nouns. The only exception to this pattern I found is given in (\ref{CompoundVN3}). Though it is still in gender 1/2, it lacks the nasal prefix in the singular. 

\begin{exe}
\ex\label{CompoundVN3} tsíɛ̀-sámɛ̀, ba-tsíɛ̀-sámɛ̀ `circumcision' \\ < tsíɛ̀ `cut' + nsámbɔ̀ `penis'
\end{exe}

\noindent Having a singular second constituent, the plural form remains a compound noun. The phonologically changed form of the argument nominal suggests that this compound is further along the lexicalization path.

\subsubsection{Underived noun-noun compounds}
\label{sec:NNCompound}

The second category of nominal compounds take the structure of N + N compounds. They differ from deverbal compounds in that their constituents are not derived.  The most common lexical items involved in [NN] compounds include {\itshape mwánɔ̀} `child'  as a diminutive marker, as shown in (\ref{CompoundNN1}). Semantically, the diminutive can refer both to the small size of a referent or a small amount.

\begin{exe}
\ex\label{CompoundNN1} 
\begin{xlist}
\ex mwánɔ̀-mùdã̂ `girl' mwánɔ̀ `child' + mùdã̂ `woman'
\ex mwánɔ̀-mùdũ̂ `boy' < mwánɔ̀ `child' + mùdũ̂ `man'
\ex mwánɔ̀-nlàwɔ́ `twig' < mwánɔ̀ `child' + nlàwɔ́ `branch'
\ex mwánɔ̀-sâ `little something' < mwánɔ̀ `child' + sâ `thing'
\end{xlist}
\end{exe}

Pluralization of such compounds requires both constituents to occur in their plural form, as shown in (\ref{CompoundNN1a}).

\begin{exe}
\ex\label{CompoundNN1a} 
\begin{xlist}
\ex bwánɔ̀-bùdã̂ `girls' 
\ex bwánɔ̀-bùdũ̂ `boys'
\ex bwánɔ̀-mìnlàwɔ́ `twigs'
\ex bwánɔ̀-besâ `little things' 
\end{xlist}
\end{exe}

In diminutive compounds, the second constituent serves as the syntactic and semantic head. As such, agreement targets agree with the second constituent and not with the first, as shown in (\ref{CompoundNNAGR}).

\begin{exe} 
\ex\label{CompoundNNAGR}
\begin{xlist}
\ex\label{CompoundNNAGR1}
 \glll  bwánɔ̀-békúmbé {\bfseries bé} bà njí nà byɔ̂ {\bfseries bé} tɛ́lɛ́ mà{\bfseries bé}  \\
         b-wánɔ̀-be-kúmbé bé ba njì-H nà by-ɔ̂ be-H tɛ́lɛ-H mà-bé \\
          ba2-child-be8-tin 8:ATT 2.PST1 come-R COM 8-NSBJ 8-PRES stand-R here-8   \\
    \trans `The few tin roofs that they brought stand here.'
\ex\label{CompoundNNAGR2}
 \glll  *bwánɔ̀-békúmbé {\bfseries bá} bà njí nà byɔ̂ {\bfseries bá} tɛ́lɛ́ mà{\bfseries bá}  \\
         b-wánɔ̀-be-kúmbé bé ba njì-H nà by-ɔ̂ be-H tɛ́lɛ-H mà-bé \\
          ba2-child-be8-tin 8:ATT 2.PST1 come-R COM 8-NSBJ 8-PRES stand-R here-8   \\
    \trans `The few tin roofs that they brought stand here.'
\end{xlist}
\end{exe}


Underived noun-noun compounds other than diminutives seem to describe an inherent property, such as gender or size, as shown in (\ref{CompoundNN2}). As with deverbal [NN] compounds, these compounds appear to originate in attributive constructions.   

\begin{exe}
\ex\label{CompoundNN2} 
\begin{xlist}
\ex sɔ́-mùdã̂ `female friend' < sɔ́ `friend' + mùdã̂ `woman'
\ex kfúbɔ̀-dyá `tall chicken' < kfúbɔ̀ `chicken' + dyá `length'
\end{xlist}
\end{exe}

\noindent There seems to be a lexicalization scale from attributive constructions which require the attributive marker, as described in \sectref{sec:CONC}, those which optionally omit the attributive marker, and finally those where the omission of the attributive marker is lexicalized in a way that the attributive-less construction, as in (\ref{CompoundNN2}), differs in meaning from the one with an attributive marker, as illustrated in (\ref{CompoundNN3}). I only view the latter type as compounds. Since examples with such a meaning contrast are hard to find, examples of these compounds are few in number.

\begin {exe} \ex\label {CompoundNN3}
\begin{xlist}
\parbox[t]{2.5in}{ \ex \gll
sɔ́ wà m-ùdã̂  \\ 
$\emptyset$1.friend 1:ATT N1-woman \\
`the friend of the woman' }
\parbox[t]{2.5in}{ \ex \gll
kfúbɔ̀ wà dyá \\
   $\emptyset$1.chicken 1:ATT $\emptyset$1.length \\
 `the remote chicken'}
\end{xlist}
\end {exe}

Impressionistically, [NN] compounds in (\ref{CompoundNN1}) differ structurally from the diminutive compounds in (\ref{CompoundNN2}) with respect to their headedness. In the diminutives, the semantic and syntactic head is the second constituent, while, in the other compounds, the first constituent functions as the head. The first constituent as head would be expected from the compounds' origin in a N + N attributive construction where the first consituent is the head as well. Given the limitation of examples, it is not possible at this point to explain the switch of headedness in diminutives.


\paragraph{Note on absent derivation phenomena} I conclude this chapter with a note on another derivation type common across Bantu languages, namely noun-to-noun derivation. As \citet[82]{schadeberg2003} describes, noun-to-noun derivation is commonly achieved by shifting nouns to different genders. I have not  observed this in my Gyeli data. Instead, Gyeli has different lexical stems or diminutive compounds to encode, for instance, size differences that may be expressed in different genders in other Bantu languages. 
