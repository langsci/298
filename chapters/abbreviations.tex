For notation conventions, I use the Leipzig Glossing Rules. These may differ from abbreviations typically used in the lexicon. Abbreviations in the lexicon are generally in small characters ending in a dot while most abbreviations in glosses (except for noun class labels)  are represented in capital letters.


\begin{multicols}{2}

\noindent * \hfill ungrammatical form  \\
%.  \\
%: \\
°\hfill reconstructed form  \\
( )\hfill  element in brackets is optional   \\
- \hfill morpheme boundary \\
<  \hfill derived from \\
$\emptyset$ \hfill prefixless noun class\\
1-9 \hfill  agreement class 1-9   \\
%1nc  \hfill noun class 1    \\
1P  \hfill first person plural    \\
2P  \hfill second person plural    \\
1S  \hfill first person singular    \\
2S  \hfill second person singular    \\
ADJ \hfill adjective \\
ADV \hfill adverbial clause \\
adv.  \hfill adverb    \\
AGR  \hfill agreement    \\
ANA \hfill anaphoric marker \\ 
AP  \hfill associative plural   \\
appl. \hfill applicative \\
ATT  \hfill attributive marker   \\
autoc. \hfill autocausative \\
AUX \hfill auxiliary \\
ba  \hfill {\it ba}- noun class \\
be \hfill {\it be}- noun class \\
C \hfill consonant    \\
caus. \hfill causative \\
CF \hfill citation form    \\
cl. \hfill agreement class    \\
COM \hfill comitative marker \\
COMP \hfill complement clause \\
COMPL \hfill absolute completive \\
COND \hfill conditional clause \\
CONTR \hfill contrastive marker \\
COP \hfill SCOP copula \\
%CON  connective morpheme    \\
DEM \hfill  demonstrative    \\
DIST \hfill  distal    \\
DO \hfill  direct object    \\
EMPH \hfill emphatic pronoun \\
EXCL \hfill exclamation \\
EXP \hfill expansion \\
EXT \hfill extension \\
FUT \hfill  future   \\
H  \hfill high tone    \\
HAB \hfill habitual \\
HL  \hfill falling contour tone    \\
HORT \hfill hortative \\
HTS \hfill high tone spreading \\
ID \hfill identificational marker \\
IDEO \hfill ideophone \\
IMP \hfill   imperative    \\
INCH \hfill  inchoative    \\
INF \hfill infinitival clause \\
intr. \hfill intransitive \\
inv. \hfill invariable \\
IO \hfill  indirect object    \\
L \hfill  low tone    \\
le \hfill {\it le}- noun class \\
LH \hfill  raising contour tone    \\
LOC \hfill  locative    \\
ma \hfill {\it ma}- noun class \\
mi \hfill {\it mi}- noun class \\
N  \hfill nasal; {\it N}- noun class \\
n.  \hfill noun   \\
NC \hfill nasal + consonant \\
NCA \hfill non-complete accomplishment \\
NEG  \hfill negation    \\
NOM  \hfill nominalization    \\
NP  \hfill noun phrase   \\
npp.  \hfill nominalized past participle   \\
num. \hfill numeral \\
O  \hfill onset   \\
OBJ \hfill object \\
OBJ.LINK \hfill object linker \\
pass. \hfill passive \\
PL  \hfill plural marker    \\
pl. \hfill plural \\
%\columnbreak
PN  \hfill proper name    \\
POS \hfill part of speech \\
posit. \hfill positional \\
POSS \hfill possessive    \\
PRES \hfill present    \\
PRF \hfill perfect \\
PRED \hfill predicate \\
PRIOR \hfill priorative \\
PROG \hfill progressive    \\
PROSP \hfill prospective \\
PROX \hfill proximal    \\
PST1  \hfill recent past   \\
PST2  \hfill remote past    \\
Q  \hfill question particle    \\
QI \hfill quotative index \\
Q(tag) \hfill question tag \\
%qual. \hfill qualifier \\
R \hfill realis mood \\

RD \hfill reported discourse \\
recip. \hfill reciprocal \\
REL \hfill relative clause \\
RETRO \hfill retrospective \\
S  \hfill singular    \\
SBJ \hfill subject \\
SBJV  \hfill subjunctive    \\
%QUOT   \hfill quotative marker    \\
%STAMP  \hfill subject-tense-aspect-mood-polarity    \\
SEQU \hfill sequential marker \\
sg. \hfill singular \\
stat. \hfill stative \\
SUB \hfill subordinate \\
TBU  \hfill tone bearing unit    \\
TM \hfill tense-mood \\
tr. \hfill transitive \\
V \hfill vowel   \\
v. \hfill verb   \\
VOT \hfill voice onset time \\
X \hfill oblique \\
\end{multicols}
