%\pagebreak
\chapter*{\center Acknowledgments}
\label{sec:acknowledgments}
\addcontentsline{toc}{chapter}{Acknowledgments}

This grammar would not have been possible without the many Gyeli speakers I have worked with over the years and who patiently taught me about their language. I am especially grateful to the people of Ngolo, in particular to Mama David, Ada Joseph, Mambi, Nandtoungou, Nze, Tsimbo, Nkolo Dorothée, Segyua, `Délégué' Bikoun, Tata, and Aminu.

Thanks also to my Kwasio assistants and friends Bimbvoung Emmanuel Calvin, Djiedjhie François, and Nouangama Severin who did not only help with interpreting, translations, and annotations, but who made my life in the field so much easier and enjoyable. Thanks for always being around and taking care of me when I was sick with malaria and chikungunya.
I am also particularly grateful to my fellow team members Daniel Duke and Emmanuel Ngue Um and our cameraman Christopher Lorenz.

My fieldwork was funded by the VolkswagenFoundation grant `84976' and a generous extension phase `87014'  within the DoBeS (Documentation of Endangered Languages) Initiative. I am grateful for the opportunity the grants gave me and for all the assistance, especially by Mrs. Szöllosi-Brenig.

I would like to thank my advisors Tom Güldemann and Maarten Mous for their helpful feedback throughout the course of writing my dissertation and beyond when revising it for publication.  I have discussed many aspects of this grammar with various people over the last years. I particularly thank Viktoria Apel, Pierpaolo Di Carlo, Ines Fiedler, Hana Filip, Jeff Good, Larry Hyman, Lutz Marten, Joyce McDonough, and Murray Schellenberg.

Last, but not least, I am very grateful to my family and friends 
who supported me in the field and took active interest in all the news I brought from Cameroon. Special thanks to my wonderful husband Scott for his patience with the long absences that fieldwork makes necessary, for sharing my excitement and worries, and for proof-reading this grammar.